\documentclass[14pt,a4paper,fleqn]{extarticle}
\usepackage[T2A,T1]{fontenc}
\usepackage[utf8]{inputenc}
\usepackage[russian]{babel}
\usepackage{amsmath}
\usepackage{graphicx}
\usepackage{tabularx}
\usepackage{boldline}
\usepackage{makecell}

\graphicspath{ {./images/} }
\setlength{\mathindent}{0pt}
\setlength\parindent{0pt}
\begin{document}
	\begin{titlepage}
		\includegraphics[scale=0.12]{logo}
		\begin{center}
			\textbf{МИНОБРНАУКИ РОССИИ}\\
			\vspace{0.2cm}
			\textbf{Федеральное государственное бюджетное образовательное учреждение высшего образования}\\
			\textbf{«САНКТ-ПЕТЕРБУРГСКИЙ ГОСУДАРСТВЕННЫЙ ЭКОНОМИЧЕСКИЙ УНИВЕРСИТЕТ»}\\
			\vspace{0.6cm}
			Факультет информатики и прикладной математики\\
			Кафедра прикладной математики и экономико-математических методов\\
			\vspace{1cm}
			\textbf{ДОКЛАД}\\
			по дисциплине:\\
			\textbf{«Математическое моделирование»}\\
			на тему:\\
			\textbf{«Аттрактор Рёсслера»}\\
		\end{center}
		\vspace{1cm}
		Направление: 01.03.02\\
		Обучающийся: Бронников Егор Игоревич\\
		Группа: ПМ-1901\\
		\vfill
		\begin{center}
			Санкт-Петербург\\
			2021\\
		\end{center}
	\end{titlepage}
\section*{Вступление}
\textit{Аттрактор Рёсслера} -- это аттрактор, которым обладает система дифференциальных уравнений Рёсслера:
\begin{align*}
	\begin{cases}
		\frac{dx}{dt} = -y - z\\
		\frac{dy}{dt} = x + ay \hspace{5cm} , a > 0, b > 0, r > 0\\
		\frac{dz}{dt} = b + z(x - r)\\
	\end{cases}
\end{align*}
Эта система дифференциальных уравнений описывает модель динамики химических реакций, протекающих в некоторой смеси с перемешиванием.\\
При значениях параметров $a = b = 0.2$ и $2.6 \leq r \leq 4.2$ уравнения Рёсслера обладают устойчивым предельным циклом.\\
Сам Рёсслер изучал систему при постоянных $a = 0.2$, $b = 0.2$ и $r = 5.7$, но также часто используются и значения $a = 0.1$, $b = 0.1$, и $r = 14$.\\
Также для аттрактора Рёсслера харатерна фрактальная структура в фазовой плоскости, т.е. явление самоподобия. Важно,  что  на  аттракторе 
Рёсслера  траектории  не  пересекают  сами  себя.
\section*{Анализ поведения системы на плоскости}
При $z = 0$ система примет следующий вид:
\begin{align*}
	\begin{cases}
		\frac{dx}{dt} = -y\\
		\frac{dy}{dt} = x + ay \hspace{5cm} , a > 0, b > 0\\
	\end{cases}
\end{align*}
Поэтому устойчивость движения в плоскости $z = 0$ определяется собственными значениями -- $\lambda_{1,2} = \dfrac{a \pm \sqrt{a^2-4}}{2}$.\\
Когда 0 < $a$ < 2, собственные значения имеют положительную вещественную часть и комплексно сопряжены. Поэтому фазовые траектории расходятся от начала координат по спирали.
\newpage
Теперь проанализируем изменение координаты $z$, считая $0 < a < 2$. Пока $x$ меньше $r$, множитель $x - r$ в уравнении на $\dfrac{dz}{dt}$ будет удерживать траекторию близкой к плоскости $x$, $y$. Как только $x$ станет больше $r$, $z$ -- координата начнёт расти. В свою очередь, большой параметр -- $z$ начнёт тормозить рост $x$ в $\dfrac{dx}{dt}$.
\section*{Неподвижные точки}
\begin{align*}
	\begin{cases}
		\frac{dx}{dt} = -y - z = 0\\
		\frac{dy}{dt} = x + ay = 0 \hspace{5cm} , a > 0, b > 0, r > 0\\
		\frac{dz}{dt} = b + z(x - r) = 0\\
	\end{cases}
\end{align*}
Решая следующую систему, мы получим, что у нас есть 2 неподвижные точки:\\
$\begin{pmatrix}
	\dfrac{r + \sqrt{r^2 - 4ab}}{2}, & \dfrac{-r - \sqrt{r^2 - 4ab}}{2a}, & \dfrac{r + \sqrt{r^2 - 4ab}}{2a}\\
\end{pmatrix}$\\\\
$\begin{pmatrix}
	\dfrac{r - \sqrt{r^2 - 4ab}}{2}, & \dfrac{-r + \sqrt{r^2 - 4ab}}{2a}, & \dfrac{r - \sqrt{r^2 - 4ab}}{2a}\\
\end{pmatrix}$\\\\
Как видно на изображении проекции аттрактора Рёсслера, одна из этих точек расположена в центре спирали аттрактора, а другая находится далеко от неё.
\section*{Изменение параметров}
Поведение аттрактора Рёсслера сильно зависит от значений постоянных параметров. Изменение каждого параметра даёт определённый эффект, в результате чего в системе может возникнуть устойчивая неподвижная точка, предельный цикл или решения системы станут уходить на бесконечность.
\newpage
\noindent\makebox[\linewidth]{\rule{\paperwidth}{0.4pt}}
Бифуркационные диаграммы являются стандартным инструментом для анализа поведения динамических систем, в том числе и аттрактора Рёсслера. Они создаются путём решения уравнений системы, где фиксируются две переменные и изменяется одна. При построении такой диаграммы получаются почти полностью «закрашенные» регионы; это и есть область динамического хаоса.\\
\noindent\makebox[\linewidth]{\rule{\paperwidth}{0.4pt}}
\subsection*{Изменение параметра $a$}
Зафиксируем $b = 0.2, r = 5.7$ и будем изменять параметр $a$.
В итоге опытным путём получили такие результаты:
\begin{itemize}
	\setlength\itemsep{0.0001mm}
	\item $a \leq 0$ -- сходится к устойчивой точке;
	\item $a = 0.1$ -- крутится с периодом 2;
	\item $a = 0.2$ -- хаос;
	\item $a = 0.3$ -- хаотичный аттрактор;
	\item $a = 0.35$ -- аналогичен предыдущему, но хаос проявляется сильнее;
	\item $a = 0.38$ -- аналогичен предыдущему, но хаос проявляется ещё сильнее.
\end{itemize}
\subsection*{Изменение параметра $b$}
Зафиксируем $a = 0.2$, $r = 5.7$ и будем менять теперь параметр $b$. При $b$ стремящемся к нулю аттрактор неустойчив. Когда $b$ станет больше $a$ и $r$, система уравновесится и перейдёт в стационарное состояние.
\subsection*{Изменение параметра $r$}
Зафиксируем $a = b = 0.1$ и будем изменять $r$.
\newpage
\noindent\makebox[\linewidth]{\rule{\paperwidth}{0.4pt}}
Из бифуркационной диаграммы видно, что при маленьких $r$ система периодична, но при увеличении быстро становится хаотичной.
\noindent\makebox[\linewidth]{\rule{\paperwidth}{0.4pt}}\\
Рисунки показывают, как именно меняется хаотичность системы при увеличении $r$. Например при $r = 4$ аттрактор будет иметь период равный единице, и на диаграмме будет одна единственная линия, то же самое повторится когда $r = 3$ и так далее; пока $r$ не станет больше 12: последнее периодичное поведение характеризуется именно этим значением, дальше повсюду идёт хаос.
\section*{Заключение}
Аттрактор Рёсслера наблюдается во многих системах. Например, он применяется для описания потоков жидкости, а также при изучении поведения различных химических реакций и молекулярных процессов.
\end{document}