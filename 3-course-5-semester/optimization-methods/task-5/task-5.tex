\documentclass[14pt,a4paper,fleqn]{extarticle}
\usepackage[T2A,T1]{fontenc}
\usepackage[utf8]{inputenc}
\usepackage[russian]{babel}
\usepackage{amsmath}
\usepackage{graphicx}
\usepackage{tabularx}
\usepackage{boldline}
\usepackage{makecell}

\graphicspath{ {./images/} }
\setlength{\mathindent}{0pt}
\setlength\parindent{0pt}


\begin{document}
	\begin{titlepage}
		\includegraphics[scale=0.12]{logo}
		\begin{center}
			\textbf{МИНОБРНАУКИ РОССИИ}\\
			\vspace{0.2cm}
			\textbf{Федеральное государственное бюджетное образовательное учреждение высшего образования}\\
			\textbf{«САНКТ-ПЕТЕРБУРГСКИЙ ГОСУДАРСТВЕННЫЙ ЭКОНОМИЧЕСКИЙ УНИВЕРСИТЕТ»}\\
			\vspace{0.6cm}
			Факультет информатики и прикладной математики\\
			Кафедра прикладной математики и экономико-математических методов\\
			\vspace{1cm}
			\textbf{ОТЧЁТ}\\
			по дисциплине:\\
			\textbf{«Методы оптимизации»}\\
			на тему:\\
			\textbf{«Построение и решение двойственной задачи. Задание 5»}\\
		\end{center}
		\vspace{1cm}
		Направление: 01.03.02\\
		Обучающийся: Бронников Егор Игоревич\\
		Группа: ПМ-1901\\
		\vfill
		\begin{center}
			Санкт-Петербург\\
			2021\\
		\end{center}
	\end{titlepage}
\section*{Задание 2.1}
\subsection*{Дано}
Целевая функция:\\
$f = -x_1+x_2 \longrightarrow max$\\\\
Ограничения:
\begin{align*}
	\begin{cases}
		-x_1 + 2x_2 \geq -1\\
		-2x_1 + x_2 \leq 2\\
		3x_1 + x_2 \leq 3\\
	\end{cases}
\end{align*}
$x_1 \geq 0, x_2 \geq 0$
\subsection*{Задание}
\subsubsection*{Каноническая форма прямой задачи}
1. Вводим слабые переменные $y_1 \geq 0, y_2 \geq 0, y_3 \geq 0$:\\
$-x_1 + 2x_2 - y_1 = -1$\\
$-2x_1 + x_2 + y_2 = 2$\\
$3x_1 + x_2 + y_3 = 3$\\

2. Делаем правые части равенств положительными:\\
$x_1 - 2x_2 + y_1 = 1$\\
$-2x_1 + x_2 + y_2 = 2$\\
$3x_1 + x_2 + y_3 = 3$\\

Таким образом, прямая задача сведена к канонической форме.
\newpage
\subsubsection*{Формулируем двойственную задачу}
Функция цели:\\
$\phi = -\lambda_1 + 2\lambda_2 + 3\lambda_3 \longrightarrow min$\\\\
Ограничения:\\
$\lambda_1 - 2\lambda_2 + 3\lambda_3 \geq -1$\\
$-2\lambda_1 + \lambda_2 + \lambda_3 \geq 1$\\
$\lambda_1 \geq 0, \lambda_2 \geq 0, \lambda_3 \geq 0$
\subsubsection*{Каноническая форма двойственной задачи}
Функция цели:\\
$\phi = -\lambda_1 + 2\lambda_2 + 3\lambda_3 \longrightarrow min$\\\\
1. Вводим слабые переменные $\xi_1 \geq 0, \xi_2 \geq 0$:\\
$\lambda_1 - 2\lambda_2 + 3\lambda_3 - \xi_1 = -1$\\
$-2\lambda_1 + \lambda_2 + \lambda_3 - \xi_2 = 1$\\
$\lambda_1 \geq 0, \lambda_2 \geq 0, \lambda_3 \geq 0$\\

2. Делаем правые части равенств положительными:\\
$-\lambda_1 + 2\lambda_2 - 3\lambda_3 + \xi_1 = 1$\\
$-2\lambda_1 + \lambda_2 + \lambda_3 - \xi_2 = 1$\\
$\lambda_1 \geq 0, \lambda_2 \geq 0, \lambda_3 \geq 0$
\newpage
\subsubsection*{Решение двойственной задачи}
\noindent\makebox[\linewidth]{\rule{\paperwidth}{0.4pt}}\\
\textbf{Напоминание из задачи 2.1}\\
$y_2, y_3$ --- свободные переменные\\
$x_1, x_2, y_1$ --- базисные переменные\\

1. Выразим базисные переменные через свободные:\\
$x_1 = \frac{1}{5} + \frac{y_2}{5} - \frac{y_3}{5}$\\
$x_2 = 2 + 2x_1 - y_2 = \frac{12}{5} - \frac{3}{5}y_2 - \frac{2}{5}y_3$\\
$y_1 = 5 + 3x_1 - 2y_2 = \frac{28}{5} - \frac{7}{5}y_2 - \frac{3}{5}y_3$\\
\begin{center}$\downarrow$\end{center}
$x_1 = \frac{1}{5} + \frac{y_2}{5} - \frac{y_3}{5}$\\
$x_2 = \frac{12}{5} - \frac{3}{5}y_2 - \frac{2}{5}y_3$\\
$y_1 = \frac{28}{5} - \frac{7}{5}y_2 - \frac{3}{5}y_3$\\

2. Выразим функцию цели $f$ через свободные переменные:\\
$f = x_1 - y_2 + 2 = \frac{11}{5} - \frac{4}{5}y_2 - \frac{y_3}{5} \longrightarrow max$
\begin{center}$\downarrow$\end{center}
$f = \frac{11}{5} - \frac{4}{5}y_2 - \frac{y_3}{5} \longrightarrow max$\\
\noindent\makebox[\linewidth]{\rule{\paperwidth}{0.4pt}}\\
Из оптимального решения симплекс-методом прямой задачи видно, что переменные базисные переменные $x_1, x_2$ выражаются через свободные переменные $y_2, y_3$ следующим образом:\\
$x_1 = \frac{1}{5} + \frac{y_2}{5} - \frac{y_3}{5}$\\
$x_2 = \frac{12}{5} - \frac{3}{5}y_2 - \frac{2}{5}y_3$\\\\
Функция цели:\\
$f = \frac{11}{5} - \frac{4}{5}y_2 - \frac{y_3}{5} \longrightarrow max$\\
$y_2$ и $y_3$ входят в выражение для оптимальной функции цели прямой задачи с коэффициентами $-\frac{4}{5}$ и $-\frac{1}{5}$.\\\\
Переменные $y_2$ и $y_3$ индуцируют следующие ограничения двойственной задачи:
\begin{center}
$y_2 \rightarrow \lambda_2 \geq 0 \hspace*{1cm} y_3 \rightarrow \lambda_3 \geq 0$
\end{center}
\noindent\makebox[\linewidth]{\rule{\paperwidth}{0.4pt}}\\
\textbf{Лемма:}\\
Если  в  прямой  (двойственной)  задаче  начальная 
базисная переменная входит в оптимальное решение целевой 
функции  как  свободная  переменная,  то  коэффициент  при  ней  в 
целевой  функции  равен  разности  между  правой  и  левой  частями 
ограничения двойственной (прямой) задачи, индуцированного 
рассматриваемой начальной базисной переменной.\\
\noindent\makebox[\linewidth]{\rule{\paperwidth}{0.4pt}}\\
По лемме получается:
\begin{center}
$-\frac{4}{5} = 0 - \lambda_2 \hspace*{1cm} -\frac{1}{5} = 0 - \lambda_3$
\end{center}
Тогда решение двойственной задачи выглядит следующим образом:\\
$\lambda_1 = 0, \lambda_2 = \frac{4}{5}, \lambda_3 = \frac{1}{5}$\\\\
Функция цели:\\
$\phi = -\lambda_1 + 2\lambda_2 + 3\lambda_3 = \frac{8}{5} + \frac{3}{5} = \frac{11}{5}$
\begin{center}$\downarrow$\end{center}
$\phi = \frac{11}{5} = 2.2$\\
\noindent\makebox[\linewidth]{\rule{\paperwidth}{0.4pt}}\\
\textbf{Напоминание из задачи 2.1}\\
$x_1 = \frac{1}{5} = 0.2$\\
$x_2 = \frac{12}{5} = 2.4$\\
$f = \frac{11}{5} = 2.2$\\
\noindent\makebox[\linewidth]{\rule{\paperwidth}{0.4pt}}\\
То есть $\phi(\Lambda^*) = f(X^*)$ и критерий оптимальности Канторовича выполняется.\\
\textbf{Ответ:} $\lambda_1 = 0, \lambda_2 = \frac{4}{5}, \lambda_3 = \frac{1}{5}, \phi = \frac{11}{5}$
\newpage
\section*{Задание 4.1}
\subsection*{Дано}
Целевая функция:\\
$f = 4x_1+x_2 \longrightarrow max$\\\\
Ограничения:
\begin{align*}
	\begin{cases}
		3x_1 - 2x_2 \geq -8\\
		3x_1 + x_2 \geq 3\\
		x_2 \leq 8\\
		x_1 \leq 4\\
	\end{cases}
\end{align*}
$x_1 \geq 0, x_2 \geq 0$

\subsection*{Задание}
\subsubsection*{Каноническая форма прямой задачи}
1. Вводим слабые переменные $y_1 \geq 0, y_2 \geq 0, y_3 \geq 0, y_4 \geq 0$:\\
$3x_1 - 2x_2 - y_1 = -8$\\
$3x_1 + x_2 - y_2 = 3$\\
$x_2 + y_3 = 8$\\
$x_1 + y_4 = 4$\\

2. Делаем правые части равенств положительными:\\
$-3x_1 + 2x_2 + y_1 = 8$\\
$3x_1 + x_2 - y_2 = 3$\\
$x_2 + y_3 = 8$\\
$x_1 + y_4 = 4$\\

Таким образом, прямая задача сведена к канонической форме.
\newpage
\subsubsection*{Метод штрафов}
Введём искусственную переменную --- $r \geq 0$.\\\\
Целевая функция:\\
$f = 4x_1+x_2 \longrightarrow max$\\\\
Ограничения:\\
$-3x_1 + 2x_2 + y_1 = 8 \hspace*{1cm} \rightarrow \lambda_1$\\
$3x_1 + x_2 - y_2 + r = 3 \hspace*{0.8cm} \rightarrow \lambda_2$\\
$x_2 + y_3 = 8 \hspace*{3cm} \rightarrow \lambda_3$\\
$x_1 + y_4 = 4 \hspace*{3cm} \rightarrow \lambda_4$\\\\
$x_i \geq 0 \hspace{0.2cm} \forall i = \overline{1,2}; \hspace{0.5cm} y_j \geq 0 \hspace{0.2cm} \forall j =  \overline{1,4}$\\\\
Перепишем функцию цели:\\
$f = 4x_1 + x_2 - Mr = 4x_1 + x_2 - M(3 - 3x_1 - x_2 + y_2)$
\begin{center}$\downarrow$\end{center}
$f = -3M + (3M + 4)x_1 + (M + 1)x_2 - My_2$\\\\
Пусть $M = 100$, тогда функция цели примет следующий вид:\\
$f = -300 + 304x_1 + 101x_2 - 100y_2$
\subsubsection*{Формулируем двойственную задачу}
Функция цели:\\
$\phi = 8\lambda_1 + 3\lambda_2 + 8\lambda_3 + 4\lambda_4 - 300 \longrightarrow min$\\\\
Ограничения:\\
$-3\lambda_1 + 3\lambda_2 + \lambda_4 \geq 304$\\
$2\lambda_1 + \lambda_2 + \lambda_3 \geq 101$\\
$-\lambda_2 \geq -100$\\
$\lambda_1 \geq 0, \lambda_3 \geq 0, \lambda_4 \geq 0$
\newpage
\subsubsection*{Решение двойственной задачи}
\noindent\makebox[\linewidth]{\rule{\paperwidth}{0.4pt}}\\
\textbf{Напоминание из задачи 4.1}\\
Базисные переменные: $y_1, x_1, x_2, y_2$.\\
Свободные переменный: $y_3, y_4, r$.\\\\
\begin{tabularx}{\textwidth}{|p{1cm}|p{1cm}|p{1cm}|p{1cm}|p{1cm}|p{1cm}|p{1cm}|p{1cm}|p{1.2cm}|c}
	\hline
	БП & $x_1$ & $x_2$ & $y_1$ & $y_2$ & $y_3$ & $y_4$ & $r$ & Своб. член & \\
	\hline
	f & 0 & 0 & 0 & 0 & 1 & 4 & 100 & 24 & \\
	\hline
	$y_1$ & 0 & 0 & 1 & 0 & -2 & 3 & 0 & 4 & \\
	\hline
	$x_1$ & 1 & 0 & 0 & 0 & 0 & 1 & 0 & 4\\
	\hline
	$x_2$ & 0 & 1 & 0 & 0 & 1 & 0 & 0 & 8 \\
	\hline
	$y_2$ & 0 & 0 & 0 & 1 & 1 & 3 & -1 & 17\\
	\hline
\end{tabularx}
\newline\newline
Таким образом, получается:\\
$f + y_3 + 4y_4 + 100r = 24 \hspace*{0.5cm} \rightarrow \hspace*{0.5cm} f = 24$\\
$y_1 - 2y_3 + 3y_4 = 4 \hspace*{1.9cm} \rightarrow \hspace*{0.5cm} y_1 = 4$\\
$x_1 + y_4 = 4 \hspace*{3.4cm} \rightarrow \hspace*{0.5cm} x_1 = 4$\\
$x_2 + y_3 = 8 \hspace*{3.4cm} \rightarrow \hspace*{0.5cm} x_2 = 8$\\
$y_2 + y_3 + 3y_4 - r = 17 \hspace*{1.05cm} \rightarrow \hspace*{0.5cm} y_2 = 17$\\
$y_3 = 0, y_4 = 0, r = 0$\\
\noindent\makebox[\linewidth]{\rule{\paperwidth}{0.4pt}}\\
Целевая функция прямой задачи выражается так:\\
$f = 24 - y_3 - 4y_4 - 100r \longrightarrow max$\\
Переменные $y_2$, $y_3$ и $y_4$ индуцируют следующие ограничения двойственной задачи:
\begin{center}
	$y_2 \rightarrow -\lambda_2 \geq -100 \hspace*{1cm} y_3 \rightarrow \lambda_3 \geq 0 \hspace*{1cm} y_4 \rightarrow \lambda_4 \geq 0$
\end{center}
Тогда по лемме:
\begin{center}
	$0 = -100 + \lambda_2 \hspace*{1cm} -1 = 0 - \lambda_3 \hspace*{1cm} -4 = 0 - \lambda_4$
\end{center}
\newpage
Тогда решение двойственной задачи выглядит следующим образом:\\
$\lambda_1 = 0, \lambda_2 = 100, \lambda_3 = 1, \lambda_4 = 4$\\\\
Функция цели:\\
$\phi = 8\lambda_1 + 3\lambda_2 + 8\lambda_3 + 4\lambda_4 - 300 = 8*0 + 3*100 + 8*1 + 4*4 - 300 = 24$
\begin{center}$\downarrow$\end{center}
$\phi = 24$\\
\noindent\makebox[\linewidth]{\rule{\paperwidth}{0.4pt}}\\
\textbf{Напоминание из задачи 4.1}\\
$x_1 = 4$\\
$x_2 = 8$\\
$f = 24$\\
\noindent\makebox[\linewidth]{\rule{\paperwidth}{0.4pt}}\\
То есть $\phi(\Lambda^*) = f(X^*)$ и критерий оптимальности Канторовича выполняется.\\
\textbf{Ответ:} $\lambda_1 = 0, \lambda_2 = 100, \lambda_3 = 1, \lambda_4 = 4, \phi = 24$
\end{document}