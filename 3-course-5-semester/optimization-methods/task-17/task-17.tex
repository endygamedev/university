\documentclass[14pt,a4paper,fleqn]{extarticle}
\usepackage[T2A,T1]{fontenc}
\usepackage[utf8]{inputenc}
\usepackage[russian]{babel}
\usepackage{amsmath}
\usepackage{graphicx}
\usepackage{tabularx}
\usepackage{boldline}
\usepackage{makecell}
\usepackage{arydshln}
\usepackage{mathtools}

\graphicspath{ {./images/} }
\setlength{\mathindent}{0pt}
\setlength\parindent{0pt}


\begin{document}
	\begin{titlepage}
		\includegraphics[scale=0.12]{logo}
		\begin{center}
			\textbf{МИНОБРНАУКИ РОССИИ}\\
			\vspace{0.2cm}
			\textbf{Федеральное государственное бюджетное образовательное учреждение высшего образования}\\
			\textbf{«САНКТ-ПЕТЕРБУРГСКИЙ ГОСУДАРСТВЕННЫЙ ЭКОНОМИЧЕСКИЙ УНИВЕРСИТЕТ»}\\
			\vspace{0.6cm}
			Факультет информатики и прикладной математики\\
			Кафедра прикладной математики и экономико-математических методов\\
			\vspace{1cm}
			\textbf{ОТЧЁТ}\\
			по дисциплине:\\
			\textbf{«Методы оптимизации»}\\
			на тему:\\
			\textbf{«Задание 17. Метод сопряжённых направлений»}\\
		\end{center}
		\vspace{1cm}
		Направление: 01.03.02\\
		Обучающийся: Бронников Егор Игоревич\\
		Группа: ПМ-1901\\
		\vfill
		\begin{center}
			Санкт-Петербург\\
			2021\\
		\end{center}
	\end{titlepage}
	\textbf{Дано:}\\
	$f(x_1,x_2,x_3) = (3x_1-3x_2-5)^2 + (6x_1-x_2-x_3-2)^2+(2x_1+5x_2+x_3-1)^2$\\\\
	\textbf{Условие:}\\
	Найти стационарную точку методом сопряжённых направлений.\\
	
	\textbf{Решение:}\\
	Проверим выпуклость целевой функции. Определим первые производные функции, матрицу Гессе функции $f(x_1,x_2,x_3)$ и её угловые миноры:\\
	
	$\dfrac{df}{dx_1} = 98x_1 - 10x_2 - 8x_3 - 58$\\\\
	$\dfrac{df}{dx_2} = -10x_1 + 70x_2 + 12x_3 + 24$\\\\
	$\dfrac{df}{dx_3} = -8x_1 + 12x_2 + 4x_3 + 2$
	
	\begin{align*}
		H(X) = \begin{pmatrix}
			98 & -10 & -8\\
			-10 & 70 & 12\\
			-8 & 12 & 4\\
		\end{pmatrix}
	\end{align*}
	Вычисляем главные миноры:\\
	$M_1(\boldsymbol{H}) = 98 > 0, \hspace*{0.2cm} M_2(\boldsymbol{H}) = 6760 > 0, \hspace*{0.2cm} M_3(\boldsymbol{H}) = |\boldsymbol{H}| = 10368 > 0$\\
	
	Матрица \textbf{H} -- положительно определённая матрица и, следовательно, $f(x_1, x_2, x_3)$ -- выпуклая функция, которая имеет минимум в некоторой точке $X^*$.\\
	
	$grad\hspace*{0.05cm}f(X) = (\dfrac{df}{dx_1}, \dfrac{df}{dx_2}, \dfrac{df}{dx_3}) =$\\\\
	$= (98x_1 - 10x_2 - 8x_3 - 58, -10x_1 + 70x_2 + 12x_3 + 24, -8x_1 + 12x_2 + 4x_3 + 2)$\\
	
	Выберем в качестве начальной точки $X^0 = (0,0,0)$. Тогда на первом шаге итерации получаем:
	\newpage
	$f(X^0) = 30, \hspace*{0.1cm} grad\hspace*{0.05cm}f(X^0) = (-58,24,2)$\\
	
	Рассчитываем сопряжённый вектор:\\
	$\boldsymbol{S}^0 = -grad\hspace*{0.05cm}f(X^0) = -(-58,24,2)$\\
	
	Рассчитываем смещение вдоль направления $\boldsymbol{S}^0$:\\
	$t_1 = \dfrac{-grad\hspace*{0.05cm}f(X^0) \cdot \boldsymbol{S}^{0t}}{\boldsymbol{S}^0\boldsymbol{H}\boldsymbol{S}^{0t}} = \dfrac{-(-58,24,2)\begin{pmatrix} 58\\ -24\\ -2\\ \end{pmatrix}}{(58,-24,-2)\begin{pmatrix} 98 & -10 & -8\\ -10 & 70 & 12\\ -8 & 12 & 4\\ \end{pmatrix}\begin{pmatrix} 58\\ -24\\ -2\\ \end{pmatrix}} = $\\
	$= \dfrac{493}{50107} \approx 0.00983894$\\\\
	
	\small $X^1 = X^0 + t_1\boldsymbol{S}^{0} = (0,0,0) + 0.00983894(58,-24,-2) = (0.570659, -0.236135, -0.0196779)$\\
	
	\normalsize $f(X^1) = 10.5976, grad\hspace*{0.05cm}f(X^1) = (0.443355, 1.52783, -5.4776)$\\
	
	Находим коэффициент $\beta_0$ из условия сопряжённости:\\
	
	\small $\beta_0 = \dfrac{(\boldsymbol{H}\boldsymbol{S^{0t}})\cdot grad\hspace*{0.05cm}f(X^1)}{\boldsymbol{S^{0}}\boldsymbol{H}\boldsymbol{S^{0t}}} = \dfrac{\begin{pmatrix} 98 & -10 & -8\\ -10 & 70 & 12\\ -8 & 12 & 4\\ \end{pmatrix}\begin{pmatrix} 58\\ -24\\ -2\\ \end{pmatrix}(0.443355, 1.52783, -5.4776)}{(58,-24,-2)\begin{pmatrix} 98 & -10 & -8\\ -10 & 70 & 12\\ -8 & 12 & 4\\ \end{pmatrix}\begin{pmatrix} 58\\ -24\\ -2\\ \end{pmatrix}} = $\\
	$= 0.0082497$\\
	
	\normalsize Второй шаг итерации:\\
	
	$\boldsymbol{S^{1}} = -grad\hspace*{0.05cm}f(X^1) + \beta_0\boldsymbol{S^{0}} = (0.0351276, -1.72582, 5.4611)$\\
	
	$t_2 = \dfrac{-grad\hspace*{0.05cm}f(X^1) \cdot \boldsymbol{S}^{1t}}{\boldsymbol{S}^1\boldsymbol{H}\boldsymbol{S}^{1t}} = 0.325827$\\
	
	\newpage
	$X^2 = X^1 + t_2\boldsymbol{S}^{1} = (0.582105, -0.798454, 1.7597)$\\
	
	$f(X^2) = 5.2972, grad\hspace*{0.05cm}f(X^2) = (-7.04677, -16.5964, -5.19949)$\\
	
	$\beta_1 = 10.8232$\\
	
	Третий шаг итерации:\\
	
	$\boldsymbol{S^{2}} = -grad\hspace*{0.05cm}f(X^2) + \beta_1\boldsymbol{S^{1}} = (7.42694, -2.08269, 64.3065)$\\
	
	$t_3 = \dfrac{-grad\hspace*{0.05cm}f(X^2) \cdot \boldsymbol{S}^{2t}}{\boldsymbol{S}^2\boldsymbol{H}\boldsymbol{S}^{2t}} = 0.0300868$\\
	
	$X^3 = X^2 + t_3\boldsymbol{S}^{2} = (0.805556, -0.861111, 3.69444)$\\
	
	$f(X^3) \approx 0, grad\hspace*{0.05cm}f(X^3) \approx (0, 0, 0)$\\
	
	$\beta_2 \approx 0$\\
	
	$\boldsymbol{S^{3}} = -grad\hspace*{0.05cm}f(X^3) + \beta_2\boldsymbol{S^{2}} \approx (0,0,0)$\\
	
	Останавливаемся на третьем шаге и заканчиваем итерационный процесс по критерию $\dfrac{||grad\hspace*{0.05cm}f(X^{k+1})||}{|f(X^{k+1})|} \leq \delta_3$ и потому что $\beta_2 = 0$ и $\boldsymbol{S^{3}} = (0,0,0)$. Данное решение задачи не зависит от выбора начальной точки $X^0$, а её точное решение равно $X^* = (\dfrac{29}{36}, -\dfrac{31}{36}, \dfrac{133}{36})$, $f(X^*) = 0$.
\end{document}