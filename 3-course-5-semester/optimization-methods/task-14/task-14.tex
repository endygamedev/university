\documentclass[14pt,a4paper,fleqn]{extarticle}
\usepackage[T2A,T1]{fontenc}
\usepackage[utf8]{inputenc}
\usepackage[russian]{babel}
\usepackage{amsmath}
\usepackage{graphicx}
\usepackage{tabularx}
\usepackage{boldline}
\usepackage{makecell}
\usepackage{arydshln}
\usepackage{mathtools}

\graphicspath{ {./images/} }
\setlength{\mathindent}{0pt}
\setlength\parindent{0pt}


\begin{document}
	\begin{titlepage}
		\includegraphics[scale=0.12]{logo}
		\begin{center}
			\textbf{МИНОБРНАУКИ РОССИИ}\\
			\vspace{0.2cm}
			\textbf{Федеральное государственное бюджетное образовательное учреждение высшего образования}\\
			\textbf{«САНКТ-ПЕТЕРБУРГСКИЙ ГОСУДАРСТВЕННЫЙ ЭКОНОМИЧЕСКИЙ УНИВЕРСИТЕТ»}\\
			\vspace{0.6cm}
			Факультет информатики и прикладной математики\\
			Кафедра прикладной математики и экономико-математических методов\\
			\vspace{1cm}
			\textbf{ОТЧЁТ}\\
			по дисциплине:\\
			\textbf{«Методы оптимизации»}\\
			на тему:\\
			\textbf{«Задание 14. Условный экстремум»}\\
		\end{center}
		\vspace{1cm}
		Направление: 01.03.02\\
		Обучающийся: Бронников Егор Игоревич\\
		Группа: ПМ-1901\\
		\vfill
		\begin{center}
			Санкт-Петербург\\
			2021\\
		\end{center}
	\end{titlepage}
	\section*{Дано}
	Функция:\\
	$f = 2x_1^2 + 3x_2^2 + 4x_3^2 + 2x_1x_2 + 2x_1x_3 - x_2x_3 - 3x_1 - 5x_2 - 55x_3$\\
	
	Ограничения:\\
	$2x_1 - x_2 + x_3 = -2$\\
	$x_1 - 2x_2 + 3x_3 = -7$
	
	\section*{Задание 1}
	\textbf{Условие:}\\
	Составить функцию Лагранжа.\\
	
	\textbf{Решение:}\\
	Составим функцию Лагранжа $L(\Lambda, X)$:\\
	
	$L(\Lambda, X) = 2x_1^2 + 3x_2^2 + 4x_3^2 + 2x_1x_2 + 2x_1x_3 - x_2x_3 - 3x_1 - 5x_2 - 55x_3 + \lambda_1(2x_1 - x_2 + x_3 + 2) + \lambda_2(x_1 - 2x_2 + 3x_3 + 7)$
	
	\newpage
	\section*{Задание 2}
	\textbf{Условие:}\\
	Определить стационарную точку и проверить её на экстремум.\\
	
	\textbf{Решение:}\\
	\textit{Способ 1:}\\
	Составим и решим систему:\\
	
	\scalebox{1}{
		$\begin{cases}
			\vspace{0.3cm}
			\dfrac{dL(\Lambda, X)}{dx_1} = 4x_1 + 2x_2 + 2x_3 - 3 + 2\lambda_1 + \lambda_2 = 0\\
			\vspace{0.3cm}
			\dfrac{dL(\Lambda, X)}{dx_2} = 6x_2 + 2x_1 - x_3 - 5 - \lambda_1 - 2\lambda_2 = 0\\
			\vspace{0.3cm}
			\dfrac{dL(\Lambda, X)}{dx_3} = 8x_3 + 2x_1 - x_2 - 55 + \lambda_1 + 3\lambda_2 = 0\\
			\vspace{0.3cm}
			\dfrac{dL(\Lambda, X)}{d\lambda_1} = 2x_1 - x_2 + x_3 + 2 = 0\\
			\dfrac{dL(\Lambda, X)}{d\lambda_2} = x_1 - 2x_2 + 3x_3 + 7 = 0\\
		\end{cases}$}\\\\
	
	Решая данную систему, получим стационарную точку, вектор параметров Лагранжа и значение функции в стационарной точке:\\
	
	$Y = (x_1, x_2, x_3) = (\dfrac{5}{4}, \dfrac{21}{4}, \dfrac{3}{4}) = (1.25, 5.25, 0.75)$\\\\
	$\Lambda = (\lambda_1, \lambda_2) = (-\dfrac{75}{4}, \dfrac{47}{2}) = (-18.75, 23.5)$\\\\
	$f(\dfrac{5}{4}, \dfrac{21}{4}, \dfrac{3}{4}) = \dfrac{223}{8}$\\
	
	Найдём такие вектора для которых $grad \psi_j (Y) \cdot \delta X = 0$:\\
	$j = 1: 2\delta x_1 - \delta x_2 + \delta x_3 = 0$\\
	$j = 2: \delta x_1 - 2\delta x_2 + 3\delta x_3 = 0$
	
	\newpage
	
	Решим систему приняв $\delta x_3$ за параметр:\\
	
	\scalebox{1}{
	$\begin{cases}
		\vspace{0.3cm}
		2\delta x_1 - \delta x_2 = -\delta x_3\\
		\delta x_1 - 2\delta x_2 = -3\delta x_3
	\end{cases}$
	$\Rightarrow$
	$\begin{cases}
	\vspace{0.3cm}
	\delta x_1 = \dfrac{\delta x_3}{3}\\
	\delta x_2 = \dfrac{5 \delta x_3}{3}
	\end{cases}$}\\\\
	
	Получили следующий вектор: $\delta X = (\dfrac{\delta x_3}{3}, \dfrac{5 \delta x_3}{3}, \delta x_3)$\\
	Составим матрицу вторых производных $L^{''}_{XX}$:
	\begin{align*}
		L^{''}_{XX} = \begin{pmatrix}
			4 & 2 & 2\\
			2 & 6 & -1\\
			2 & -1 & 8\\
		\end{pmatrix}
	\end{align*}

	Теперь определим знак квадратичной формы:\\
	$\delta X \cdot L^{''}_{XX} \cdot \delta X^T = \dfrac{76\delta x^2_3}{3} > 0 \Rightarrow Y = (\dfrac{5}{4}, \dfrac{21}{4}, \dfrac{3}{4})$ -- точка минимума.\\
	
	\textit{Способ 2:}\\
	Проверим точку на экстремум при помощи матрицы Гессе:
	\begin{align*}
		H(\Lambda, X) = \begin{pmatrix}
			0 & 0 & 2 & -1 & 1\\
			0 & 0 & 1 & -2 & 3\\
			2 & 1 & 4 & 2 & 2\\
			-1 & -2 & 2 & 6 & -1\\
			1 & 3 & 2 & -1 & 8\\
		\end{pmatrix}
	\end{align*}
	В рассматриваемой задаче N = 3 и M = 2, необходимо посчитать миноры порядка 4 и 5, проверить их на знаки.\\
	
	$M_4(H) = 9 > 0, \hspace*{0.2cm} M_5(H) = 228 > 0$\\
	Оба минора положительны, значит точка $(\dfrac{5}{4}, \dfrac{21}{4}, \dfrac{3}{4})$ -- точка минимума.
	
	\newpage
	\section*{Задание 3}
	\textbf{Условие:}\\
	Найти стационарную точку методом Якоби, проверить её на экстремум и исследовать решение на чувствительность.\\
	
	\textbf{Решение:}\\
	Выберем в качестве свободной перменной -- $x_3 (Z)$, а $x_1$ и $x_2 (S)$ в качестве базисных. Посчитаем градиенты целевой функции по этим переменным:\\
	$grad_Z f(X) = (\dfrac{df}{dx_3}) = (8x_3 + 2x_1 - x_2 - 55)$\\\\
	$grad_S f(X) = (\dfrac{df}{dx_1}, \dfrac{df}{dx_2}) = (4x_1 + 2x_2 + 2x_3 - 3, 6x_2 + 2x_1 - x_3 - 5)$\\
	
	Теперь посчитаем матрицу управления и матрицу Якоби:\\\\
	$J(X) = \begin{pmatrix}
			\dfrac{d\psi_1}{dx_1} & \dfrac{d\psi_1}{dx_2}\\
			\dfrac{d\psi_2}{dx_1} & \dfrac{d\psi_2}{dx_2}\\
		\end{pmatrix} =
		\begin{pmatrix}
			2 & -1\\
			1 & -2\\
		\end{pmatrix}\\
	$\\\\
	$C(X) = \begin{pmatrix}
			\dfrac{d\psi_1}{dx_3}\\
			\dfrac{d\psi_2}{dx_3}\\
		\end{pmatrix} =
		\begin{pmatrix}
			1\\
			3\\
		\end{pmatrix}$\\\\
	Составим условный градиент -- $grad_* f(x)$:\\\\
	$grad_* f(X) = grad_Z f(X) - grad_S f(X) J^{-1} C =$\\\\
	$= (8x_3 + 2x_1 - x_2 - 55) - (4x_1 + 2x_2 + 2x_3 - 3, 6x_2 + 2x_1 - x_3 - 5) J^{-1} C =$\\\\
	$(-\dfrac{193}{3} + \dfrac{20x_1}{3} + \dfrac{29x_2}{3} + 7x_3)$
	\newpage
	Решаем систему, чтобы найти стационарную точку:\\\\
		\scalebox{1}{
		$\begin{cases}
			\vspace{0.3cm}
			grad_*f(x) = 0\\
			\vspace{0.3cm}
			2x_1 - x_2 + x_3 + 2 = 0\\
			x_1 - 2x_2 + 3x_3 + 7 = 0
		\end{cases}$
		$\Rightarrow$
		$\begin{cases}
			\vspace{0.3cm}
			x_1 = \dfrac{5}{4} = 1.25\\
			\vspace{0.3cm}
			x_2 = \dfrac{21}{4} = 5.25\\
			x_3 = \dfrac{3}{4} = 0.75\\
		\end{cases}$}\\\\
	
	$\delta x_3 = -J^{-1} C \begin{pmatrix}
		\delta x_1\\
		\delta x_2\\
	\end{pmatrix}$\\\\
	Проверим на экстремум, составив матрицу Гессе:\\
	$H(Y) = \begin{pmatrix}
		\dfrac{grad_*f(Y)}{dx_3}\\
	\end{pmatrix} = 
	\begin{pmatrix}
		\dfrac{29}{3}\\
	\end{pmatrix}$\\\\
	Матрица Гессе положительна определена и по достаточному условию экстремума точка $Y = (1.25, 5.25, 0.75)$ является точкой минимума.\\\\
	\textbf{Анализ чувствительности}\\\\
	$grad_S f(Y) = (14, \dfrac{113}{4})$\\\\
	Расчитаем коэффициенты чувствительности:\\
	$\dfrac{\delta f(Y)}{\delta B} = grad_S f(Y)J^{-1}(Y) = (\dfrac{75}{4}, -\dfrac{47}{2}) = (18.75, -23.5)$\\\\
	При увеличении $b_1$ на единицу, функция цели увеличится на 18.75, а при увеличении $b_2$ на единицу, функция цели уменьшится на $-23.5$. Целевая функция будет быстрее возрастать по первой переменной.
\end{document}