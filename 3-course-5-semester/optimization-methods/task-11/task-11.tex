\documentclass[14pt,a4paper,fleqn]{extarticle}
\usepackage[T2A,T1]{fontenc}
\usepackage[utf8]{inputenc}
\usepackage[russian]{babel}
\usepackage{amsmath}
\usepackage{graphicx}
\usepackage{tabularx}
\usepackage{boldline}
\usepackage{makecell}
\usepackage{arydshln}
\usepackage{mathtools}

\graphicspath{ {./images/} }
\setlength{\mathindent}{0pt}
\setlength\parindent{0pt}


\begin{document}
	\begin{titlepage}
		\includegraphics[scale=0.12]{logo}
		\begin{center}
			\textbf{МИНОБРНАУКИ РОССИИ}\\
			\vspace{0.2cm}
			\textbf{Федеральное государственное бюджетное образовательное учреждение высшего образования}\\
			\textbf{«САНКТ-ПЕТЕРБУРГСКИЙ ГОСУДАРСТВЕННЫЙ ЭКОНОМИЧЕСКИЙ УНИВЕРСИТЕТ»}\\
			\vspace{0.6cm}
			Факультет информатики и прикладной математики\\
			Кафедра прикладной математики и экономико-математических методов\\
			\vspace{1cm}
			\textbf{ОТЧЁТ}\\
			по дисциплине:\\
			\textbf{«Методы оптимизации»}\\
			на тему:\\
			\textbf{«Решение задачи дискретной оптимизации методом Гомори. Задание 11»}\\
		\end{center}
		\vspace{1cm}
		Направление: 01.03.02\\
		Обучающийся: Бронников Егор Игоревич\\
		Группа: ПМ-1901\\
		\vfill
		\begin{center}
			Санкт-Петербург\\
			2021\\
		\end{center}
	\end{titlepage}

	\section*{Испорченная задача №4}
	
	Целевая функция:\\
	$f = 4x_1 + x_2 \longrightarrow max$\\\\
	Ограничения:
	\begin{align*}
		\begin{cases}
			3x_1 - 2x_2 \geq -8\\
			3x_1 + x_2 \geq 3\\
			\boldsymbol{5} x_2 \leq 8\\
			\boldsymbol{3} x_1 \leq 4\\
		\end{cases}
	\end{align*}
	$x_1 \geq 0, x_2 \geq 0$
	\section*{Каноническая форма}
	1. Вводим слабые переменные $y_1 \geq 0$, $y_2 \geq 0$, $y_3 \geq 0$, $y_4 \geq 0$:\\
	$3x_1 - 2x_2 - y_1 = -8$\\
	$3x_1 + x_2 - y_2 = 3$\\
	$5x_2 + y_3 = 8$\\
	$3x_1 + y_4 = 4$\\
	
	2. Делаем правые части равенств положительными:\\
	$-3x_1 + 2x_2 + y_1 = 8$\\
	$3x_1 + x_2 - y_2 = 3$\\
	$5x_2 + y_3 = 8$\\
	$3x_1 + y_4 = 4$\\
	$x_1 \geq 0$, $x_2 \geq 0$\\
	
	Таким образом, задача сведена к канонической форме.
	\newpage
	Отсюда получается:\\
	$y_1 = 8 + 3x_1 - 2x_2$\\
	$y_2 = -3 + 3x_1 + x_2$\\
	$y_3 = 8 - 5x_2$\\
	$y_4 = 4 - 3x_1$\\
	
	Базисное решение:
	\begin{center}
		$y_1 = 8, y_2 = -3, y_3 = 8, y_4 = 4,$\\
		$x_1 = 0, x_2 = 0$
	\end{center}
	которое не удовлетворяет естественным ограничениям:
	\begin{center}
		$y_i \geq 0 \hspace*{0.3cm} \forall i = \overline{1,4}$
	\end{center}
	и поэтому оно не является допустимым.
	
	\section*{Двойственный симлекс-метод}
	\subsection*{1 итерация}
	Базисные переменные: $y_1, y_2, y_3, y_4$.\\
	Свободные переменный: $x_1, x_2$.\\\\
	\begin{tabularx}{\textwidth}{V{1}p{1.45cm}V{1}p{1.45cm}V{5}p{1.45cm}V{5}p{1.45cm}V{1}p{1.45cm}V{1}p{1.45cm}V{1}p{1.4cm}V{1}p{1.4cm}V{1}}
		\hline
		БП & $x_1$ & $\boldsymbol{x_2}$ & $y_1$ & $y_2$ & $y_3$ & $y_4$ & СЧ\\
		\hline
		$f$ & -4 \footnotesize -3 & \underline{-1} \footnotesize -1 & 0 \footnotesize 0 & 0 \footnotesize 1 & 0 \footnotesize 0 & 0 \footnotesize 0 & 0 \footnotesize -3\\
		\hline
		$y_1$ & -3 \footnotesize 6 & \underline{2} \footnotesize 2 & 1 \footnotesize 0 & 0 \footnotesize -2 & 0 \footnotesize 0 & 0 \footnotesize 0 & 8 \footnotesize 6\\
		\Xhline{5\arrayrulewidth}
		$\boldsymbol{y_2}$ & -3 \footnotesize \underline{3} & \textbf{-1} \footnotesize 1 & 0 \footnotesize \underline{0} & 1 \footnotesize \underline{-1} & 0 \footnotesize \underline{0} & 0 \footnotesize \underline{0} & -3 \footnotesize \underline{3}\\
		\Xhline{5\arrayrulewidth}
		$y_3$ & 0 \footnotesize 15 & \underline{5} \footnotesize 5 & 0 \footnotesize 0 & 0 \footnotesize -5 & 1 \footnotesize 0 & 0 \footnotesize 0 & 8 \footnotesize 15\\
		\hline
		$y_4$ & 3 \footnotesize 0 & \underline{0} \footnotesize 0 & 0 \footnotesize 0 & 0 \footnotesize 0 & 0 \footnotesize 0 & 1 \footnotesize 0 & 4 \footnotesize 0\\
		\hline
		& \small $\frac{4}{3}$ & \textbf{\textit{1}} & & & & & \\
		\hdashline
	\end{tabularx}
	\newline\newline
	Меняем свободную переменную $x_2$ и базисную переменную $y_2$ местами.\\
	$x_2 \leftrightarrow y_2$
	\newpage
	\subsection*{2 итерация}
	Базисные переменные: $x_2, y_1, y_3, y_4$.\\
	Свободные переменный: $x_1, y_2$.\\\\
	\begin{tabularx}{\textwidth}{V{1}p{1.45cm}V{5}p{1.45cm}V{5}p{1.45cm}V{1}p{1.45cm}V{1}p{1.45cm}V{1}p{1.45cm}V{1}p{1.4cm}V{1}p{1.4cm}V{1}}
		\hline
		БП & $\boldsymbol{x_1}$ & $x_2$ & $y_1$ & $y_2$ & $y_3$ & $y_4$ & СЧ\\
		\hline
		$f$ & \underline{-1} \footnotesize -1 & 0 \footnotesize 0 & 0 \footnotesize 0 & -1 \footnotesize $\frac{1}{3}$ & 0 \footnotesize $\frac{1}{15}$ & 0 \footnotesize 0 & 3 \footnotesize -$\frac{7}{15}$\\
		\hline
		$y_1$ & \underline{-9} \footnotesize -9 & 0 \footnotesize 0 & 1 \footnotesize 0 & 2 \footnotesize 3 & 0 \footnotesize $\frac{3}{5}$ & 0 \footnotesize 0 & 2 \footnotesize -$\frac{21}{5}$\\
		\hline
		$x_2$ & \underline{3} \footnotesize 3 & 1 \footnotesize 0 & 0 \footnotesize 0 & -1 \footnotesize -1 & 0 \footnotesize -$\frac{1}{5}$ & 0 \footnotesize 0 & 3 \footnotesize $\frac{7}{5}$\\
		\Xhline{5\arrayrulewidth}
		$\boldsymbol{y_3}$ & \textbf{-15} \footnotesize 1 & 0 \footnotesize \underline{0} & 0 \footnotesize \underline{0} & 5 \footnotesize \underline{-$\frac{1}{3}$} & 1 \footnotesize \underline{-$\frac{1}{15}$} & 0 \footnotesize \underline{0} & -7 \footnotesize \underline{$\frac{7}{15}$}\\
		\Xhline{5\arrayrulewidth}
		$y_4$ & \underline{3} \footnotesize 3 & 0 \footnotesize 0 & 0 \footnotesize 0 & 0 \footnotesize -1 & 0 \footnotesize -$\frac{1}{5}$ & 1 \footnotesize 0 & 4 \footnotesize $\frac{7}{5}$\\
		\hline
		& \small $\boldsymbol{\frac{1}{15}}$ & & & & & & \\
		\hdashline
	\end{tabularx}
	\newline\newline
	Меняем свободную переменную $x_1$ и базисную переменную $y_3$ местами.\\
	$x_1 \leftrightarrow y_3$
	\subsection*{3 итерация}
	Базисные переменные: $x_1, x_2, y_1, y_4$.\\
	Свободные переменный: $y_2, y_3$.\\\\
	\begin{tabularx}{\textwidth}{V{1}p{1.2cm}V{1}p{1.2cm}V{1}p{1.2cm}V{1}p{1.45cm}V{5}p{1.45cm}V{5}p{1.45cm}V{1}p{1.45cm}V{1}p{1.46cm}V{1}p{1.46cm}}
		\hline
		БП & $x_1$ & $x_2$ & $y_1$ & $\boldsymbol{y_2}$ & $y_3$ & $y_4$ & СЧ \\
		\hline
		$f$ & 0 \footnotesize 0 & 0 \footnotesize 0 & 0 \footnotesize 0 & \small -$\frac{4}{3}$ \footnotesize -$\frac{4}{3}$ & \small -$\frac{1}{15}$ \footnotesize -$\frac{4}{15}$ & 0 \footnotesize -$\frac{4}{3}$ & \small $\frac{52}{15}$ \footnotesize -$\frac{51}{12}$\\
		\hline
		$y_1$ & 0 \footnotesize 0 & 0 \footnotesize 0 & 1 \footnotesize 0 & -1 \footnotesize -1 & \small -$\frac{3}{5}$ \footnotesize -$\frac{1}{5}$ & 0 \footnotesize -1 & \small $\frac{31}{5}$ \footnotesize -$\frac{13}{5}$\\
		\hline
		$x_2$ & 0 \footnotesize 0 & 1 \footnotesize 0 & 0 \footnotesize 0 & 0 \footnotesize 0& \small $\frac{1}{5}$ \footnotesize 0 & 0 \footnotesize 0 & \small $\frac{8}{5}$ \footnotesize 0\\
		\hline
		$x_1$ & 1 \footnotesize 0 & 0 \footnotesize 0 & 0 \footnotesize 0 & \small \underline{-$\frac{1}{3}$} \footnotesize -$\frac{1}{3}$ & \small -$\frac{1}{15}$ \footnotesize -$\frac{1}{15}$ & 0 \footnotesize -$\frac{1}{5}$ & \small $\frac{7}{15}$ \footnotesize -$\frac{13}{15}$\\
		\Xhline{5\arrayrulewidth}
		$\boldsymbol{y_4}$ & 0 \footnotesize \underline{0} & 0 \footnotesize \underline{0} & 0 \footnotesize \underline{0} & \textbf{1} \footnotesize 1 & \small $\frac{1}{5}$ \footnotesize \underline{$\frac{1}{5}$} & 1 \footnotesize \underline{1} & \small $\frac{13}{5}$ \footnotesize \underline{$\frac{13}{5}$} & \small $\frac{13}{5}$\\
		\Xhline{5\arrayrulewidth}
	\end{tabularx}
	\newline\newline
	Меняем свободную переменную $y_2$ и базисную переменную $y_4$ местами.\\
	$y_2 \leftrightarrow y_4$
	\newpage
	\subsection*{Результат вычислений}
	Базисные переменные: $x_1, x_2, y_1, y_2$.\\
	Свободные переменный: $y_3, y_4$.\\\\
	\begin{tabularx}{\textwidth}{V{1}p{1.45cm}V{1}p{1.45cm}V{1}p{1.45cm}V{1}p{1.45cm}V{1}p{1.45cm}V{1}p{1.45cm}V{1}p{1.4cm}V{1}p{1.52cm}V{1}}
		\hline
		БП & $x_1$ & $x_2$ & $y_1$ & $y_2$ & $y_3$ & $y_4$ & СЧ\\
		\hline
		$f$ & 0 & 0 & 0 & 0 & \small $\frac{1}{5}$ & \small $\frac{4}{3}$ & \small $6\frac{14}{15}$\\
		\hline
		$y_1$ & 0 & 0 & 1 & 0 & \small -$\frac{2}{5}$ & 1 & \small $8\frac{4}{5}$\\
		\hline
		$x_2$ & 0 & 1 & 0 & 0 & \small $\frac{1}{5}$ & 0 & \small $1\frac{3}{5}$\\
		\hline
		$x_1$ & 1 & 0 & 0 & 0 & 0 & \small $\frac{1}{3}$ & \small $1\frac{1}{3}$\\
		\hline
		$y_2$ & 0 & 0 & 0 & 0 & \small $\frac{1}{5}$ & \small $\frac{4}{3}$ & \small $2\frac{3}{5}$\\
		\hline
	\end{tabularx}
	\newline\newline
	Мы получили начальное нецелочисленное решение:\\
	$f = 6\frac{14}{15}$\\
	$x_1 = 1\frac{1}{3}, x_2 = 1\frac{3}{5}$\\
	$y_1 = 8\frac{4}{5}, y_2 = 2\frac{3}{5}, y_3 = 0, y_4 = 0$
	\section*{Метод Гомори}
	\subsection*{1 итерация}
	Наибольшая дробная часть соответствует переменной $\{x_2\} = \frac{3}{5}$, составим соответствующее дополнительное ограничение. Третья строка записывается следующим образом:
	\begin{center}
		$x_2 + \frac{1}{5}y_3 = 1\frac{3}{5} \hspace*{1cm} \Rightarrow \hspace*{1cm} x_2 = 1\frac{3}{5} - \frac{1}{5}y_3$
	\end{center}
	\begin{center}
		$\frac{3}{5} - \frac{1}{5}y_3 \leq 0 \hspace*{0.6cm} \Rightarrow \hspace*{1cm} -y_3 \leq -3$
	\end{center}
	Введём слабую переменную $y_5$:
	\begin{center}
		$-y_3 + y_5 = -3$
	\end{center}
	Решим задачу двойственным симлекс-методом с дополнительным ограничением и возьмём переменную $y_5$ в качестве базисной переменной.
	\newpage
	\subsubsection*{Двойственный симплекс-метод}
	Базисные переменные: $x_1, x_2, y_1, y_2, y_5$.\\
	Свободные переменный: $y_3, y_4$.\\\\
	\begin{tabularx}{\textwidth}{V{1}p{0.7cm}V{1}p{1.3cm}V{1}p{1.34cm}V{1}p{1.3cm}V{1}p{1.3cm}V{4}p{1.3cm}V{5}p{1.3cm}V{1}p{1.3cm}V{1}p{1.21cm}V{1}}
		\hline
		БП & $x_1$ & $x_2$ & $y_1$ & $y_2$ & $\boldsymbol{y_3}$ & $y_4$ & $y_5$ & СЧ\\
		\hline
		$f$ & 0 \footnotesize 0 & 0 \footnotesize 0 & 0 \footnotesize 0 & 0 \footnotesize 0 & \small \underline{$\frac{1}{5}$} \footnotesize $\frac{1}{5}$ & \small $\frac{4}{3}$ \footnotesize 0 & 0 \footnotesize -$\frac{1}{5}$& \small $\frac{104}{15}$ \footnotesize $\frac{3}{5}$\\
		\hline
		$y_1$ & 0 \footnotesize 0 & 0 \footnotesize 0 & 1 \footnotesize 0 & 0 \footnotesize 0 & \small \underline{-$\frac{2}{5}$} \footnotesize -$\frac{2}{5}$ & 1 \footnotesize 0 & 0 \footnotesize $\frac{2}{5}$ & \small $\frac{44}{5}$ \footnotesize -$\frac{6}{5}$\\
		\hline
		$x_2$ & 0 \footnotesize 0 & 1 \footnotesize 0 & 0 \footnotesize 0 & 0 \footnotesize 0 & \small \underline{$\frac{1}{5}$} \footnotesize $\frac{1}{5}$ & 0 \footnotesize 0 & 0 \footnotesize -$\frac{1}{5}$ & \small $\frac{8}{5}$ \footnotesize $\frac{3}{5}$\\
		\hline
		$x_1$ & 1 \footnotesize 0 & 0 \footnotesize 0 & 0 \footnotesize 0 & 0 \footnotesize 0 & \underline{0} \footnotesize 0 & \small $\frac{1}{3}$ \footnotesize 0 & 0 \footnotesize 0 & \small $\frac{4}{3}$ \footnotesize 0\\
		\hline
		$y_2$ & 0 \footnotesize 0 & 0 \footnotesize 0 & 0 \footnotesize 0 & 0 \footnotesize 0 & \small \underline{$\frac{1}{5}$} \footnotesize $\frac{1}{5}$ & \small $\frac{4}{3}$ \footnotesize 0 & 0 \footnotesize -$\frac{1}{5}$ & \small $\frac{13}{5}$ \footnotesize $\frac{3}{5}$\\
		\Xhline{5\arrayrulewidth}
		$\boldsymbol{y_5}$ & 0 \footnotesize \underline{0} & 0 \footnotesize \underline{0} & 0 \footnotesize \underline{0} & 0 \footnotesize \underline{0} & \textbf{-1} \footnotesize 1 & 0 \footnotesize \underline{0} & 1 \footnotesize \underline{-1} & \small -3 \footnotesize \underline{3}\\
		\Xhline{4\arrayrulewidth}
		& & & & & \small $\boldsymbol{\frac{1}{5}}$ & & &\\
	\end{tabularx}
	\newline\newline
	Меняем свободную переменную $y_3$ и базисную переменную $y_5$ местами.\\
	$y_3 \leftrightarrow y_5$
	\subsubsection*{Результат вычислений}
	Базисные переменные: $x_1, x_2, y_1, y_2, y_3$.\\
	Свободные переменный: $y_4, y_5$.\\\\
	\begin{tabularx}{\textwidth}{V{1}p{0.7cm}V{1}p{1.3cm}V{1}p{1.34cm}V{1}p{1.3cm}V{1}p{1.3cm}V{1}p{1.3cm}V{1}p{1.3cm}V{1}p{1.3cm}V{1}p{1.33cm}V{1}}
		\hline
		БП & $x_1$ & $x_2$ & $y_1$ & $y_2$ & $y_3$ & $y_4$ & $y_5$ & СЧ\\
		\hline
		$f$ & 0 & 0 & 0 & 0 & 0 & \small $\frac{4}{3}$ & \small $\frac{1}{5}$ & \small $\frac{19}{3}$\\
		\hline
		$y_1$ & 0 & 0 & 1 & 0 & 0 & 1 & \small -$\frac{2}{5}$ & 10\\
		\hline
		$x_2$ & 0 & 1 & 0 & 0 & 0 & 0 & \small $\frac{1}{5}$ & 1\\
		\hline
		$x_1$ & 1 & 0 & 0 & 0 & 0 & \small $\frac{1}{3}$ & 0 & \small $\frac{4}{3}$\\
		\hline
		$y_2$ & 0 & 0 & 0 & 0 & 0 & \small $\frac{4}{3}$ & \small $\frac{1}{5}$ & 2\\
		\hline
		$y_3$ & 0 & 0 & 0 & 0 & 1 & 0 & -1 & 3\\
		\hline
	\end{tabularx}
	\newpage
	\subsection*{2 итерация}
	Наибольшая дробная часть соответствует переменной $\{x_1\} = \frac{1}{3}$, составим соответствующее дополнительное ограничение. Четвёртая строка записывается следующим образом:
	\begin{center}
		$x_1 + \frac{1}{3}y_4 = 1\frac{1}{3} \hspace*{1cm} \Rightarrow \hspace*{1cm} x_1 = 1\frac{1}{3} - \frac{1}{3}y_4$
	\end{center}
	\begin{center}
		$\frac{1}{3} - \frac{1}{3}y_4 \leq 0 \hspace*{0.6cm} \Rightarrow \hspace*{1cm} -y_4 \leq -1$
	\end{center}
	Введём слабую переменную $y_6$:
	\begin{center}
		$-y_4 + y_6 = -1$
	\end{center}
	Решим задачу двойственным симлекс-методом с дополнительным ограничением и возьмём переменную $y_6$ в качестве базисной переменной.
	\subsubsection*{Двойственный симплекс-метод}
	Базисные переменные: $x_1, x_2, y_1, y_2, y_3, y_6$.\\
	Свободные переменный: $y_4, y_5$.\\\\
	\begin{tabularx}{\textwidth}{V{1}p{0.7cm}V{1}p{1.1cm}V{1}p{1.1cm}V{1}p{1.1cm}V{1}p{1.1cm}V{1}p{1.1cm}V{4}p{1.1cm}V{5}p{1.1cm}V{1}p{1.1cm}V{1}p{1.15cm}V{1}}
		\hline
		БП & $x_1$ & $x_2$ & $y_1$ & $y_2$ & $y_3$ & $\boldsymbol{y_4}$ & $y_5$ & $y_6$ & СЧ\\
		\hline
		$f$ & 0 \footnotesize 0 & 0 \footnotesize 0 & 0 \footnotesize 0 & 0 \footnotesize 0 & 0 \footnotesize 0 & \small \underline{$\frac{4}{3}$} \footnotesize $\frac{4}{3}$ & \small $\frac{1}{5}$ \footnotesize 0 & 0 \footnotesize -$\frac{4}{3}$ & \small $\frac{19}{3}$ \footnotesize $\frac{4}{3}$\\
		\hline
		$y_1$ & 0 \footnotesize 0 & 0 \footnotesize 0 & 1 \footnotesize 0 & 0 \footnotesize 0 & 0 \footnotesize 0 & \underline{1} \footnotesize 1 & \small -$\frac{2}{5}$ \footnotesize 0 & 0 \footnotesize -1 & 10 \footnotesize 1\\
		\hline
		$x_2$ & 0 \footnotesize 0 & 1 \footnotesize 0 & 0 \footnotesize 0 & 0 \footnotesize 0 & 0 \footnotesize 0 & \underline{0} \footnotesize 0 & \small $\frac{1}{5}$ \footnotesize 0 & 0 \footnotesize 0 & 1 \footnotesize 0\\
		\hline
		$x_1$ & 1 \footnotesize 0 & 0 \footnotesize 0 & 0 \footnotesize 0 & 0 \footnotesize 0 & 0 \footnotesize 0 & \small \underline{$\frac{1}{3}$} \footnotesize $\frac{1}{3}$ & 0 \footnotesize 0 & 0 \footnotesize -$\frac{1}{3}$ & \small $\frac{4}{3}$ \footnotesize $\frac{1}{3}$\\
		\hline
		$y_2$ & 0 \footnotesize 0 & 0 \footnotesize 0 & 0 \footnotesize 0 & 0 \footnotesize 0 & 0 \footnotesize 0 & \small \underline{$\frac{4}{3}$} \footnotesize $\frac{4}{3}$ & \small $\frac{1}{5}$ \footnotesize 0 & 0 \footnotesize -$\frac{4}{3}$ & 2 \footnotesize $\frac{4}{3}$\\
		\hline
		$y_3$ & 0 \footnotesize 0 & 0 \footnotesize 0 & 0 \footnotesize 0 & 0 \footnotesize 0 & 1 \footnotesize 0 & \underline{0} \footnotesize 0 & -1 \footnotesize 0 & 0 \footnotesize 0 & 3 \footnotesize 0\\
		\Xhline{4\arrayrulewidth}
		$\boldsymbol{y_6}$ & 0 \footnotesize \underline{0} & 0 \footnotesize \underline{0} & 0 \footnotesize \underline{0} & 0 \footnotesize \underline{0} & 0 \footnotesize \underline{0} & \textbf{-1} \footnotesize 1 & 0 \footnotesize \underline{0} & 1 \footnotesize \underline{-1} & -1 \footnotesize \underline{1}\\
		\Xhline{4\arrayrulewidth}
		& & & & & & $\boldsymbol{\frac{4}{3}}$& & & \\
	\end{tabularx}
	\newline\newline
	Меняем свободную переменную $y_4$ и базисную переменную $y_6$ местами.\\
	$y_4 \leftrightarrow y_6$
	\newpage
	\subsubsection*{Результат вычислений}
	Базисные переменные: $x_1, x_2, y_1, y_2, y_3, y_4$.\\
	Свободные переменный: $y_5, y_6$.\\\\
	\begin{tabularx}{\textwidth}{V{1}p{0.7cm}V{1}p{1.1cm}V{1}p{1.1cm}V{1}p{1.1cm}V{1}p{1.1cm}V{1}p{1.1cm}V{1}p{1.1cm}V{1}p{1.1cm}V{1}p{1.1cm}V{1}p{1.23cm}V{1}}
		\hline
		БП & $x_1$ & $x_2$ & $y_1$ & $y_2$ & $y_3$ & $y_4$ & $y_5$ & $y_6$ & СЧ\\
		\hline
		$f$ & 0 & 0 & 0 & 0 & 0 & 0 & \small $\frac{1}{5}$ & \small $\frac{4}{3}$ & 5\\
		\hline
		$y_1$ & 0 & 0 & 1 & 0 & 0 & 0 & \small -$\frac{2}{5}$ & 1 & 9\\
		\hline
		$x_2$ & 0 & 1 & 0 & 0 & 0 & 0 & \small $\frac{1}{5}$ & 0 & 1 \\
		\hline
		$x_1$ & 1 & 0 & 0 & 0 & 0 & 0 & 0 & \small $\frac{1}{3}$ & 1\\
		\hline
		$y_2$ & 0 & 0 & 0 & 0 & 0 & 0 & \small $\frac{1}{5}$ & $\frac{4}{3}$ & \small $\frac{2}{3}$\\
		\hline
		$y_3$ & 0 & 0 & 0 & 0 & 1 & 0 & -1 & 0 & 3\\
		\hline
		$y_4$ & 0 & 0 & 0 & 0 & 0 & 1 & 0 & -1 & 1\\
		\hline
	\end{tabularx}
	\newline\newline
	Таким образом, мы получили целочисленное решение:
	\begin{center}
		$f = 5, x_1 = 1, x_2 = 1$
	\end{center}
	\textbf{Ответ:} $f = 5, x_1 = 1, x_2 = 1$.
\end{document}