\documentclass[14pt,a4paper,fleqn]{extarticle}
\usepackage[T2A,T1]{fontenc}
\usepackage[utf8]{inputenc}
\usepackage[russian]{babel}
\usepackage{amsmath}
\usepackage{graphicx}
\usepackage{tabularx}
\usepackage{boldline}
\usepackage{makecell}
\usepackage{arydshln}
\usepackage{mathtools}

\graphicspath{ {./images/} }
\setlength{\mathindent}{0pt}
\setlength\parindent{0pt}


\begin{document}
	\begin{titlepage}
		\includegraphics[scale=0.12]{logo}
		\begin{center}
			\textbf{МИНОБРНАУКИ РОССИИ}\\
			\vspace{0.2cm}
			\textbf{Федеральное государственное бюджетное образовательное учреждение высшего образования}\\
			\textbf{«САНКТ-ПЕТЕРБУРГСКИЙ ГОСУДАРСТВЕННЫЙ ЭКОНОМИЧЕСКИЙ УНИВЕРСИТЕТ»}\\
			\vspace{0.6cm}
			Факультет информатики и прикладной математики\\
			Кафедра прикладной математики и экономико-математических методов\\
			\vspace{1cm}
			\textbf{ОТЧЁТ}\\
			по дисциплине:\\
			\textbf{«Методы оптимизации»}\\
			на тему:\\
			\textbf{«Задание 12. Нелинейная оптимизация»}\\
		\end{center}
		\vspace{1cm}
		Направление: 01.03.02\\
		Обучающийся: Бронников Егор Игоревич\\
		Группа: ПМ-1901\\
		\vfill
		\begin{center}
			Санкт-Петербург\\
			2021\\
		\end{center}
	\end{titlepage}
	\section*{Дано}
	$f(x_1,x_2) = 3x_1^2 - 2x_1x_2 + 4x_2^2 - 3x_1 + 2x_2 + 5$
	\section*{Задание 1}
	\textbf{Условие:}\\
	Найти стационарную точку и вычислить в ней значение функции.\\
	
	\textbf{Решение:}\\
	
	\scalebox{1}{
	$\begin{cases}
		\vspace{0.3cm}
		\dfrac{df}{dx_1} = 6x_1 - 2x_2 - 3 = 0\\
		\dfrac{df}{dx_2} = -2x_1 + 8x_2 + 2 = 0
	\end{cases}$}\\\\

	Стационарная точка: $(\dfrac{5}{11}$, $-\dfrac{3}{22})$\\
	Значение функции в стационарной точке: $f(\dfrac{5}{11}$, $-\dfrac{3}{22}) = \dfrac{46}{11}$
	
	\section*{Задание 2}
	\textbf{Условие:}\\
	Определить экстремум, если он есть.\\

	\textbf{Решение:}\\
	
	$\dfrac{d^2f}{dx_1^2} = 6$\\\\
	$\dfrac{d^2f}{dx_2^2} = 8$\\\\
	$\dfrac{d^2f}{dx_1x_2} = -2$\\
	
	Построим матрицу Гессе:
	\begin{align*}
		H(X) = \begin{pmatrix}
			6 & -2\\
			-2 & 8
		\end{pmatrix}
	\end{align*}
	
	Вычисляем главные миноры:\\
	$M_1(H) = 6 > 0, \hspace*{0.2cm} M_2(H) = 44 > 0$\\
	
	В точке $(\dfrac{5}{11}$, $-\dfrac{3}{22})$ матрица Гессе положительна определена, значит точка $(\dfrac{5}{11}$, $-\dfrac{3}{22})$ -- экстремум и является точкой минимума.
	
	\section*{Задание 3}
	\textbf{Условие:}\\
	Проверить функцию на выпуклость/вогнутость.\\
	
	\textbf{Решение:}\\
	
	В точке $(\dfrac{5}{11}$, $-\dfrac{3}{22})$ матрица Гессе положительно определена, а значит функция является выпуклой.
\end{document}