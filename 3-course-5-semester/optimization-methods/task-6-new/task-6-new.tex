\documentclass[14pt,a4paper,fleqn]{extarticle}
\usepackage[T2A,T1]{fontenc}
\usepackage[utf8]{inputenc}
\usepackage[russian]{babel}
\usepackage{amsmath}
\usepackage{graphicx}
\usepackage{tabularx}
\usepackage{boldline}
\usepackage{makecell}
\usepackage{arydshln}

\graphicspath{ {./images/} }
\setlength{\mathindent}{0pt}
\setlength\parindent{0pt}


\begin{document}
	\begin{titlepage}
		\includegraphics[scale=0.12]{logo}
		\begin{center}
			\textbf{МИНОБРНАУКИ РОССИИ}\\
			\vspace{0.2cm}
			\textbf{Федеральное государственное бюджетное образовательное учреждение высшего образования}\\
			\textbf{«САНКТ-ПЕТЕРБУРГСКИЙ ГОСУДАРСТВЕННЫЙ ЭКОНОМИЧЕСКИЙ УНИВЕРСИТЕТ»}\\
			\vspace{0.6cm}
			Факультет информатики и прикладной математики\\
			Кафедра прикладной математики и экономико-математических методов\\
			\vspace{1cm}
			\textbf{ОТЧЁТ}\\
			по дисциплине:\\
			\textbf{«Методы оптимизации»}\\
			на тему:\\
			\textbf{«Решение задачи линейного программирования двойственным симплекс-методом. Задание 6»}\\
		\end{center}
		\vspace{1cm}
		Направление: 01.03.02\\
		Обучающийся: Бронников Егор Игоревич\\
		Группа: ПМ-1901\\
		\vfill
		\begin{center}
			Санкт-Петербург\\
			2021\\
		\end{center}
	\end{titlepage}
	Рассмотрим и решим двойственную задачу для прямой задачи 2.1.
	\noindent\makebox[\linewidth]{\rule{\paperwidth}{0.4pt}}\\
	\textbf{Напоминание из задания 5}\\\\
	Целевая функция:\\
	$f = -x_1+x_2 \longrightarrow max$\\\\
	Ограничения:
	\begin{align*}
		\begin{cases}
			-x_1 + 2x_2 \geq -1\\
			-2x_1 + x_2 \leq 2\\
			3x_1 + x_2 \leq 3\\
		\end{cases}
	\end{align*}
	$x_1 \geq 0, x_2 \geq 0$
	\subsection*{Задание}
	\subsubsection*{Каноническая форма прямой задачи}
	1. Вводим слабые переменные $y_1 \geq 0, y_2 \geq 0, y_3 \geq 0$:\\
	$-x_1 + 2x_2 - y_1 = -1$\\
	$-2x_1 + x_2 + y_2 = 2$\\
	$3x_1 + x_2 + y_3 = 3$\\
	
	2. Делаем правые части равенств положительными:\\
	$x_1 - 2x_2 + y_1 = 1$\\
	$-2x_1 + x_2 + y_2 = 2$\\
	$3x_1 + x_2 + y_3 = 3$\\
	
	Таким образом, прямая задача сведена к канонической форме.
	\newpage
	\subsubsection*{Формулируем двойственную задачу}
	Функция цели:\\
	$\phi = -\lambda_1 + 2\lambda_2 + 3\lambda_3 \longrightarrow min$\\\\
	Ограничения:\\
	$\lambda_1 - 2\lambda_2 + 3\lambda_3 \geq -1$\\
	$-2\lambda_1 + \lambda_2 + \lambda_3 \geq 1$\\
	$\lambda_1 \geq 0, \lambda_2 \geq 0, \lambda_3 \geq 0$\\
	\noindent\makebox[\linewidth]{\rule{\paperwidth}{0.4pt}}
	\section*{Дано}
	Функция цели:\\
	$\phi = -\lambda_1 + 2\lambda_2 + 3\lambda_3 \longrightarrow min$\\\\
	$\lambda_1 - 2\lambda_2 + 3\lambda_3 \geq -1$\\
	$-2\lambda_1 + \lambda_2 + \lambda_3 \geq 1$\\
	$\lambda_1 \geq 0, \lambda_2 \geq 0, \lambda_3 \geq 0$
	\section*{Задание}
	\subsection*{Каноническая форма}
	Функция цели:\\
	$\phi = -\lambda_1 + 2\lambda_2 + 3\lambda_3 \longrightarrow min$\\\\
	1. Вводим слабые переменные $\xi_1 \geq 0, \xi_2 \geq 0$:\\
	$\lambda_1 - 2\lambda_2 + 3\lambda_3 - \xi_1 = -1$\\
	$-2\lambda_1 + \lambda_2 + \lambda_3 - \xi_2 = 1$\\
	$\lambda_1 \geq 0, \lambda_2 \geq 0, \lambda_3 \geq 0$
	\newpage
	2. Делаем правые части равенств положительными:\\
	$-\lambda_1 + 2\lambda_2 - 3\lambda_3 + \xi_1 = 1$\\
	$-2\lambda_1 + \lambda_2 + \lambda_3 - \xi_2 = 1$\\
	$\lambda_1 \geq 0, \lambda_2 \geq 0, \lambda_3 \geq 0$\\
	
	Таким образом, задача сведена к канонической форме.\\\\
	Отсюда получается:\\
	$\xi_1 = 1 + \lambda_1 - 2\lambda_2 + 3\lambda_3$\\
	$\xi_2 = -1 - 2\lambda_1 + \lambda_2 + \lambda_3$\\\\
	Базисное решение:
	\begin{center}
		$\xi_1 = 1, \xi_2 = -1, \lambda_i = 0 \hspace*{0.3cm} \forall i = \overline{1,3}$
	\end{center}
	которое не удовлетворяет естественным ограничениям:
	\begin{center}
		$\xi_i \geq 0 \hspace*{0.3cm} \forall i = \overline{1,2}$
	\end{center}
	и поэтому оно не является допустимым.
	\subsection*{Двойственный симлекс-метод}
	\subsubsection*{1 итерация}
	Базисные переменные: $\xi_1, \xi_2$.\\
	Свободные переменный: $\lambda_1, \lambda_2, \lambda_3$.\\\\
	\begin{tabularx}{\textwidth}{V{1}p{1.7cm}V{1}p{1.7cm}V{5}p{1.7cm}V{4}p{1.7cm}V{1}p{1.7cm}V{1}p{1.7cm}V{1}p{1.74cm}V{1}}
		\hline
		БП & $\lambda_1$ & $\boldsymbol{\lambda_2}$ & $\lambda_3$ & $\xi_1$ & $\xi_2$ & СЧ\\
		\hline
		$\phi$ & 1 \footnotesize 4 & \underline{-2} \scriptsize -2 & -3 \footnotesize -2 & 0 \footnotesize 0 & 0 \footnotesize 2 & 0 \footnotesize -2\\
		\hline
		$\xi_1$ & -1 \footnotesize -4 & \underline{2} \scriptsize 2 & -3 \footnotesize 2 & 1 \footnotesize 0 & 0 \footnotesize -2 & 1 \footnotesize 2\\
		\Xhline{5\arrayrulewidth}
		$\boldsymbol{\xi_2}$ & 2 \footnotesize \underline{-2} & \textbf{-1} \scriptsize 1 & -1 \footnotesize \underline{1} & 0 \footnotesize \underline{0} & 1 \footnotesize \underline{-1} & -1 \footnotesize \underline{1}\\
		\Xhline{5\arrayrulewidth}
		& & \textbf{\textit{2}} & 3 & & &\\
		\hdashline
	\end{tabularx}
	\newline\newline
	Меняем свободную переменную $\lambda_2$ и базисную переменную $\xi_2$ местами.\\
	$\lambda_2 \leftrightarrow \xi_2$
	\newpage
	\subsubsection*{2 итерация}
	Базисные переменные: $\xi_1, \lambda_2$.\\
	Свободные переменный: $\lambda_1, \lambda_3, \xi_2$.\\\\
	\begin{tabularx}{\textwidth}{V{1}p{1.7cm}V{1}p{1.7cm}V{1}p{1.7cm}V{5}p{1.7cm}V{5}p{1.7cm}V{1}p{1.7cm}V{1}p{1.74cm}V{1}}
		\hline
		БП & $\lambda_1$ & $\lambda_2$ & $\boldsymbol{\lambda_3}$ & $\xi_1$ & $\xi_2$ & СЧ\\
		\hline
		$\phi$ & -3 \footnotesize $\frac{3}{5}$ & 0 \scriptsize 0 & \underline{-1} \footnotesize -1 & 0 \footnotesize $\frac{1}{5}$ & -2 \footnotesize $\frac{2}{5}$ & 2 \footnotesize -$\frac{1}{5}$\\
		\Xhline{5\arrayrulewidth}
		$\boldsymbol{\xi_1}$ & 3 \footnotesize \underline{-$\frac{3}{5}$} & 0 \scriptsize \underline{0} & \textbf{-5} \footnotesize 1 & 1 \footnotesize \underline{-$\frac{1}{5}$} & 2 \footnotesize \underline{-$\frac{2}{5}$} & -1 \footnotesize \underline{$\frac{1}{5}$}\\
		\Xhline{5\arrayrulewidth}
		$\lambda_2$ & -2 \footnotesize -$\frac{3}{5}$ & 1 \scriptsize 0 & \underline{1} \footnotesize 1 & 0 \footnotesize -$\frac{1}{5}$ & -1 \footnotesize -$\frac{2}{5}$ & 1 \footnotesize $\frac{1}{5}$\\
		\hdashline
		& & & \small $\boldsymbol{\frac{1}{5}}$ & & &\\
		\hdashline
	\end{tabularx}
	\newline\newline
	Меняем свободную переменную $\lambda_3$ и базисную переменную $\xi_1$ местами.\\
	$\lambda_3 \leftrightarrow \xi_1$
	\subsubsection*{Результаты вычислений}
	Базисные переменные: $\lambda_2, \lambda_3$.\\
	Свободные переменный: $\lambda_1, \xi_1, \xi_2$.\\\\
	\begin{tabularx}{\textwidth}{V{1}p{1.7cm}V{1}p{1.7cm}V{1}p{1.7cm}V{1}p{1.7cm}V{1}p{1.7cm}V{1}p{1.7cm}V{1}p{1.85cm}V{1}}
		\hline
		БП & $\lambda_1$ & $\lambda_2$ & $\lambda_3$ & $\xi_1$ & $\xi_2$ & СЧ\\
		\hline
		$\phi$ & \small -$\frac{18}{5}$ & 0 & 0 & \small -$\frac{1}{5}$ & \small -$\frac{12}{5}$ & \small $\frac{11}{5}$\\
		\hline
		$\lambda_3$ & \small -$\frac{3}{5}$ & 0 & 1 & \small -$\frac{1}{5}$ & \small -$\frac{2}{5}$ & \small $\frac{1}{5}$\\
		\hline
		$\lambda_2$ & \small -$\frac{7}{5}$ & 1 & 1 & \small $\frac{1}{5}$ & \small -$\frac{3}{5}$ & \small $\frac{4}{5}$\\
		\hline
	\end{tabularx}
	\newline\newline
	Таким образом, получается:\\
	$\phi = \frac{11}{5}$\\
	$\lambda_2 = \frac{4}{5}, \lambda_3 = \frac{1}{5}$\\
	$\lambda_1 = 0, \xi_1 = 0, \xi_2 = 0$\\
	\noindent\makebox[\linewidth]{\rule{\paperwidth}{0.4pt}}\\
	\textbf{Напоминание из задания 5}
	\begin{center}
		...
	\end{center}
	Тогда решение двойственной задачи выглядит следующим образом:\\
	$\lambda_1 = 0, \lambda_2 = \frac{4}{5}, \lambda_3 = \frac{1}{5}$\\\\
	Функция цели:\\
	$\phi = -\lambda_1 + 2\lambda_2 + 3\lambda_3 = \frac{8}{5} + \frac{3}{5} = \frac{11}{5}$
	\begin{center}$\downarrow$\end{center}
	$\phi = \frac{11}{5} = 2.2$\\
	\noindent\makebox[\linewidth]{\rule{\paperwidth}{0.4pt}}\\\\
	\textbf{Ответ:} $\lambda_1 = 0, \lambda_2 = \frac{4}{5}, \lambda_3 = \frac{1}{5}, \phi = \frac{11}{5}$
\end{document}