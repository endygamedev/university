\documentclass[14pt,a4paper,fleqn]{extarticle}
\usepackage[T2A,T1]{fontenc}
\usepackage[utf8]{inputenc}
\usepackage[russian]{babel}
\usepackage{amsmath}
\setlength{\mathindent}{0pt}
\setlength\parindent{0pt}


\begin{document}
\begin{titlepage}
	\begin{center}
		\vspace*{1cm}
		\textbf{МИНОБРНАУКИ РОССИИ}\\
		\vspace{0.2cm}
		\textbf{Федеральное государственное бюджетное образовательное учреждение высшего образования}\\
		\textbf{«САНКТ-ПЕТЕРБУРГСКИЙ ГОСУДАРСТВЕННЫЙ ЭКОНОМИЧЕСКИЙ УНИВЕРСИТЕТ»}\\
		\vspace{0.6cm}
		Факультет информатики и прикладной математики\\
		Кафедра прикладной математики и экономико-математических методов\\
		\vspace{1cm}
		\textbf{ОТЧЁТ}\\
		по дисциплине:\\
		\textbf{«Методы оптимизации»}\\
		на тему:\\
		\textbf{«Решение задачи линейного программирования симплекс-методом. Вариант 3.1»}\\
	\end{center}
	\vspace{1cm}
	Направление: 01.03.02\\
	Обучающийся: Бронников Егор Игоревич\\
	Группа: ПМ-1901\\
	\vfill
	\begin{center}
		Санкт-Петербург\\
		2021\\
	\end{center}
\end{titlepage}

\section*{Дано}
Целевая функция:\\
$f = -x_1+x_2 \longrightarrow max$\\\\
Ограничения:
\begin{align*}
	\begin{cases}
		-x_1 + 2x_2 \geq -1\\
		-2x_1 + x_2 \leq 2\\
		3x_1 + x_2 \leq 3\\
	\end{cases}
\end{align*}
$x_1 \geq 0, x_2 \geq 0$

\section*{Задание}
\subsection*{Стандратная форма}
Целевая функция:\\
$f = -x_1+x_2 \longrightarrow max$\\\\
Ограничения:
\begin{align*}
	\begin{cases}
		x_1 - 2x_2 \leq 1\\
		-2x_1 + x_2 \leq 2\\
		3x_1 + x_2 \leq 3\\
	\end{cases}
\end{align*}
$x_1 \geq 0, x_2 \geq 0$
\newpage
\subsection*{Каноническая форма}
Целевая функция:\\
$f = -x_1+x_2 \longrightarrow max$\\\\
Ограничения:
\begin{align*}
	\begin{cases}
		-x_1 + 2x_2 \geq -1\\
		-2x_1 + x_2 \leq 2\\
		3x_1 + x_2 \leq 3\\
	\end{cases}
\end{align*}
$x_1 \geq 0, x_2 \geq 0$\\\\
1. Вводим слабые переменные $y_1 \geq 0, y_2 \geq 0, y_3 \geq 0$:\\
$-x_1 + 2x_2 - y_1 = -1$\\
$-2x_1 + x_2 + y_2 = 2$\\
$3x_1 + x_2 + y_3 = 3$\\

2. Делаем правые части равенств положительными:\\
$x_1 - 2x_2 + y_1 = 1$\\
$-2x_1 + x_2 + y_2 = 2$\\
$3x_1 + x_2 + y_3 = 3$\\

Таким образом, задача сведена к канонической форме.

\subsection*{Матричная форма}
$A \times X^T = B^T$, где:
\begin{align*}
	A = \begin{pmatrix}
		1 & -2 & 1 & 0 & 0\\
		-2 & 1 & 0 & 1 & 0\\
		3 & 1 & 0 & 0 & 1\\
		\end{pmatrix}
\end{align*}
\begin{align*}
	X = \begin{pmatrix}
		x_1 & x_2 & y_1 & y_2 & y_3
		\end{pmatrix}
\end{align*}
\begin{align*}
	B = \begin{pmatrix}
		1 & 2 & 3
	\end{pmatrix}
\end{align*}
\subsection*{Симплекс-метод}
Целевая функция:\\
$f = -x_1 + x_2 \longrightarrow max$\\\\
Ограничения в канонической форме:\\
$x_1 - 2x_2 + y_1 = 1$\\
$-2x_1 + x_2 + y_2 = 2$\\
$3x_1 + x_2 + y_3 = 3$\\\\
$x_1 \geq 0, x_2 \geq 0, y_1 \geq 0, y_2 \geq 0, y_3 \geq 0$\\\\
$P = N - M = 5 - 3 = 2$ --- свободные переменные\\
$M = 3$ --- базисные переменные
\subsubsection*{1 итерация}
$x_1, x_2$ --- свободные переменные\\
$y_1, y_2, y_3$ --- базисные переменные\\

1. Выразим базисные переменные через свободные:\\
$y_1 = 1 - x_1 + 2x_2$\\
$y_2 = 2 + 2x_1 - x_2$\\
$y_3 = 3 - 3x_1 - x_2$\\

2. Выразим функцию цели $f$ через свободные переменные:\\
$f = -x_1 + x_2 \longrightarrow max$\\

3. Вычисляем базисное решение и функцию цели ($x_1 = 0, x_2 = 0$):\\
$y_1 = 1$\\
$y_2 = 2$\\
$y_3 = 3$\\
$f = 0$
\newpage
4.Проанализируем функцию цели и выберем наращиваемую переменную:\\
$f = -x_1 + x_2 \longrightarrow max$\\\\
Будем наращивать $x_2 \uparrow$, тогда $x_1 = 0$\\\\
$y_1 = 1 + 2x_2 = 0 \hspace{0.3cm} \Rightarrow x_2 = -\frac{1}{2} < 0$ --- не интересует\\
$y_2 = 2 - x_2 = 0 \hspace{0.5cm} \Rightarrow x_2 = 2$\\
$y_3 = 3 - x_2 = 0 \hspace{0.5cm} \Rightarrow x_2 = 3$\\\\
$min\{2, 3\} = 2$, тогда меняем свободную переменную $x_2$ и базисную переменную $y_2$ местами:\\
$x_2 \leftrightarrow y_2$\\
$y_2 = 2 + 2x_1 - x_2 \leftrightarrow x_2 = 2 + 2x_1 - y_2$

\subsubsection*{2 итерация}
$x_1, y_2$ --- свободные переменные\\
$x_2, y_1, y_3$ --- базисные переменные\\

1. Выразим базисные переменные через свободные:\\
$x_2 = 2 + 2x_1 - y_2$\\
$y_1 = 1 - x_1 + 2x_2 = 1 - x_1 + 4 + 4x_1 - 2y_2 = 5 + 3x_1 - 2y_2$\\
$y_3 = 3 - 3x_1 - x_2 = 3 - 3x_1 - 2 - 2x_1 + y_2 = 1 - 5x_1 + y_2$
\begin{center}$\downarrow$\end{center}
$x_2 = 2 + 2x_1 - y_2$\\
$y_1 = 5 + 3x_1 - 2y_2$\\
$y_3 = 1 - 5x_1 + y_2$\\
\newpage

2. Выразим функцию цели $f$ через свободные переменные:\\
$f = -x_1 + x_2 = x_1 + 2 + 2x_1 - y_2 = x_1 - y_2 + 2 \longrightarrow max$
\begin{center}$\downarrow$\end{center}
$f = x_1 - y_2 + 2 \longrightarrow max$\\

3. Вычисляем базисное решение и функцию цели ($x_1 = 0, y_2 = 0$):\\
$x_2 = 2$\\
$y_1 = 5$\\
$y_3 = 1$\\
$f = 2$\\

4.Проанализируем функцию цели и выберем наращиваемую переменную:\\
$f = x_1 - y_2 + 2 \longrightarrow max$\\\\
Будем наращивать $x_1 \uparrow$, тогда $y_2 = 0$\\\\
$x_2 = 2 + 2x_1 = 0 \hspace{0.5cm} \Rightarrow x_1 = -1 < 0$ --- не интересует\\
$y_1 = 5 + 3x_1 = 0 \hspace{0.5cm} \Rightarrow x_1 = -\frac{3}{5} < 0$ --- не интересует\\
$y_3 = 1 - 5x_1 = 0 \hspace{0.5cm} \Rightarrow x_1 = \frac{1}{5}$\\\\
Меняем свободную переменную $x_1$ и базисную переменную $y_3$ местами:\\
$x_1 \leftrightarrow y_3$\\
$y_3 = 1 - 5x_1 + y_2 \leftrightarrow x_1 = \frac{1}{5} + \frac{y_2}{5} - \frac{y_3}{5}$

\subsubsection*{3 итерация}
$y_2, y_3$ --- свободные переменные\\
$x_1, x_2, y_1$ --- базисные переменные\\

1. Выразим базисные переменные через свободные:\\
$x_1 = \frac{1}{5} + \frac{y_2}{5} - \frac{y_3}{5}$\\
$x_2 = 2 + 2x_1 - y_2 = \frac{12}{5} - \frac{3}{5}y_2 - \frac{2}{5}y_3$\\
$y_1 = 5 + 3x_1 - 2y_2 = \frac{28}{5} - \frac{7}{5}y_2 - \frac{3}{5}y_3$\\
\begin{center}$\downarrow$\end{center}
$x_1 = \frac{1}{5} + \frac{y_2}{5} - \frac{y_3}{5}$\\
$x_2 = \frac{12}{5} - \frac{3}{5}y_2 - \frac{2}{5}y_3$\\
$y_1 = \frac{28}{5} - \frac{7}{5}y_2 - \frac{3}{5}y_3$\\

2. Выразим функцию цели $f$ через свободные переменные:\\
$f = x_1 - y_2 + 2 = \frac{11}{5} - \frac{4}{5}y_2 - \frac{y_3}{5} \longrightarrow max$
\begin{center}$\downarrow$\end{center}
$f = \frac{11}{5} - \frac{4}{5}y_2 - \frac{y_3}{5} \longrightarrow max$\\

3. Вычисляем базисное решение и функцию цели ($y_2 = 0, y_3 = 0$):\\
$x_1 = \frac{1}{5}$\\
$x_2 = \frac{12}{5}$\\
$y_1 = \frac{28}{5}$\\
$f =  \frac{11}{5}$\\

4.Проанализируем функцию цели и выберем наращиваемую переменную:\\
$f = \frac{11}{5} - \frac{4}{5}y_2 - \frac{y_3}{5} \longrightarrow max$\\\\
Коэффициенты при свободных переменных -- отрицательные, следовательно итерационный процесс окончен.\\
$y_2 = 0, y_3 = 0$\\
$x_1 = \frac{1}{5} = 0.2$\\
$x_2 = \frac{12}{5} = 2.4$\\
$y_1 = \frac{28}{5} = 5.6$\\
$f = \frac{11}{5} = 2.2$\\\\
\textbf{Ответ:} $x_1 = 0.2, x_2 = 2.4, f = 2.2$

\end{document}