\documentclass[14pt,a4paper,fleqn]{extarticle}
\usepackage[T2A,T1]{fontenc}
\usepackage[utf8]{inputenc}
\usepackage[russian]{babel}
\usepackage{amsmath}
\usepackage{graphicx}
\usepackage{tabularx}
\usepackage{boldline}
\usepackage{makecell}

\graphicspath{ {./images/} }
\setlength{\mathindent}{0pt}
\setlength\parindent{0pt}


\begin{document}
	\begin{titlepage}
		\includegraphics[scale=0.12]{logo}
		\begin{center}
			\textbf{МИНОБРНАУКИ РОССИИ}\\
			\vspace{0.2cm}
			\textbf{Федеральное государственное бюджетное образовательное учреждение высшего образования}\\
			\textbf{«САНКТ-ПЕТЕРБУРГСКИЙ ГОСУДАРСТВЕННЫЙ ЭКОНОМИЧЕСКИЙ УНИВЕРСИТЕТ»}\\
			\vspace{0.6cm}
			Факультет информатики и прикладной математики\\
			Кафедра прикладной математики и экономико-математических методов\\
			\vspace{1cm}
			\textbf{ОТЧЁТ}\\
			по дисциплине:\\
			\textbf{«Методы оптимизации»}\\
			на тему:\\
			\textbf{«Решение задачи линейного программирования табличным симплекс-методом. Вариант 4.1»}\\
		\end{center}
		\vspace{1cm}
		Направление: 01.03.02\\
		Обучающийся: Бронников Егор Игоревич\\
		Группа: ПМ-1901\\
		\vfill
		\begin{center}
			Санкт-Петербург\\
			2021\\
		\end{center}
	\end{titlepage}
\section*{Дано}
Целевая функция:\\
$f = 4x_1+x_2 \longrightarrow max$\\\\
Ограничения:
\begin{align*}
	\begin{cases}
		3x_1 - 2x_2 \geq -8\\
		3x_1 + x_2 \geq 3\\
		x_2 \leq 8\\
		x_1 \leq 4\\
	\end{cases}
\end{align*}
$x_1 \geq 0, x_2 \geq 0$

\section*{Задание}
\subsection*{Стандратная форма}
Целевая функция:\\
$f = 4x_1+x_2 \longrightarrow max$\\\\
Ограничения:
\begin{align*}
	\begin{cases}
		-3x_1 + 2x_2 \leq 8\\
		-3x_1 - x_2 \leq -3\\
		x_2 \leq 8\\
		x_1 \leq 4\\
	\end{cases}
\end{align*}
$x_1 \geq 0, x_2 \geq 0$
\newpage

\subsection*{Каноническая форма}
Целевая функция:\\
$f = 4x_1+x_2 \longrightarrow max$\\\\
Ограничения:
\begin{align*}
	\begin{cases}
		3x_1 - 2x_2 \geq -8\\
		3x_1 + x_2 \geq 3\\
		x_2 \leq 8\\
		x_1 \leq 4\\
	\end{cases}
\end{align*}
$x_1 \geq 0, x_2 \geq 0$\\

1. Вводим слабые переменные $y_1 \geq 0, y_2 \geq 0, y_3 \geq 0, y_4 \geq 0$:\\
$3x_1 - 2x_2 - y_1 = -8$\\
$3x_1 + x_2 - y_2 = 3$\\
$x_2 + y_3 = 8$\\
$x_1 + y_4 = 4$\\

2. Делаем правые части равенств положительными:\\
$-3x_1 + 2x_2 + y_1 = 8$\\
$3x_1 + x_2 - y_2 = 3$\\
$x_2 + y_3 = 8$\\
$x_1 + y_4 = 4$\\

Таким образом, задача сведена к канонической форме.

\subsection*{Матричная форма}
$A \times X^T = B^T$, где:
\begin{align*}
	A = \begin{pmatrix}
		-3 & 2 & 1 & 0 & 0 & 0\\
		3 & 1 & 0 & -1 & 0 & 0\\
		0 & 1 & 0 & 0 & 1 & 0\\
		1 & 0 & 0 & 0 & 0 & 1\\
	\end{pmatrix}
	\hspace{0.5cm}
	X = \begin{pmatrix}
		x_1 & x_2 & y_1 & y_2 & y_3 & y_4
	\end{pmatrix}
	\hspace{0.5cm}
	B = \begin{pmatrix}
		8 & 3 & 8 & 4
	\end{pmatrix}
\end{align*}
\newpage
\subsection*{Метод штрафов}
Введём искусственную переменную --- $r \geq 0$.\\\\
Целевая функция:\\
$f = 4x_1+x_2 \longrightarrow max$\\\\
Ограничения:\\
$-3x_1 + 2x_2 + y_1 = 8$\\
$3x_1 + x_2 - y_2 + r = 3$\\
$x_2 + y_3 = 8$\\
$x_1 + y_4 = 4$\\\\
$x_i \geq 0 \hspace{0.2cm} \forall i = \overline{1,2}; \hspace{0.5cm} y_j \geq 0 \hspace{0.2cm} \forall j =  \overline{1,4}$\\

В качестве базисных переменных возьмём $y_1, r, y_3, y_4$, свободные переменный --- $x_1, x_2, y_2$.\\\\
Выразим базисные переменные через свободные:\\
$y_1 = 8 + 3x_1 - 2x_2 $\\
$r = 3 - 3x_1 - x_2 + y_2$\\
$y_3 = 8 - x_2$\\
$y_4 = 4 - x_1$\\\\
Перепишем функцию цели:\\
$f = 4x_1 + x_2 - Mr = 4x_1 + x_2 - M(3 - 3x_1 - x_2 + y_2)$
\begin{center}$\downarrow$\end{center}
$f = -3M + (3M + 4)x_1 + (M + 1)x_2 - My_2$\\\\
Пусть $M = 100$, тогда функция цели примет следующий вид:\\
$f = -300 + 304x_1 + 101x_2 - 100y_2$\\

Перепишем функцию цели:\\
$f - 304x_1 - 101x_2 + 100y_2 = -300$
\newpage

\subsubsection*{1 итерация}
Базисные переменные: $y_1, r, y_3, y_4$.\\
Свободные переменный: $x_1, x_2, y_2$.\\\\
\begin{tabularx}{\textwidth}{|p{1cm}V{6}p{1.5cm}V{6}p{1.6cm}|p{0.5cm}|p{1.5cm}|p{0.5cm}|p{0.5cm}|p{1.2cm}|p{1.5cm}|c}
	\hline
	БП & $\boldsymbol{x_1}$ & $x_2$ & $y_1$ & $y_2$ & $y_3$ & $y_4$ & $r$ & Своб. член & \\
	\hline
	
	f & \small \underline{\textit{-304}} \scriptsize -304 & \small -101 \tiny -304/3 & 0 \scriptsize 0 & \small 100  \scriptsize 304/3 & 0 \scriptsize 0 & 0 \scriptsize 0 & 0 \tiny -304/3 & \small -300 \tiny -304 & \\
	
	\hline
	
	$y_1$ & \underline{-3} \scriptsize -3 & 2 \scriptsize -1 & 1 \scriptsize 0 & 0 \scriptsize 1 & 0 \scriptsize 0 & 0 \scriptsize 0 & 0 \scriptsize -1 & 8 \scriptsize -3 & $-\dfrac{8}{3} < 0$ \\
	
	\Xhline{6\arrayrulewidth}
	
	$\boldsymbol{r}$ & \textbf{3} \scriptsize 1 & 1 \underline{\scriptsize 1/3} & 0 \underline{\scriptsize 0} & -1 \underline{\tiny -1/3} & 0 \underline{\scriptsize 0} & 0  \underline{\scriptsize 0} & 1 \underline{\scriptsize 1/3} & 3 \underline{\scriptsize 1} & 1 -- \textit{min} \\
	
	\Xhline{6\arrayrulewidth}
	$y_3$ & \underline{0} \scriptsize 0 & 1 \scriptsize 0 & 0 \scriptsize 0 & 0 \scriptsize 0 & 1 \scriptsize 0 & 0 \scriptsize 0 & 0 \scriptsize 0 & 8 \scriptsize 0 & $\dfrac{8}{0}$ -- /0 ! \\
	\hline
	$y_4$ & \underline{1} \scriptsize 1 & 0 \scriptsize 1/3 & 0 \scriptsize 0 & 0 \scriptsize -1/3 & 0 \scriptsize 0 & 1 \scriptsize 0 & 0 \scriptsize 1/3 & 4 \scriptsize 1 & 4 \\
	\hline
\end{tabularx}
\newline\newline
Меняем свободную переменную $x_1$ и базисную переменную $r$ местами.\\
$x_1 \leftrightarrow r$

\subsubsection*{2 итерация}
Базисные переменные: $y_1, x_1, y_3, y_4$.\\
Свободные переменный: $x_2, y_2, r$.\\\\
\begin{tabularx}{\textwidth}{|p{1cm}|p{1cm}|p{1cm}|p{1cm}V{6}p{1cm}V{6}p{1cm}|p{1cm}|p{1.2cm}|p{1.2cm}|c}
	\hline
	БП & $x_1$ & $x_2$ & $y_1$ & $\boldsymbol{y_2}$ & $y_3$ & $y_4$ & $r$ & Своб. член & \\
	\hline
	
	f & 0 \scriptsize 0 & \small $\frac{1}{3}$ \tiny 4/3 & 0 \scriptsize 0 & \underline{\small -$\frac{4}{3}$} \tiny -4/3 & 0 \scriptsize 0 & 0 \scriptsize -4 & \small $\frac{304}{3}$ \tiny 4/3 & 4 \scriptsize -12 & \\
	
	\hline
	
	$y_1$ & 0 \scriptsize 0 & 3 \scriptsize 1 & 1 \scriptsize 0 & \underline{-1} \scriptsize -1 & 0 \scriptsize 0 & 0 \scriptsize -3 & 1 \scriptsize 1 & 11 \scriptsize -9 & $-\dfrac{11}{1} < 0$ \\
	
	\hline
	
	$x_1$ & 1 \scriptsize 0 & \small $\frac{1}{3}$ \scriptsize 1/3 & 0 \scriptsize 0 & \underline{\small -$\frac{1}{3}$} \tiny -1/3 & 0 \scriptsize 0 & 0 \scriptsize -1 & \small $\frac{1}{3}$ \scriptsize 1/3 & 1 \scriptsize -3 & -3 < 0 \\
	
	\hline
	$y_3$ & 0 \scriptsize 0 & 1 \scriptsize 0 & 0 \scriptsize 0 & \underline 0 \scriptsize 0 & 1 \scriptsize 0 & 0 \scriptsize 0 & 0 \scriptsize 0 & 8 \scriptsize 0 & $\dfrac{8}{0}$ -- /0 ! \\
	
	\Xhline{7\arrayrulewidth}
	$\boldsymbol{y_4}$ & 0 \underline{\scriptsize 0} & \small -$\frac{1}{3}$ \underline{\scriptsize -1} & 0 \underline{\scriptsize 0} & \small $\boldsymbol{\frac{1}{3}}$ \scriptsize 1 & 0 \underline{\scriptsize 0} & 1 \scriptsize \underline 3 & \small -$\frac{1}{3}$ \underline{\scriptsize -1} & 3 \underline{\scriptsize 9} & 9 -- \textit{min} \\
	\Xhline{6\arrayrulewidth}
\end{tabularx}
\newline\newline
Меняем свободную переменну $y_2$ и базисную переменную $y_4$ местами.\\
$y_2 \leftrightarrow y_4$
\newpage

\subsubsection*{3 итерация}
Базисные переменные: $y_1, x_1, y_2, y_3$.\\
Свободные переменный: $x_2, y_4, r$.\\\\
\begin{tabularx}{\textwidth}{|p{1cm}|p{1cm}V{7}p{1cm}V{6}p{1cm}|p{1cm}|p{1cm}|p{1cm}|p{1cm}|p{1.2cm}|c}
	\hline
	БП & $x_1$ & $\boldsymbol{x_2}$ & $y_1$ & $y_2$ & $y_3$ & $y_4$ & $r$ & Своб. член & \\
	\hline
	
	f & 0 \scriptsize 0 & \underline{\textit{-1}} \scriptsize -1 & 0 \scriptsize 0 & 0 \scriptsize 0 & 0 \scriptsize -1 & 4 \scriptsize 0 & 100 \scriptsize 0 & 16 \scriptsize -8 & \\
	
	\hline
	
	$y_1$ & 0 \scriptsize 0 & \underline{2} \scriptsize 2 & 1 \scriptsize 0 & 0 \scriptsize 0 & 0 \scriptsize 2 & 3 \scriptsize 0 & 0 \scriptsize 0 & 20 \scriptsize 16 & 10 \\
	
	\hline
	
	$x_1$ & 1 \scriptsize 0 & \underline{0} \scriptsize 0 & 0 \scriptsize 0 & 0 \scriptsize 0 & 0 \scriptsize 0 & 1 \scriptsize 0 & 0 \scriptsize 0 & 4 \scriptsize 0 & $\dfrac{4}{0}$ -- /0 ! \\
	
	\Xhline{6\arrayrulewidth}
	$\boldsymbol{y_3}$ & 0 \scriptsize \underline 0 & \textbf{1} \scriptsize 1 & 0 \scriptsize \underline 0 & 0 \scriptsize \underline 0 & 1 \scriptsize \underline 1 & 0 \scriptsize \underline 0 & 0 \scriptsize \underline 0 & 8 \scriptsize \underline 8 & 8 -- \textit{min} \\
	\Xhline{6\arrayrulewidth}
	
	$y_2$ & 0 \scriptsize 0 & \underline{-1} \scriptsize -1 & 0 \scriptsize 0 & 1 \scriptsize 0 & 0 \scriptsize -1 & 3 \scriptsize 0 & -1 \scriptsize 0 & 9 \scriptsize -8 & -9 < 0 \\
	\hline
\end{tabularx}
\newline\newline
Меняем свободную переменну $x_2$ и базисную переменную $y_3$ местами.\\
$x_2 \leftrightarrow y_3$

\subsubsection*{Оптимальное решение}
Базисные переменные: $y_1, x_1, x_2, y_2$.\\
Свободные переменный: $y_3, y_4, r$.\\\\
\begin{tabularx}{\textwidth}{|p{1cm}|p{1cm}|p{1cm}|p{1cm}|p{1cm}|p{1cm}|p{1cm}|p{1cm}|p{1.2cm}|c}
	\hline
	БП & $x_1$ & $x_2$ & $y_1$ & $y_2$ & $y_3$ & $y_4$ & $r$ & Своб. член & \\
	\hline
	f & 0 & 0 & 0 & 0 & 1 & 4 & 100 & 24 & \\
	\hline
	$y_1$ & 0 & 0 & 1 & 0 & -2 & 3 & 0 & 4 & \\
	\hline
	$x_1$ & 1 & 0 & 0 & 0 & 0 & 1 & 0 & 4\\
	\hline
	$x_2$ & 0 & 1 & 0 & 0 & 1 & 0 & 0 & 8 \\
	\hline
	$y_2$ & 0 & 0 & 0 & 1 & 1 & 3 & -1 & 17\\
	\hline
\end{tabularx}
\newline\newline
Таким образом, получается:\\
$f + y_3 + 4y_4 + 100r = 24 \hspace*{0.5cm} \rightarrow \hspace*{0.5cm} f = 24$\\
$y_1 - 2y_3 + 3y_4 = 4 \hspace*{1.9cm} \rightarrow \hspace*{0.5cm} y_1 = 4$\\
$x_1 + y_4 = 4 \hspace*{3.4cm} \rightarrow \hspace*{0.5cm} x_1 = 4$\\
$x_2 + y_3 = 8 \hspace*{3.4cm} \rightarrow \hspace*{0.5cm} x_2 = 8$\\
$y_2 + y_3 + 3y_4 - r = 17 \hspace*{1.05cm} \rightarrow \hspace*{0.5cm} y_2 = 17$\\
$y_3 = 0, y_4 = 0, r = 0$\\
\textbf{Ответ:} $x_1 = 4, x_2 = 8, f = 24$
\subsection*{Двухэтапный метод}
Целевая функция:\\
$f = 4x_1+x_2 \longrightarrow max$\\\\
Ограничения:\\
$-3x_1 + 2x_2 + y_1 = 8$\\
$3x_1 + x_2 - y_2 + r = 3 \hspace*{0.5cm} \rightarrow \hspace*{0.5cm} r = 3 - 3x_1 - x_2 + y_2$\\
$x_2 + y_3 = 8$\\
$x_1 + y_4 = 4$\\\\
$x_i \geq 0 \hspace{0.2cm} \forall i = \overline{1,2}; \hspace{0.5cm} y_j \geq 0 \hspace{0.2cm} \forall j =  \overline{1,4}$\\\\
Введём вспомогательную функцию цели:\\
$\phi = r = 3 - 3x_1 - x_2 + y_2 \longrightarrow min$\\\\
Перепишем функцию $\phi$:\\
$\phi + 3x_1 + x_2 - y_2 = 3$
\subsubsection*{1 итерация}
Базисные переменные: $y_1, r, y_3, y_4$.\\
Свободные переменный: $x_1, x_2, y_2$.\\\\
\begin{tabularx}{\textwidth}{|p{1cm}V{6}p{1cm}V{7}p{1cm}|p{1cm}|p{1cm}|p{1cm}|p{1cm}|p{1cm}|p{1cm}|c}
	\hline
	БП & $\boldsymbol{x_1}$ & $x_2$ & $y_1$ & $y_2$ & $y_3$ & $y_4$ & $r$ & Своб. член & \\
	\hline
	
	$\phi$ & \underline{\textit{3}} \scriptsize 3 & 1 \scriptsize 1 & 0 \scriptsize 0 & -1  \scriptsize -1 & 0 \scriptsize 0 & 0 \scriptsize 0 & 0 \scriptsize 1 & 3 \scriptsize 3 & \\
	
	\hline
	
	$y_1$ & \underline{-3} \scriptsize -3 & 2 \scriptsize -1 & 1 \scriptsize 0 & 0 \scriptsize 1 & 0 \scriptsize 0 & 0 \scriptsize 0 & 0 \scriptsize -1 & 8 \scriptsize -3 & $-\dfrac{8}{3} < 0$ \\
	
	\Xhline{7\arrayrulewidth}
	
	$\boldsymbol{r}$ & \textbf{3} \scriptsize 1 & 1 \underline{\scriptsize 1/3} & 0 \underline{\scriptsize 0} & -1 \underline{\tiny -1/3} & 0 \underline{\scriptsize 0} & 0  \underline{\scriptsize 0} & 1 \underline{\scriptsize 1/3} & 3 \underline{\scriptsize 1} & 1 -- \textit{min} \\
	
	\Xhline{7\arrayrulewidth}
	$y_3$ & \underline{0} \scriptsize 0 & 1 \scriptsize 0 & 0 \scriptsize 0 & 0 \scriptsize 0 & 1 \scriptsize 0 & 0 \scriptsize 0 & 0 \scriptsize 0 & 8 \scriptsize 0 & $\dfrac{8}{0}$ -- /0 ! \\
	\hline
	$y_4$ & \underline{1} \scriptsize 1 & 0 \scriptsize 1/3 & 0 \scriptsize 0 & 0 \scriptsize -1/3 & 0 \scriptsize 0 & 1 \scriptsize 0 & 0 \scriptsize 1/3 & 4 \scriptsize 1 & 4 \\
	\hline
\end{tabularx}
\newpage
Меняем свободную переменную $x_1$ и базисную переменную $r$ местами.\\
$x_1 \leftrightarrow r$
\subsubsection*{Оптимальное решение}
Базисные переменные: $y_1, x_1, y_3, y_4$.\\
Свободные переменный: $x_2, y_2, r$.\\\\
\begin{tabularx}{\textwidth}{|p{1cm}|p{1cm}|p{1cm}|p{1cm}|p{1cm}|p{1cm}|p{1cm}|p{1.2cm}|p{1.2cm}|c}
	\hline
	БП & $x_1$ & $x_2$ & $y_1$ & $y_2$ & $y_3$ & $y_4$ & $r$ & Своб. член & \\
	\hline
	
	$\phi$ & 0 & 0 & 0 & 0 & 0 & 0 & -1 & 0 & \\
	
	\hline
	
	$y_1$ & 0 & 3 & 1 & -1 & 0 & 0 & 1 & 11 & \\
	
	\hline
	
	$x_1$ & 1 & \small $\frac{1}{3}$ & 0 & \small -$\frac{1}{3}$ & 0 & 0 & \small $\frac{1}{3}$ & 1 & \\
	
	\hline
	$y_3$ & 0 & 1 & 0 & 0 & 1 & 0 & 0 & 8 & \\
	
	\hline
	$y_4$ & 0 & \small -$\frac{1}{3}$ & 0 & \small $\frac{1}{3}$ & 0 & 1 & \small -$\frac{1}{3}$ & 3 & \\
	\hline
\end{tabularx}
\newline\newline
$r = 0$ и значение вспомогательной функции $\phi = 0$.\\
Таким образом, получается:\\
$3x_2 + y_1 - y_2 = 11$\\
$x_1 + \frac{1}{3}x_2 - \frac{1}{3}y_2 = 1 \hspace*{0.5cm} \rightarrow \hspace*{0.5cm} x_1 = - \frac{1}{3}x_2 + \frac{1}{3}y_2 + 1 $\\
$x_2 + y_3 = 8$\\
$-\frac{1}{3}x_2 + \frac{1}{3}y_2 + y_4 = 3$\\\\
Выразим функцию цели через свободные переменные:\\
$f = 4x_1 + x_2 = 4-\frac{4}{3}x_2 + \frac{4}{3}y_2 + x_2 = 4 - \frac{1}{3}x_2 + \frac{4}{3}y_2$
\begin{center}$\downarrow$\end{center}
$f = 4 - \frac{1}{3}x_2 + \frac{4}{3}y_2 \longrightarrow max$\\\\
Перепишем функцию цели:\\
$f + \frac{1}{3}x_2 - \frac{4}{3}y_2 = 4$
\newpage
\subsubsection*{1 итерация}
Базисные переменные: $y_1, x_1, y_3, y_4$.\\
Свободные переменный: $x_2, y_2$.\\\\
\begin{tabularx}{\textwidth}{|p{1cm}|p{1cm}|p{1cm}|p{1cm}V{6}p{1cm}V{7}p{1cm}|p{1cm}|p{1.2cm}|c}
	\hline
	БП & $x_1$ & $x_2$ & $y_1$ & $\boldsymbol{y_2}$ & $y_3$ & $y_4$ & Своб. член & \\
	\hline
	
	f & 0 \scriptsize 0 & \small $\frac{1}{3}$ \tiny 4/3 & 0 \scriptsize 0 & \underline{\small -$\frac{4}{3}$} \tiny -4/3 & 0 \scriptsize 0 & 0 \scriptsize -4 & 4 \scriptsize -12 & \\
	
	\hline
	
	$y_1$ & 0 \scriptsize 0 & 3 \scriptsize 1 & 1 \scriptsize 0 & \underline{-1} \scriptsize -1 & 0 \scriptsize 0 & 0 \scriptsize -3 & 11 \scriptsize -9 & $-\dfrac{11}{1} < 0$ \\
	
	\hline
	
	$x_1$ & 1 \scriptsize 0 & \small $\frac{1}{3}$ \scriptsize 1/3 & 0 \scriptsize 0 & \underline{\small -$\frac{1}{3}$} \tiny -1/3 & 0 \scriptsize 0 & 0 \scriptsize -1 & 1 \scriptsize -3 & -3 < 0 \\
	
	\hline
	$y_3$ & 0 \scriptsize 0 & 1 \scriptsize 0 & 0 \scriptsize 0 & \underline 0 \scriptsize 0 & 1 \scriptsize 0 & 0 \scriptsize 0 & 8 \scriptsize 0 & $\dfrac{8}{0}$ -- /0 ! \\
	
	\Xhline{6\arrayrulewidth}
	$\boldsymbol{y_4}$ & 0 \underline{\scriptsize 0} & \small -$\frac{1}{3}$ \underline{\scriptsize -1} & 0 \underline{\scriptsize 0} & \small $\boldsymbol{\frac{1}{3}}$ \scriptsize 1 & 0 \underline{\scriptsize 0} & 1 \scriptsize \underline 3 & 3 \underline{\scriptsize 9} & 9 -- \textit{min} \\
	\Xhline{7\arrayrulewidth}
\end{tabularx}
\newline\newline
Меняем свободную переменну $y_2$ и базисную переменную $y_4$ местами.\\
$y_2 \leftrightarrow y_4$
\subsubsection*{2 итерация}
Базисные переменные: $y_1, x_1, y_2, y_3$.\\
Свободные переменный: $x_2, y_4$.\\\\
\begin{tabularx}{\textwidth}{|p{1cm}|p{1cm}V{7}p{1cm}V{6}p{1cm}|p{1cm}|p{1cm}|p{1cm}|p{1.2cm}|c}
	\hline
	БП & $x_1$ & $\boldsymbol{x_2}$ & $y_1$ & $y_2$ & $y_3$ & $y_4$ & Своб. член & \\
	\hline
	
	f & 0 \scriptsize 0 & \underline{\textit{-1}} \scriptsize -1 & 0 \scriptsize 0 & 0 \scriptsize 0 & 0 \scriptsize -1 & 4 \scriptsize 0 & 16 \scriptsize -8 & \\
	
	\hline
	
	$y_1$ & 0 \scriptsize 0 & \underline{2} \scriptsize 2 & 1 \scriptsize 0 & 0 \scriptsize 0 & 0 \scriptsize 2 & 3 \scriptsize 0 & 20 \scriptsize 16 & 10 \\
	
	\hline
	
	$x_1$ & 1 \scriptsize 0 & \underline{0} \scriptsize 0 & 0 \scriptsize 0 & 0 \scriptsize 0 & 0 \scriptsize 0 & 1 \scriptsize 0 & 4 \scriptsize 0 & $\dfrac{4}{0}$ -- /0 ! \\
	
	\Xhline{7\arrayrulewidth}
	$\boldsymbol{y_3}$ & 0 \scriptsize \underline 0 & \textbf{1} \scriptsize 1 & 0 \scriptsize \underline 0 & 0 \scriptsize \underline 0 & 1 \scriptsize \underline 1 & 0 \scriptsize \underline 0 & 8 \scriptsize \underline 8 & 8 -- \textit{min} \\
	\Xhline{6\arrayrulewidth}
	
	$y_2$ & 0 \scriptsize 0 & \underline{-1} \scriptsize -1 & 0 \scriptsize 0 & 1 \scriptsize 0 & 0 \scriptsize -1 & 3 \scriptsize 0 & 9 \scriptsize -8 & -9 < 0 \\
	\hline
\end{tabularx}
\newline\newline
Меняем свободную переменну $x_2$ и базисную переменную $y_3$ местами.\\
$x_2 \leftrightarrow y_3$
\newpage
\subsubsection*{Оптимальное решение}
Базисные переменные: $y_1, x_1, x_2, y_2$.\\
Свободные переменный: $y_3, y_4$.\\\\
\begin{tabularx}{\textwidth}{|p{1cm}|p{1cm}|p{1cm}|p{1cm}|p{1cm}|p{1cm}|p{1cm}|p{1.2cm}|c}
	\hline
	БП & $x_1$ & $x_2$ & $y_1$ & $y_2$ & $y_3$ & $y_4$ & Своб. член & \\
	\hline
	f & 0 & 0 & 0 & 0 & 1 & 4 & 24 & \\
	\hline
	$y_1$ & 0 & 0 & 1 & 0 & -2 & 3 & 4 & \\
	\hline
	$x_1$ & 1 & 0 & 0 & 0 & 0 & 1 & 4\\
	\hline
	$x_2$ & 0 & 1 & 0 & 0 & 1 & 0 & 8 \\
	\hline
	$y_2$ & 0 & 0 & 0 & 1 & 1 & 3 & 17\\
	\hline
\end{tabularx}
\newline\newline
Таким образом, получается:\\
$f + y_3 + 4y_4 = 24 \hspace*{1.05cm} \rightarrow \hspace*{0.5cm} f = 24$\\
$y_1 - 2y_3 + 3y_4 = 4 \hspace*{0.9cm} \rightarrow \hspace*{0.5cm} y_1 = 4$\\
$x_1 + y_4 = 4 \hspace*{2.42cm} \rightarrow \hspace*{0.5cm} x_1 = 4$\\
$x_2 + y_3 = 8 \hspace*{2.42cm} \rightarrow \hspace*{0.5cm} x_2 = 8$\\
$y_2 + y_3 + 3y_4 = 17 \hspace*{0.9cm} \rightarrow \hspace*{0.5cm} y_2 = 17$\\
$y_3 = 0, y_4 = 0$\\
\textbf{Ответ:} $x_1 = 4, x_2 = 8, f = 24$
\end{document}