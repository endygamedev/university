\documentclass[14pt,fleqn]{extarticle}
\usepackage[T2A,T1]{fontenc}
\usepackage[utf8]{inputenc}
\usepackage[russian]{babel}
\usepackage{amsmath}
\usepackage{graphicx}
\usepackage{tabularx}
\usepackage{boldline}
\usepackage{makecell}
\usepackage{arydshln}
\usepackage{mathtools}
\usepackage{centernot}
\usepackage{enumitem}
\usepackage{nccmath}
\usepackage{amssymb}
\usepackage{amsmath}
\usepackage{import}
\usepackage[a4paper, total={6.5in, 9.5in}]{geometry}
\usepackage{hyperref}

\graphicspath{ {./content/images} }
\setlength{\mathindent}{0pt}
\setlength\parindent{0pt}

\def\at{
	\left.
	\vphantom{\int}
	\right|
}


\begin{document}
	\begin{titlepage}
		\includegraphics[scale=0.1]{logo}
		\begin{center}
			\textbf{МИНОБРНАУКИ РОССИИ}\\
			\vspace{0.2cm}
			\textbf{Федеральное государственное бюджетное образовательное учреждение высшего образования}\\
			\textbf{<<САНКТ-ПЕТЕРБУРГСКИЙ ГОСУДАРСТВЕННЫЙ ЭКОНОМИЧЕСКИЙ УНИВЕРСИТЕТ>>}\\
			\vspace{0.6cm}
			Факультет информатики и прикладной математики\\
			Кафедра прикладной математики и экономико-математических методов\\
			\vspace{1cm}
			\textbf{ОТЧЁТ}\\
			по дисциплине:\\
			\textbf{<<Модели экономической динамики>>}\\
		\end{center}
		\vspace{1cm}
		Студента: Бронникова Егора Игоревича\\
		Курс: 4\\
		Группа: ПМ-1901\\
		Форма обучения: очная\\
		\vspace{0.5cm}
		\newline
		Форма представления на кафедру выполненных заданий:\\
		отчёт в электронной форме\\
		\vspace{0.1cm}
		\newline
		\begin{center} % just for vertical spacing and killing indent
			\begin{tabular*}{\textwidth}{@{}l@{\extracolsep{\fill}}r@{}}
				Оценка по результатам текущего & \\
				контроля (КТ№3) & \\
				\noindent\rule{5cm}{0.4pt}  & \noindent\rule{5cm}{0.4pt} \\
			\end{tabular*}
		\end{center}
		\vfill
		\begin{center}
			Санкт-Петербург\\
			2022\\
		\end{center}
	\end{titlepage}
	
	\tableofcontents
	\newpage
	\import{content/}{task1.tex}
	\newpage
	\import{content/}{task2.tex}
	\newpage
	\import{content/}{task3.tex}
	\newpage
	\import{content/}{task4.tex}
	\newpage
	\import{content/}{task5.tex}
	\newpage
	\import{content/}{task6.tex}
	\newpage
	\import{content/}{task7.tex}
	\newpage
	\import{content/}{task8.tex}
	\newpage
	\import{content/}{task9.tex}
	\newpage
	\import{content/}{task10.tex}
	\newpage
	\import{content/}{task11.tex}
	\newpage
	\import{content/}{task12.tex}
	\newpage
	\import{content/}{task13.tex}
	\newpage
	\import{content/}{task14.tex}
	\newpage
	\import{content/}{task15.tex}
	\newpage
	\import{content/}{task16.tex}
	\newpage
	\import{content/}{conclusion.tex}
\end{document}