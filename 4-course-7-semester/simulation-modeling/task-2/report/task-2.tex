\documentclass[14pt,fleqn]{extarticle}
\usepackage[T2A,T1]{fontenc}
\usepackage[utf8]{inputenc}
\usepackage[russian]{babel}
\usepackage{amsmath}
\usepackage{graphicx}
\usepackage{tabularx}
\usepackage{boldline}
\usepackage{makecell}
\usepackage{arydshln}
\usepackage{mathtools}
\usepackage{centernot}
\usepackage{enumitem}
\usepackage{nccmath}
\usepackage[a4paper, total={6.5in, 9.5in}]{geometry}

\graphicspath{ {./images/} }
\setlength{\mathindent}{0pt}
\setlength\parindent{0pt}

\def\at{
	\left.
	\vphantom{\int}
	\right|
}


\begin{document}
	\begin{titlepage}
		\includegraphics[scale=0.12]{logo}
		\begin{center}
			\textbf{МИНОБРНАУКИ РОССИИ}\\
			\vspace{0.2cm}
			\textbf{Федеральное государственное бюджетное образовательное учреждение высшего образования}\\
			\textbf{<<САНКТ-ПЕТЕРБУРГСКИЙ ГОСУДАРСТВЕННЫЙ ЭКОНОМИЧЕСКИЙ УНИВЕРСИТЕТ>>}\\
			\vspace{0.6cm}
			Факультет информатики и прикладной математики\\
			Кафедра прикладной математики и экономико-математических методов\\
			\vspace{1cm}
			\textbf{ОТЧЁТ}\\
			по дисциплине:\\
			\textbf{<<Имитационное моделирование>>}\\
			на тему:\\
			\textbf{<<Моделирование случайных величин. Статистические испытания. Задание №2>>}\\
		\end{center}
		\vspace{1cm}
		Направление: 01.03.02\\
		Обучающийся: Бронников Егор Игоревич\\
		Группа: ПМ-1901\\
		\vfill
		\begin{center}
			Санкт-Петербург\\
			2022\\
		\end{center}
	\end{titlepage}
    \subsection*{Задание №1}
    Реализовать генератор случайных чисел, используя метод серединных квадратов (фон Нейман). Проанализировать свойства полученной последовательности.\\
    \newline

    В методе серединных квадратов изначально задаётся количество разрядов числа ($k$) и начальное значение $R_0$. Далее число $R_0$ возводится в квадрат и из середины квадрата числа берётся $k$-значное число, которое снова возводится в квадрат, и так далее.\\
    Обязательным условием является то, что количество разрядов ($k$) должно быть чётным числом.\\
    \newline
    Данный алгоритм был реализован на языке программирования Python. (Рисунок \ref{fig:mid_square_method_code})
    \begin{figure}[h]
        \centering \includegraphics[scale=0.8]{code1}
        \caption{Реализация метода серединных квадратов на языке программирования Python}
        \label{fig:mid_square_method_code}
    \end{figure}
    \newpage
    Алгоритм был запущен с параметрами $k = 10$, $R_0 = 25$, $n = 7$, где $n$ -- это количество сгенерированных случайных чисел. Можно проследить, что при достаточно малых разрядах и малом начальном значении быстро получается вырождение, что плохо. (Рисунок \ref{fig:mid_square_method_result})
    \begin{figure}[h]
        \centering \includegraphics[scale=0.8]{result1}
        \caption{Результаты генерации случайных чисел методом серединных квадратов}
        \label{fig:mid_square_method_result}
    \end{figure}
    
    \subsection*{Задание №2}
    Реализовать линейный конгруэнтный датчик случайных чисел. Сгенерировать последовательность вещественных чисел, распределённых равномерно: 1) на интервале [0,1); \; 2) на интервале [a,b). Проанализировать полученные последовательности. Определить период, построить гистограмму.\\
    \newline

    Линейный конгруэнтный метод -- это один из рекуррентных методов генерации случайных чисел. Следующий элемент последовательности может быть найден по следующей формуле:
    \begin{center}
        $ r_{i+1} = (k \cdot r_i + b) \; mod \; M$
    \end{center}
    
    \newpage
    Линейная конгруэнтная последовательность, определённая числами $M$, $k$, $b$, $r_0$ периодична с периодом, не превышающим $M$. При этом длина периода равна $M$ тогда и только тогда, когда:
    \begin{enumerate}[topsep=0pt,itemsep=-1ex,partopsep=1ex,parsep=1ex]
        \item числа $b$ и $M$ взаимно простые;
        \item $k-1$ кратно $p$ для каждого простого $p$, являющегося делителем $M$;
        \item $k-1$ кратно 4, если $M$ кратно 4.
    \end{enumerate}
    Сначала был реализован алгоритм для интервала [0;1) на языке программирования Python. (Рисунок \ref{fig:linear_congruent_gauge_0_1_code})
    \begin{figure}[h]
        \centering \includegraphics[scale=0.8]{code21}
        \caption{Реализация линейного конгруэнтного счётчика на интервале [0,1)}
        \label{fig:linear_congruent_gauge_0_1_code}
    \end{figure}

    Для того чтобы получить случайные числа в интервале от [0,1) нужно поделить каждый случайный сгенерированный элемент последовательности на $M$. Также данная функция выводит период сгенерированной последовательности.
    \begin{figure}[h]
        \centering \includegraphics[scale=0.52]{result21}
        \caption{Результаты генерации случайных чисел линейным конгруэнтным счётчиком на интервале [0,1)}
        \label{fig:linear_congruent_gauge_0_1_result}
    \end{figure}
    \newpage
    В качестве аргументов были выбраны следующие значения: $r_0 = 3$, $k = 2$, $b = 1$, $M = 10$, $n = 12$, где $n$ -- это количество сгенерированных случайных чисел. Можно видеть, что при данном наборе аргументов длина периода составила 4. (Рисунок \ref{fig:linear_congruent_gauge_0_1_result})\\
    \newline
    Далее был реализован алгоритм для интервала [a;b). (Рисунок \ref{fig:linear_congruent_gauge_a_b_code})
    \begin{figure}[h]
        \centering \includegraphics[scale=0.6]{code22}
        \caption{Реализация линейного конгруэнтного счётчика на интервале [a,b)}
        \label{fig:linear_congruent_gauge_a_b_code}
    \end{figure}

	Для того чтобы получить случайные числа в интервале от [a,b) нужно проделать следующее преобразование:
	\begin{center}
		$(1 - \dfrac{r_i}{M})/a + \dfrac{r_i}{M} \cdot b$
	\end{center}
	То есть сначала мы генерируем числа в интервале от [0,1), а дальше преобразуем их к интервалу от [a,b).
 	\begin{figure}[h]
		\centering \includegraphics[scale=0.52]{result22}
		\caption{Результаты генерации случайных чисел линейным конгруэнтным счётчиком на интервале [a,b)}
		\label{fig:linear_congruent_gauge_a_b_result}
	\end{figure}
    \newpage
	В качестве аргументов были выбраны следующие значения: $r_0 = 3$, $k = 2$, $b = 1$, $M = 10$, $n = 10$, $a_{param} = 100$, $b_{param} = 200$, где $n$ -- это количество сгенерированных случайных чисел. Можно видеть, что при данном наборе аргументов длина периода составила 4. (Рисунок \ref{fig:linear_congruent_gauge_a_b_result})\\
	
	\subsection*{Задание №3}
	Используя метод обратной функции, получить последовательность случайных чисел, распределённых экспоненциально с заданным параметром $\lambda$. Проанализировать полученную последовательность. Оценить математическое ожидание  и дисперсию, построить гистограмму.\\
	\newline
	
	Плотность распределения экспоненциального закона:
	\begin{ceqn}
	\begin{align*}
		f(x) =
		\begin{cases}
			\lambda e^{-\lambda \cdot x} \quad x \geq 0\\
			0 \hspace{1.55cm} x < 0
		\end{cases}
	\end{align*}
	\end{ceqn}

	Функция распределения экспоненциального закона:
	\begin{ceqn}
	\begin{align*}
		F(x) =
		\begin{cases}
			1 - e^{-\lambda \cdot x} \quad x \geq 0\\
			0 \hspace{2.15cm} x < 0
		\end{cases}
	\end{align*}
	\end{ceqn}

	Получается, что обратная функция $F^{-1}(x)$ будет выглядеть следующим образом:
	\begin{center}
		$x = \dfrac{-ln(1-y)}{\lambda} = -\dfrac{ln(y)}{\lambda}$
	\end{center}
	Если подставлять вместо $y$ случайные равномерно распределённые значения, то можно получать требуемые числа.
	\newpage
	Таким образом, была реализована функция на языке программирования Python. (Рисунок \ref{fig:exp_inverse_function_method_code})
	\begin{figure}[h]
		\centering \includegraphics[scale=0.8]{code3}
		\caption{Реализация метода обратной функции для экспоненциального закона}
		\label{fig:exp_inverse_function_method_code}
	\end{figure}

	При $\lambda = 3$ и $n = 1000$ получается следующий результат. (Рисунок \ref{fig:exp_inverse_function_method_result})
	\begin{figure}[h]
		\centering \includegraphics[scale=0.8]{result3}
		\caption{Результаты генерации случайных чисел методом обратной функции для экспоненциального закона}
		\label{fig:exp_inverse_function_method_result}
	\end{figure}

	Математическое ожидание экспоненциального распределения:
	\begin{ceqn}
		\begin{align*}
			E = \dfrac{1}{\lambda}
		\end{align*}
	\end{ceqn}
	\newpage
	Если рассчитывать математическое ожидание как среднее значение в выборке, то получается следующий результат. (Рисунок \ref{fig:exp_inverse_function_method_math})
	\begin{figure}[h]
		\centering \includegraphics[scale=0.7]{result31}
		\caption{Теоретическое и расчётное значения математического ожидания ($\lambda = 3$)}
		\label{fig:exp_inverse_function_method_math}
	\end{figure}

	Можно заметить, что при $n = 1000$ значения получились достаточно близкими.\\
	Дисперсия экспоненциального распределения:
	\begin{ceqn}
		\begin{align*}
			D = \dfrac{1}{\lambda^2}
		\end{align*}
	\end{ceqn}
	Далее можно рассчитать дисперсию как среднее квадратное отклонение от среднего значения выборки. (Рисунок \ref{fig:exp_inverse_function_method_var})
	\begin{figure}[h]
		\centering \includegraphics[scale=0.7]{result32}
		\caption{Теоретическое и расчётное значения дисперсии ($\lambda = 3$)}
		\label{fig:exp_inverse_function_method_var}
	\end{figure}

	Можно заметить, что при $n = 1000$ значения получились достаточно близкими.
	\newpage
	\subsection*{Задание №4}
	Для треугольного распределения получить функцию распределения. Используя метод обратной функции, получить последовательность случайных чисел с треугольным распределением. Оценить математическое ожидание и дисперсию, построить гистограмму.\\
	\newline
	
	Пусть имеются следующие параметры: $a$ (\textit{min}), $b$ (\textit{max}), $c$ (\textit{мода}).\\
    Плотность треугольного распределения:
    \begin{ceqn}
	\begin{align*}
		f(x) =
		\begin{cases}
			\dfrac{2(x-a)}{(b-a)(c-a)} \quad , x \in [a,c]\\
			\dfrac{2(b-x)}{(b-c)(b-a)} \quad , x \in [c,b]\\
			0 \hspace{3.25cm} , x \centernot\in [a,b]\\
		\end{cases}
	\end{align*}
    \end{ceqn}

    Функция распределения треугольного закона:
    \begin{ceqn}
	\begin{align*}
		F(x) =
		\begin{cases}
			0 \hspace{4.1cm} , x < a\\
			\dfrac{(x-a)^2}{(b-a)(c-a)} \hspace{1.3cm} , x \in [a,c]\\
			1 - \dfrac{(x-b)^2}{(b-a)(b-c)} \quad , x \in [c,b]\\
			1 \hspace{4.1cm} , x > b\\
		\end{cases}
	\end{align*}
    \end{ceqn}

	Получается, что обратная функция $F^{-1}(x)$ будет выглядеть следующим образом:
	\begin{ceqn}
	\begin{align*}
		F^{-1}(y) = x =
		\begin{cases}
			a + \sqrt{(b-a)(c-a)y} \hspace{2.2cm} , 0 < y < \dfrac{c-a}{b-a}\\
			b - \sqrt{(b-a)(b-c)(1-y)} \hspace{1cm} , \dfrac{c-a}{b-a} \leq y < 1\\
		\end{cases}
	\end{align*}
    \end{ceqn}

	Если подставить вместо $y$ случайные равномерно распределённые значения, то можно получить требуемые числа.
	\newpage
	Таким образом, была реализована функция на языке программирования Python. (Рисунок \ref{fig:triangle_inverse_function_method_code})
	\begin{figure}[h]
		\centering \includegraphics[scale=0.7]{code4}
		\caption{Реализация метода обратной функции для треугольного закона}
		\label{fig:triangle_inverse_function_method_code}
	\end{figure}

	При $a = 0$, $b = 1$, $c = 0.5$ и $n = 10$ получается следующий результат. (Рисунок \ref{fig:triangle_inverse_function_method_resul1})
	\begin{figure}[h]
		\centering \includegraphics[scale=0.5]{result41}
		\caption{Результаты генерации случайных чисел методом обратной функции для треугольного закона}
		\label{fig:triangle_inverse_function_method_resul1}
	\end{figure}
	
	Математическое ожидание треугольного распределения:
	\begin{ceqn}
	\begin{align*}
		E = \dfrac{a + b + c}{3}
	\end{align*}
	\end{ceqn}
	\newpage
	Если рассчитывать математическое ожидание как среднее значение в выборке, то получается следующий результат. (Рисунок \ref{fig:triangle_inverse_function_method_math})
	\begin{figure}[h]
		\centering \includegraphics[scale=0.7]{result42}
		\caption{Теоретическое и расчётное значения математического ожидания ($a = 0$, $b = 1$, $c = 0.5$)}
		\label{fig:triangle_inverse_function_method_math}
	\end{figure}
	
	Можно заметить, что при $n = 1000$ значения получились достаточно близкими.\\
\end{document}
