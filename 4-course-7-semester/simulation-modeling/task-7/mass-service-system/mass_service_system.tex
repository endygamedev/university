\documentclass[14pt,fleqn]{extarticle}
\usepackage[T2A,T1]{fontenc}
\usepackage[utf8]{inputenc}
\usepackage[russian]{babel}
\usepackage{amsmath}
\usepackage{graphicx}
\usepackage{tabularx}
\usepackage{boldline}
\usepackage{makecell}
\usepackage{arydshln}
\usepackage{mathtools}
\usepackage{centernot}
\usepackage{enumitem}
\usepackage{nccmath}
\usepackage[a4paper, total={6.5in, 9.5in}]{geometry}

\setlength{\mathindent}{0pt}
\setlength\parindent{0pt}

\def\at{
	\left.
	\vphantom{\int}
	\right|
}


\begin{document}
	\textbf{Моделируется работа мойки автомобилей}\\
	\newline
	Пусть имеется автомойка, которая работает ежедневно. У автомойки есть несколько боксов (предположим 10). Заявка -- автомобиль, прибывший в момент времени. Если все боксы заняты, то автомобиль может подождать некоторое время, а может сразу выйти. Средняя продолжительность обслуживания -- 1.5 часа. Интенсивность потока автомобилей в час задаётся числом. После обслуживания автомобиль покидает систему.
\end{document}
