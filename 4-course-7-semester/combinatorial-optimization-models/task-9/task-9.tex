\documentclass[14pt,fleqn]{extarticle}
\usepackage[T2A,T1]{fontenc}
\usepackage[utf8]{inputenc}
\usepackage[russian]{babel}
\usepackage{amsmath}
\usepackage{graphicx}
\usepackage{tabularx}
\usepackage{boldline}
\usepackage{makecell}
\usepackage{arydshln}
\usepackage{mathtools}
\usepackage{centernot}
\usepackage{enumitem}
\usepackage{nccmath}
\usepackage{amssymb}
\usepackage[a4paper, total={6.5in, 9.5in}]{geometry}

\graphicspath{ {./images/} }
\setlength{\mathindent}{0pt}
\setlength\parindent{0pt}

\def\at{
	\left.
	\vphantom{\int}
	\right|
}


\begin{document}
	\begin{titlepage}
		\includegraphics[scale=0.12]{logo}
		\begin{center}
			\textbf{МИНОБРНАУКИ РОССИИ}\\
			\vspace{0.2cm}
			\textbf{Федеральное государственное бюджетное образовательное учреждение высшего образования}\\
			\textbf{<<САНКТ-ПЕТЕРБУРГСКИЙ ГОСУДАРСТВЕННЫЙ ЭКОНОМИЧЕСКИЙ УНИВЕРСИТЕТ>>}\\
			\vspace{0.6cm}
			Факультет информатики и прикладной математики\\
			Кафедра прикладной математики и экономико-математических методов\\
			\vspace{1cm}
			\textbf{ОТЧЁТ}\\
			по дисциплине:\\
			\textbf{<<Модели комбинаторной оптимизации>>}\\
			на тему:\\
			\textbf{<<Задание №9. Задача коммивояжера с временными окнами (mTSPTW)>>}\\
		\end{center}
		\vspace{1cm}
		Направление: 01.03.02\\
		Обучающийся: Бронников Егор Игоревич\\
		Группа: ПМ-1901\\
		\vfill
		\begin{center}
			Санкт-Петербург\\
			2022\\
		\end{center}
	\end{titlepage}
	
	\subsection*{Дано}
	\renewcommand\labelitemi{$\vcenter{\hbox{\tiny$\bullet$}}$}
	\begin{itemize}[topsep=0pt,itemsep=-1ex,partopsep=1ex,parsep=1ex]
		\item $G = (V, A)$ -- полный ориентированный взвешенный граф
		\item $V = \{0, 1, \dots, n\}$ -- множество вершин, где вершина $0$ - склад
		\item $A = \{(i,j) : i, j \in V \; i \neq j\}$ -- множество рёбер
		\item $c_{ij}$ -- использование ребра $\forall \; (i,j) \in A$
		\item $t_{ij}$ -- время на переезд из вершины $i \in V$ в вершину $j \in V$
		\item $(a_i, b_i)$ -- временное окно клиента, когда клиент должен быть обслужен $\forall \; i \in V\backslash\{0\}$
		\item $m$ -- количество коммивояжеров
	\end{itemize}

	\subsection*{Минимизация затрат за выполнение работ}

	\subsubsection*{Переменные}
	\begin{align*}
		x_{ij} = 
		\begin{cases}
			1, \quad \textup{if ребро } (i,j) \textup{ используется в маршрутах коммивояжеров }\\
			0, \quad \textup{в противном случае}
		\end{cases}
	\end{align*}
	\begin{center}
		$\forall \; (i,j) \in A$
	\end{center}
	\begin{center}
		$u_{i} \geq 0$ -- потенциалы вершин-клиентов $\quad \forall \; i \in V\backslash\{0\}$
	\end{center}
	
	\subsubsection*{Целевая функция}
	
	\[ \sum_{(i,j) \in A} c_{ij} \cdot x_{ij} \longrightarrow \min \]
	
	\subsection*{Ограничения}
	
	1) Полустепень исхода для каждой вершины-клиента должна быть равна единице:
	
	\begin{ceqn}
		\begin{align*}
			\sum_{(i,j) \in A} x_{ij} = 1, \quad \forall i \in V\backslash\{0\}
		\end{align*}
	\end{ceqn}

	2) Полустепень захода для каждой вершины-клиента должна быть равна единице:
	
	\begin{ceqn}
		\begin{align*}
			\sum_{(i,j) \in A} x_{ij} = 1, \quad \forall j \in V\backslash\{0\}
		\end{align*}
	\end{ceqn}

	\newpage

	3) Коммивояжер выполняет один маршрут и возвращается на склад:
	
	\begin{ceqn}
		\begin{align*}
			\sum_{(0,j) \in A} x_{0j} = m
		\end{align*}
	\end{ceqn}

	4) Запрет сабтуров (ограничение на потенциалы):

	\begin{ceqn}
		\begin{align*}
			u_i - u_j + (b_i - a_j + t_{ij}) \cdot x_{ij} \leq b_i - a_j, \quad \forall i, j \in V\backslash\{0\} \; i \neq j
		\end{align*}
	\end{ceqn}
	
	5) Ограничение на временные окна:
	
	\begin{ceqn}
		\begin{align*}
			a_i \leq u_i \leq b_i, \quad \forall i \in V\backslash\{0\}
		\end{align*}
	\end{ceqn}

	6) Естественные ограничения:
	\begin{ceqn}
		\begin{align*}
			x_{ij} \in \{0;1\}, \quad \forall \; (i,j) \in A
		\end{align*}
	\end{ceqn}

	\begin{ceqn}
		\begin{align*}
			u_i \geq 0, \quad \forall \; i \in V\backslash\{0\}
		\end{align*}
	\end{ceqn}

	\newpage

	\subsection*{Минимизация суммарной продолжительности выполнения маршрутов}
	
	\subsubsection*{Переменные}
	\begin{align*}
		x_{ij} = 
		\begin{cases}
			1, \quad \textup{if ребро } (i,j) \textup{ используется в маршрутах коммивояжеров }\\
			0, \quad \textup{в противном случае}
		\end{cases}
	\end{align*}
	\begin{center}
		$\forall \; (i,j) \in A$
	\end{center}
	\begin{center}
		$u_{i} \geq 0$ -- потенциалы вершин-клиентов $\quad \forall \; i \in V\backslash\{0\}$
	\end{center}
	
	\subsubsection*{Целевая функция}
	
	\[ u_n \longrightarrow \min \]
	
	\subsection*{Ограничения}
	
	1) Полустепень исхода для каждой вершины-клиента должна быть равна единице:
	
	\begin{ceqn}
		\begin{align*}
			\sum_{(i,j) \in A} x_{ij} = 1, \quad \forall i \in V\backslash\{0\}
		\end{align*}
	\end{ceqn}
	
	2) Полустепень захода для каждой вершины-клиента должна быть равна единице:
	
	\begin{ceqn}
		\begin{align*}
			\sum_{(i,j) \in A} x_{ij} = 1, \quad \forall j \in V\backslash\{0\}
		\end{align*}
	\end{ceqn}
	
	3) Коммивояжер выполняет один маршрут и возвращается на склад:
	
	\begin{ceqn}
		\begin{align*}
			\sum_{(0,j) \in A} x_{0j} = m
		\end{align*}
	\end{ceqn}
	
	4) Запрет сабтуров (ограничение на потенциалы):
	
	\begin{ceqn}
		\begin{align*}
			u_i - u_j + (b_i - a_j + t_{ij}) \cdot x_{ij} \leq b_i - a_j, \quad \forall i, j \in V\backslash\{0\} \; i \neq j
		\end{align*}
	\end{ceqn}
	
	\newpage
	
	5) Ограничение на временные окна:
	
	\begin{ceqn}
		\begin{align*}
			a_i \leq u_i \leq b_i, \quad \forall i \in V\backslash\{0\}
		\end{align*}
	\end{ceqn}

	6) Инициализация времени прибытия к первому клиенту тура и время необходимое для перехода от склада до этого клиента:

	\begin{ceqn}
		\begin{align*}
			u_i - t_{0i} \cdot x_{0i} \geq 0, \quad \forall i \in V\backslash\{0\}
		\end{align*}
	\end{ceqn}

	7) Продолжительность маршрута должна быть меньше или равна времени посещения + время, которое необходимо для возвращения на склад:
	
	\begin{ceqn}
		\begin{align*}
			u_i + t_{i0} \leq _n, \quad \forall i \in V\backslash\{0\}
		\end{align*}
	\end{ceqn}
	
	8) Естественные ограничения:
	\begin{ceqn}
		\begin{align*}
			x_{ij} \in \{0;1\}, \quad \forall \; (i,j) \in A
		\end{align*}
	\end{ceqn}
	
	\begin{ceqn}
		\begin{align*}
			u_i \geq 0, \quad \forall \; i \in V\backslash\{0\}
		\end{align*}
	\end{ceqn}
\end{document}
