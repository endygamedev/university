\documentclass[14pt,fleqn]{extarticle}
\usepackage[T2A,T1]{fontenc}
\usepackage[utf8]{inputenc}
\usepackage[russian]{babel}
\usepackage{amsmath}
\usepackage{graphicx}
\usepackage{tabularx}
\usepackage{boldline}
\usepackage{makecell}
\usepackage{arydshln}
\usepackage{mathtools}
\usepackage{centernot}
\usepackage{enumitem}
\usepackage{nccmath}
\usepackage{amssymb}
\usepackage[a4paper, total={6.5in, 9.5in}]{geometry}

\graphicspath{ {./images/} }
\setlength{\mathindent}{0pt}
\setlength\parindent{0pt}

\def\at{
	\left.
	\vphantom{\int}
	\right|
}


\begin{document}
	\begin{titlepage}
		\includegraphics[scale=0.12]{logo}
		\begin{center}
			\textbf{МИНОБРНАУКИ РОССИИ}\\
			\vspace{0.2cm}
			\textbf{Федеральное государственное бюджетное образовательное учреждение высшего образования}\\
			\textbf{<<САНКТ-ПЕТЕРБУРГСКИЙ ГОСУДАРСТВЕННЫЙ ЭКОНОМИЧЕСКИЙ УНИВЕРСИТЕТ>>}\\
			\vspace{0.6cm}
			Факультет информатики и прикладной математики\\
			Кафедра прикладной математики и экономико-математических методов\\
			\vspace{1cm}
			\textbf{ОТЧЁТ}\\
			по дисциплине:\\
			\textbf{<<Модели комбинаторной оптимизации>>}\\
			на тему:\\
			\textbf{<<Задание №1>>}\\
		\end{center}
		\vspace{1cm}
		Направление: 01.03.02\\
		Обучающийся: Бронников Егор Игоревич\\
		Группа: ПМ-1901\\
		\vfill
		\begin{center}
			Санкт-Петербург\\
			2022\\
		\end{center}
	\end{titlepage}
    \subsection*{Задача №1}
	Сформулировать точную постановку задачи разбиения множества векторов на $k$ групп (MDMWNPP) как задачи целочисленного программирования. Каждый вектор имеет длину $l$, при этом $l_1$ количественных характеристик (положительные вещественные числа), а $l_2$ качественных ($l_1 + l_2 = l$). Для каждой из качественных дано множество допустимых значений. По исходным данные рассчитываются <<идеальные>> суммарные характеристики для групп по количественным характеристикам и по количеству представителей в группах различных значений качественных характеристик. Целевая функция: минимизация суммы абсолютных отклонений суммарных характеристик групп от идеальных значений.\\
	
	\textbf{Решение}\\
	$k$ -- количество групп\\
	$n$ -- количество векторов\\
	$l$ -- количество элементов в векторе\\
	$S = \{s_i = (s_{i1}, s_{i2}, \dots, s_{il}) \; | \; \forall i \in \{1, \dots, n\} \}$ -- мультимножество\\
	$l_1$ -- количество количественных признаков в векторе\\
	$l_2$ -- количество качественных признаков в векторе\\
	Должно выполняться следующее соотношение: $l = l_1 + l_2$\\
	$D_i \quad \forall i \in \{l_1 + 1, \dots, l\}$ -- множество допустимых значений для $i$-ой качественной характеристики\\
	$s_{ij} \in \mathbb{R} \quad \forall i \in \{1, \dots, n\}, \forall j \in \{1, \dots, l_1\}$ -- множество допустимых значений для количественных характеристик\\
	
	Разобьём понятие <<идеальных>> суммарных характеристик для количественных признаков и для качественных признаков.\\
	
	Пусть $\widetilde{p}_1$ -- это <<идеальная>> сумма для количественных признаков, а $\widetilde{p}_2$ -- это <<идеальная>> сумма для качественных признаков.\\

	Тогда <<идеальная>> сумма для количественных признаков может быть найдена следующим образом:
	\begin{ceqn}
		\begin{align*}
			\widetilde{p}_1 = \dfrac{\smashoperator[r]{\sum_{i=1}^{n}} \smashoperator[r]{\sum_{j=1}^{l_1}} s_{ij}}{k}
		\end{align*}
	\end{ceqn}

	\newpage
	
	Теперь переходим к качественным признакам.\\

	Пусть $r = \{|D_{l_1+1}|, \dots, |D_{l}|\}$ -- это вектор мощностей множеств $D_i\\
	\forall i \in \{l_1+1, \dots, l\}$\\
	
	Введём матрицы $I_i = \{ y_{ijk}\}$ -- элементами которых являются значения 0 или 1. (бинарная матрица)
	\begin{align*}
		y_{ijk} = 
		\begin{cases}
			1, \quad \textup{if в } k\textup{-том векторе присутствует } j \textup{-тый признак}\\
			0, \quad \textup{в противном случае}
		\end{cases}
	\end{align*}
	$\forall i \in \{l_1+1, \dots, l\}, \; \forall j \in r, \; \forall k \in \{1, \dots, n\}$\\
	
	$\dim(I_i) = (n \times m)$, где $n$ -- это количество векторов, а $m$ -- это мощность $D_i$\\
	$\forall i \in \{l_1+1, \dots, l\}$

	\noindent\makebox[\linewidth]{\rule{\paperwidth}{0.4pt}}\\
	
	\textit{Пример}\\
	
	Рассмотрим следующее мультимножество:\\
	$S = \{ (1, 2, 3, \textup{red}, \textup{big}, \textup{triangle}), (1, 2, 3, \textup{green}, \textup{big}, \textup{rectangle}), (2, 4, 5, \textup{green}, \textup{small}, \textup{circle})\}$\\
	
	Пусть даны следующие множества допустимых значений качественных признаков:\\
	$D_1 = \{\textup{red}, \textup{green}, \textup{yellow}\}$\\
	$D_2 = \{\textup{big}, \textup{small}\}$\\
	$D_3 = \{\textup{triangle}, \textup{rectangle}, \textup{circle}\}$\\
	
	Составим матрицу $I_1$ для первого качественного признака:
	\begin{align*}
		I_1 = 
		\begin{pmatrix}
			1 & 0 & 0\\
			0 & 1 & 0\\
			0 & 1 & 0\\
		\end{pmatrix}
	\end{align*}

	Составим матрицу $I_2$ для второго качественного признака:
	\begin{align*}
		I_2 = 
		\begin{pmatrix}
			1 & 0\\
			1 & 0\\
			0 & 1\\
		\end{pmatrix}
	\end{align*}

	Составим матрицу $I_3$ для третьего качественного признака:
	\begin{align*}
		I_3 = 
		\begin{pmatrix}
			1 & 0 & 0\\
			0 & 1 & 0\\
			0 & 0 & 1\\
		\end{pmatrix}
	\end{align*}

	\noindent\makebox[\linewidth]{\rule{\paperwidth}{0.4pt}}\\
	
	Тогда можно рассчитать <<идеальную>> сумму для каждого $i$-го качественного признака:
	\begin{ceqn}
		\begin{align*}
			\widetilde{p}_{2_i} = \dfrac{\smashoperator[r]{\sum_{j=1}^{r_i}} I_{ij}}{k}
		\end{align*}
	\end{ceqn}
	
	где $I_{ij}$ -- это $j$-тый столбец матрицы $I_{i}$\\
	
	\noindent\makebox[\linewidth]{\rule{\paperwidth}{0.4pt}}\\
	
	\textit{Пример}\\
	
	Продолжаем предыдущий пример.
	\begin{ceqn}
		\begin{align*}
			\widetilde{p}_{2_1} = \left(\dfrac{1}{k}, \dfrac{2}{k}, 0\right)\\
			\widetilde{p}_{2_2} = \left(\dfrac{2}{k}, \dfrac{1}{k}\right)\\
			\widetilde{p}_{2_3} = \left(\dfrac{1}{k}, \dfrac{1}{k}, \dfrac{1}{k}\right)\\
		\end{align*}
	\end{ceqn}

	\vspace{-1cm}
	
	\noindent\makebox[\linewidth]{\rule{\paperwidth}{0.4pt}}\\

	Введём переменные $x_{ij}$.
	\begin{ceqn}
		\begin{align*}
			x_{ij} = 
			\begin{cases}
				1, \quad \textup{if} \; j \textup{-ый вектор попал в } i \textup{-ую группу} \\
				0, \quad \textup{в противном случае}
			\end{cases}
		\end{align*}
	\end{ceqn}

	$\forall i \in \{1, \dots, k\}, \; \forall j \in \{1, \dots, n\}$, где $n$ -- это количество векторов, а $k$ -- количество групп.
	
	\newpage
	
	Введём $p_1$ -- <<фактическое>> значение суммы для количественных признаков.
	\begin{ceqn}
		\begin{align*}
			p_1 = \smashoperator[r]{\sum_{i=1}^{n}} \smashoperator[r]{\sum_{j=1}^{l_1}} x_{ij} \cdot s_{ij}
		\end{align*}
	\end{ceqn}

	Введём $p_{2_j}$ -- <<фактическое>> значение суммы для $j$-го качественного признака.
	\begin{ceqn}
		\begin{align*}
			p_{2_j} = x_i \cdot I_j
		\end{align*}
	\end{ceqn}

	$\forall i \in \{1, \dots, k\}, \; \forall j \in \{l_1+1, \dots, l\}$, где $k$ -- количество групп.\\
	
	Тогда можно составить целевую функцию.\\
	
	\textit{Целевая функция}:
	\begin{ceqn}
		\begin{align*}
			|\widetilde{p}_{1} - p_1| + \smashoperator[r]{\sum_{i=l_1+1}^{l}} \smashoperator[r]{\sum_{j=1}^{r_i}} |\widetilde{p}_{2_{ij}} - p_{2_{ij}}| \longrightarrow \min
		\end{align*}
	\end{ceqn}

	\textbf{НО}, получившаяся целевая функция не удовлетворяет линейности, поэтому введём дополнительные переменные $z_1$ и $z_{2_{ij}}$.\\
	
	\textit{Целевая функция}:
	\begin{ceqn}
		\begin{align*}
			z_1 + \smashoperator[r]{\sum_{i=l_1+1}^{l}} \smashoperator[r]{\sum_{j=1}^{r_i}} z_{2_{ij}} \longrightarrow \min
		\end{align*}
	\end{ceqn}

	\textit{Ограничения}:
	\begin{ceqn}
		\begin{align*}
			\begin{cases}
				z_1 \geq p_1 - \widetilde{p}_{1}\\
				z_1 \geq \widetilde{p}_{1} - p_1\\
				z_{2_{ij}} \geq \widetilde{p}_{2_{ij}} - p_{2_{ij}}\\
				z_{2_{ij}} \geq p_{2_{ij}} - \widetilde{p}_{2_{ij}}\\
			\end{cases}
		\end{align*}
	\end{ceqn}

	\begin{ceqn}
		\begin{align*}
			\smashoperator[r]{\sum_{i=1}^{k}} x_{ij} = 1 \quad \forall j \in \{1, \dots, n\}
		\end{align*}
	\end{ceqn}

	\begin{ceqn}
		\begin{align*}
			x_{ij} \in \{0;1\} \quad \forall i \in \{1, \dots, k\}, \; \forall j \in \{1, \dots, n\}
		\end{align*}
	\end{ceqn}

	\begin{ceqn}
		\begin{align*}
			p_1 \in \mathbb{R}, \quad z_1 \in \mathbb{R}
		\end{align*}
	\end{ceqn}

	\begin{ceqn}
		\begin{align*}
			p_{2_{ij}} \in \mathbb{R}, \; z_{2_{ij}} \in \mathbb{R} \quad \forall i \in \{l_1+1, \dots, l\}, \; \forall j \in \{1, \dots, r_i\}
		\end{align*}
	\end{ceqn}


	\newpage
    \subsection*{Задача №2}
    Сформулировать точную постановку задачи разбиения мультимножества чисел на $k$ групп (multiway NPP) как задачи целочисленного программирования. Дано множество пар позиций чисел из мультимножества. Если пара чисел на этих позициях оказывается в одной группе, то из суммы чисел в группе вычитается среднее от пары этих чисел. Целевая функция: минимизация разности между наибольшей и наименьшей суммой чисел среди групп, при этом суммы чисел в группах находятся с учетом <<штрафных баллов>>.\\
    
    \textbf{Решение}\\
	$k$ -- количество групп\\
	$n$ -- количество векторов\\
	$S = \{s_1, s_2, \dots, s_n\}$ -- мультимножество\\
	$D = \{(a_1, b_1), (a_2, b_2), \dots, (a_m, b_m)\}$ - множество пар позиций чисел из мультимножества\\
	$m$ -- мощность множества пар позиций чисел\\
	
	\noindent\makebox[\linewidth]{\rule{\paperwidth}{0.4pt}}\\
	
	\textit{Пример}\\
	
	Мультимножество:\\
	$S = \{10, 20, 30, 40, 50, 60, 70, 80, 90, 100 \}$\\
	
	Множество пар позиций чисел из этого мультимножества:\\
	$D = \{(1,2), (2,3), (4,5), (4,6), (7,8)\}$\\
	
	\noindent\makebox[\linewidth]{\rule{\paperwidth}{0.4pt}}\\
	
	Пусть $p_i$ -- это сумма чисел в $i$-ой группе. $\forall i \in \{1, \dots, k\}$\\
	Тогда пусть $p_{\max}$ -- это максимальная сумма среди всех групп, а $p_{\min}$ -- это минимальная сумма среди всех групп.\\
	
	Введём переменные $x_{ij}$.
	\begin{ceqn}
		\begin{align*}
			x_{ij} = 
			\begin{cases}
				1, \quad \textup{if} \; j \textup{-ое число попало в } i \textup{-ую группу} \\
				0, \quad \textup{в противном случае}
			\end{cases}\\
			\forall i \in \{1, \dots, k\}, \quad \forall j \in \{1, \dots, n\}
		\end{align*}
	\end{ceqn}
	

	\newpage
	
	Введём параметры $y_{ij}$.
	\begin{ceqn}
		\begin{align*}
			y_{lj} = 
			\begin{cases}
				1, \quad \textup{if в паре } i \textup{ есть } j \textup{-ое число} \\
				0, \quad \textup{в противном случае}
			\end{cases}\\
			\forall l \in \{1, \dots, m\}, \quad \forall j \in \{1, \dots, n\}
		\end{align*}
	\end{ceqn}

	Если $\smashoperator[r]{\sum_{j=1}^{n}} x_{ij} \cdot y_{lj} \geq 2$, то значит в группе есть числа, позиции которых находятся в одной паре. В противном случае в группе нет чисел, позиции которых находятся в одной паре.\\
	
	Хотелось бы сделать так:
	
	\textbf{if} $\smashoperator[r]{\sum_{j=1}^{n}} x_{ij} \cdot y_{lj} \geq 2$\\
	\textbf{then:}\\
	$\smashoperator[r]{\sum_{j=1}^{n}} x_{ij} \cdot s_j - \smashoperator[r]{\sum_{j=1}^{n}} \dfrac{a_{lj} \cdot s_j}{2} \leq p_{\max}$\\
	$\smashoperator[r]{\sum_{j=1}^{n}} x_{ij} \cdot s_j - \smashoperator[r]{\sum_{j=1}^{n}} \dfrac{a_{lj} \cdot s_j}{2} \geq p_{\min}$\\
	\textbf{else:}\\
	$\smashoperator[r]{\sum_{j=1}^{n}} x_{ij} \cdot s_j \leq p_{\max}$\\
	$\smashoperator[r]{\sum_{j=1}^{n}} x_{ij} \cdot s_j \geq p_{\min}$\\
	
	$\forall i \in \{1, \dots, k\}, \quad \forall l \in \{1, \dots, m\}, \quad \forall j \in \{1, \dots, n\}$\\
	
	Импликацию можно заменить следующим образом:
	\begin{center}
		$A_1 \Rightarrow A_2 \longrightarrow$ NOT $A_1$ OR $A_2$.
	\end{center}
	Если рассуждать на языке алгебры логики, то имеются следующие события $A$ и $B$:\\
	$A$: \hspace{0.1cm} $\smashoperator[r]{\sum_{j=1}^{n}} x_{ij} \cdot y_{lj} < 2$ OR
	($\smashoperator[r]{\sum_{j=1}^{n}} x_{ij} \cdot s_j - \smashoperator[r]{\sum_{j=1}^{n}} \dfrac{a_{lj} \cdot s_j}{2} \leq p_{\max}$ AND $\smashoperator[r]{\sum_{j=1}^{n}} x_{ij} \cdot s_j - \smashoperator[r]{\sum_{j=1}^{n}} \dfrac{a_{lj} \cdot s_j}{2} \geq p_{\min}$)\\
	
	$B$: \hspace{0.1cm} $\smashoperator[r]{\sum_{j=1}^{n}} x_{ij} \cdot y_{lj} \geq 2$ OR ($\smashoperator[r]{\sum_{j=1}^{n}} x_{ij} \cdot s_j \leq p_{\max}$ AND $\smashoperator[r]{\sum_{j=1}^{n}} x_{ij} \cdot s_j \geq p_{\min}$)\\\\
	
	Нам бы хотелось в конечном итоге, чтобы выполнялись условия: $A$ OR $B$.\\
	
	\textit{Рассмотрим более подробно событие $A$.}\\
	Пусть $M_A$ -- большое число и введём параметр $z_A \in \{0;1\}$, тогда событие $A$ можно переписать следующим образом:
	\begin{ceqn}
		\begin{align*}
			\begin{cases}
				\smashoperator[r]{\sum_{j=1}^{n}} x_{ij} \cdot y_{lj} < 2 + M_A \cdot z_A\\
				\smashoperator[r]{\sum_{j=1}^{n}} x_{ij} \cdot s_j - \smashoperator[r]{\sum_{j=1}^{n}} \dfrac{a_{lj} \cdot s_j}{2} \leq p_{\max} +  M_A \cdot (1 - z_A)\\
				\smashoperator[r]{\sum_{j=1}^{n}} x_{ij} \cdot s_j - \smashoperator[r]{\sum_{j=1}^{n}} \dfrac{a_{lj} \cdot s_j}{2} \geq p_{\min} +  M_A \cdot (1 - z_A)\\
			\end{cases}\\
			\forall i \in \{1, \dots, k\}, \quad \forall l \in \{1, \dots, m\}, \quad \forall j \in \{1, \dots, n\}
		\end{align*}
	\end{ceqn}
	
	\textit{Рассмотрим более подробно событие $B$.}\\
	Пусть $M_B$ -- большое число и введём параметр $z_B \in \{0;1\}$, тогда событие $B$ можно переписать следующим образом:
	\begin{ceqn}
		\begin{align*}
			\begin{cases}
				\smashoperator[r]{\sum_{j=1}^{n}} x_{ij} \cdot y_{lj} \geq 2 + M_B \cdot z_B\\
				\smashoperator[r]{\sum_{j=1}^{n}} x_{ij} \cdot s_j \leq p_{\max} +  M_B \cdot (1 - z_B)\\
				\smashoperator[r]{\sum_{j=1}^{n}} x_{ij} \cdot s_j \geq p_{\min} + M_B \cdot (1 - z_B)\\
			\end{cases}\\
			\forall i \in \{1, \dots, k\}, \quad \forall l \in \{1, \dots, m\}, \quad \forall j \in \{1, \dots, n\}
		\end{align*}
	\end{ceqn}

	\newpage
	
	Осталось всё собрать вместе и рассмотреть событие: $A$ OR $B$.\\
	Пусть $M$ -- большое число и введём параметр $z \in \{0;1\}$, тогда событие $A$ OR $B$ можно переписать следующим образом:
	
	\begin{ceqn}
	\begin{align*}
		\begin{cases}
			\smashoperator[r]{\sum_{j=1}^{n}} x_{ij} \cdot y_{lj} < 2 + M_A \cdot z_A + M \cdot z\\
			\smashoperator[r]{\sum_{j=1}^{n}} x_{ij} \cdot s_j - \smashoperator[r]{\sum_{j=1}^{n}} \dfrac{a_{lj} \cdot s_j}{2} \leq p_{\max} +  M_A \cdot (1 - z_A) + M \cdot z\\
			\smashoperator[r]{\sum_{j=1}^{n}} x_{ij} \cdot s_j - \smashoperator[r]{\sum_{j=1}^{n}} \dfrac{a_{lj} \cdot s_j}{2} \geq p_{\min} +  M_A \cdot (1 - z_A) + M \cdot z\\
			\smashoperator[r]{\sum_{j=1}^{n}} x_{ij} \cdot y_{lj} \geq 2 + M_B \cdot z_B + M \cdot (1-z)\\
			\smashoperator[r]{\sum_{j=1}^{n}} x_{ij} \cdot s_j \leq p_{\max} +  M_B \cdot (1 - z_B) + M \cdot (1-z)\\
			\smashoperator[r]{\sum_{j=1}^{n}} x_{ij} \cdot s_j \geq p_{\min} + M_B \cdot (1 - z_B) + M \cdot (1-z)\\
		\end{cases}\\
		\forall i \in \{1, \dots, k\}, \quad \forall l \in \{1, \dots, m\}, \quad \forall j \in \{1, \dots, n\}
	\end{align*}
	\end{ceqn}
	
	\newpage
	
	\textit{Целевая функция:}
	\begin{ceqn}
		\begin{align*}
			p_{\max} - p_{\min} \longrightarrow \min
		\end{align*}
	\end{ceqn}

	\textit{Ограничения:}
	\begin{ceqn}
		\begin{align*}
			\begin{cases}
				\smashoperator[r]{\sum_{j=1}^{n}} x_{ij} \cdot y_{lj} < 2 + M_A \cdot z_A + M \cdot z\\
				\smashoperator[r]{\sum_{j=1}^{n}} x_{ij} \cdot s_j - \smashoperator[r]{\sum_{j=1}^{n}} \dfrac{a_{lj} \cdot s_j}{2} \leq p_{\max} +  M_A \cdot (1 - z_A) + M \cdot z\\
				\smashoperator[r]{\sum_{j=1}^{n}} x_{ij} \cdot s_j - \smashoperator[r]{\sum_{j=1}^{n}} \dfrac{a_{lj} \cdot s_j}{2} \geq p_{\min} +  M_A \cdot (1 - z_A) + M \cdot z\\
				\smashoperator[r]{\sum_{j=1}^{n}} x_{ij} \cdot y_{lj} \geq 2 + M_B \cdot z_B + M \cdot (1-z)\\
				\smashoperator[r]{\sum_{j=1}^{n}} x_{ij} \cdot s_j \leq p_{\max} +  M_B \cdot (1 - z_B) + M \cdot (1-z)\\
				\smashoperator[r]{\sum_{j=1}^{n}} x_{ij} \cdot s_j \geq p_{\min} + M_B \cdot (1 - z_B) + M \cdot (1-z)\\
				\smashoperator[r]{\sum_{i=1}^{k}} x_{ij} = 1
			\end{cases}\\
			\forall i \in \{1, \dots, k\}, \quad \forall l \in \{1, \dots, m\}, \quad \forall j \in \{1, \dots, n\}
		\end{align*}
	\end{ceqn}

	\begin{ceqn}
		\begin{align*}
			x_{ij} \in \{0;1\} \quad \forall i \in \{1, \dots, k\}, \; \forall j \in \{1, \dots, n\}
		\end{align*}
	\end{ceqn}

	\begin{ceqn}
	\begin{align*}
		y_{lj} \in \{0;1\} \quad \forall l \in \{1, \dots, m\}, \quad \forall j \in \{1, \dots, n\}
	\end{align*}
	\end{ceqn}

	\begin{ceqn}
	\begin{align*}
		z_A \in \{0;1\} \quad z_B \in \{0;1\} \quad z \in \{0;1\}
	\end{align*}
	\end{ceqn}

	\begin{ceqn}
	\begin{align*}
		p_{min} \in \mathbb{R}, \, p_{max} \in \mathbb{R}, \, M_A \in \mathbb{R}, \, M_B \in \mathbb{R}, \, M \in \mathbb{R}
	\end{align*}
	\end{ceqn}

\end{document}
