\documentclass[14pt,fleqn]{extarticle}
\usepackage[T2A,T1]{fontenc}
\usepackage[utf8]{inputenc}
\usepackage[russian]{babel}
\usepackage{amsmath}
\usepackage{graphicx}
\usepackage{tabularx}
\usepackage{boldline}
\usepackage{makecell}
\usepackage{arydshln}
\usepackage{mathtools}
\usepackage{centernot}
\usepackage{enumitem}
\usepackage{nccmath}
\usepackage[a4paper, total={6.5in, 9.5in}]{geometry}

\graphicspath{ {./images/} }
\setlength{\mathindent}{0pt}
\setlength\parindent{0pt}

\def\at{
	\left.
	\vphantom{\int}
	\right|
}


\begin{document}
	\begin{titlepage}
		\includegraphics[scale=0.12]{logo}
		\begin{center}
			\textbf{МИНОБРНАУКИ РОССИИ}\\
			\vspace{0.2cm}
			\textbf{Федеральное государственное бюджетное образовательное учреждение высшего образования}\\
			\textbf{<<САНКТ-ПЕТЕРБУРГСКИЙ ГОСУДАРСТВЕННЫЙ ЭКОНОМИЧЕСКИЙ УНИВЕРСИТЕТ>>}\\
			\vspace{0.6cm}
			Факультет информатики и прикладной математики\\
			Кафедра прикладной математики и экономико-математических методов\\
			\vspace{1cm}
			\textbf{ОТЧЁТ}\\
			по дисциплине:\\
			\textbf{<<Модели комбинаторной оптимизации>>}\\
			на тему:\\
			\textbf{<<Задание №1>>}\\
		\end{center}
		\vspace{1cm}
		Направление: 01.03.02\\
		Обучающийся: Бронников Егор Игоревич\\
		Группа: ПМ-1901\\
		\vfill
		\begin{center}
			Санкт-Петербург\\
			2022\\
		\end{center}
	\end{titlepage}
    \subsection*{Задача №1}
	Сформулировать точную постановку задачи разбиения множества векторов на $k$ групп (MDMWNPP) как задачи целочисленного программирования. Каждый вектор имеет длину $l$, при этом $l_1$ количественных характеристик (положительные вещественные числа), а $l_2$ качественных ($l_1 + l_2 = l$). Для каждой из качественных дано множество допустимых значений. По исходным данные рассчитываются <<идеальные>> суммарные характеристики для групп по количественным характеристикам и по количеству представителей в группах различных значений качественных характеристик. Целевая функция: минимизация суммы абсолютных отклонений суммарных характеристик групп от идеальных значений.\\\\
	
	\textbf{Решение}
	
	...
	
	\newpage
    \subsection*{Задача №2}
    Сформулировать точную постановку задачи разбиения мультимножества чисел на $k$ групп (multiway NPP) как задачи целочисленного программирования. Дано множество пар позиций чисел из мультимножества. Если пара чисел на этих позициях оказывается в одной группе, то из суммы чисел в группе вычитается среднее от пары этих чисел. Целевая функция: минимизация разности между наибольшей и наименьшей суммой чисел среди групп, при этом суммы чисел в группах находятся с учетом <<штрафных баллов>>.\\\\
    
    \textbf{Решение}
    
    ...
\end{document}
