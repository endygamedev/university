\documentclass[14pt,fleqn]{extarticle}
\usepackage[T2A,T1]{fontenc}
\usepackage[utf8]{inputenc}
\usepackage[russian]{babel}
\usepackage{amsmath}
\usepackage{graphicx}
\usepackage{tabularx}
\usepackage{boldline}
\usepackage{makecell}
\usepackage{arydshln}
\usepackage{mathtools}
\usepackage{centernot}
\usepackage{enumitem}
\usepackage{nccmath}
\usepackage{amssymb}
\usepackage[a4paper, total={6.5in, 9.5in}]{geometry}

\graphicspath{ {./images/} }
\setlength{\mathindent}{0pt}
\setlength\parindent{0pt}

\def\at{
	\left.
	\vphantom{\int}
	\right|
}


\begin{document}
	\begin{titlepage}
		\includegraphics[scale=0.12]{logo}
		\begin{center}
			\textbf{МИНОБРНАУКИ РОССИИ}\\
			\vspace{0.2cm}
			\textbf{Федеральное государственное бюджетное образовательное учреждение высшего образования}\\
			\textbf{<<САНКТ-ПЕТЕРБУРГСКИЙ ГОСУДАРСТВЕННЫЙ ЭКОНОМИЧЕСКИЙ УНИВЕРСИТЕТ>>}\\
			\vspace{0.6cm}
			Факультет информатики и прикладной математики\\
			Кафедра прикладной математики и экономико-математических методов\\
			\vspace{1cm}
			\textbf{ОТЧЁТ}\\
			по дисциплине:\\
			\textbf{<<Модели комбинаторной оптимизации>>}\\
			на тему:\\
			\textbf{<<Задание №10. Развитие задачи планирования работ IT-компании>>}\\
		\end{center}
		\vspace{1cm}
		Направление: 01.03.02\\
		Обучающийся: Бронников Егор Игоревич\\
		Группа: ПМ-1901\\
		\vfill
		\begin{center}
			Санкт-Петербург\\
			2022\\
		\end{center}
	\end{titlepage}
	
	\subsection*{Дано}
	\renewcommand\labelitemi{$\vcenter{\hbox{\tiny$\bullet$}}$}
	\begin{itemize}[topsep=0pt,itemsep=-1ex,partopsep=1ex,parsep=1ex]
		\item $k$ -- количество проектов
		\item $n$ -- количество работников
		\item $m$ -- количество задач
		\item $G = \{g_1, g_2, \dots, g_k\}$ -- множество проектов
		\item $W = \{w_1, w_2, \dots, w_n\}$ -- множество работников
		\item $Q = \{q_1, q_2, \dots, q_m\}$ -- множество задач
		\item $Q_w$ -- множество задач, которые может выполнять работник $w \in W$
		\item $W_q$ -- множество работников, которые могут выполнять задачу $q \in Q$
		\item $T_g$ -- дедлайн проекта $g \in G$
		\item $T = \max_{g \in G} T_g$ -- время выполнения последнего проекта
		\item $[0; T]$ -- горизонт планирования
		\item $t_q$ -- время выполнения задачи $q \in Q$
		\item $\pi_w$ -- список предпочтений, для каждого работника $\forall \, w \in W$, упорядоченное множество элементов $Q_w$, расположенных в порядке убывания приоритета заданий для сотрудника
		\item $M$ -- матрица смежности графов-проектов (связи задач родитель-ребёнок внутри проектов)
		\item $R: G \times Q \longrightarrow \{0; 1\}$ -- матрица принадлежности задач проектам
		\item $\forall w \in W, \; \forall i \in \pi_w: position(i) \in \{1, \dots, |Q_w|\}$ -- позиция задачи $i$ в списке предпочтений работника $w \in W$ (чем меньше данное значение, тем предпочтительнее для данного работника это задание)
		\item $\forall \, g \in G: sources(g)$ -- множество заданий, которые не имеют предыдущих заданий и которые относятся к проекту $g$
		\item $\forall \, g \in G: sinks(g)$ -- множество заданий, которые не имеют последующих заданий и которые относятся к проекту $g$
		\item $K$ -- большое число
	\end{itemize}

	\subsection*{Переменные}
	\begin{align*}
		x_{i,j,w} = 
		\begin{cases}
			1, \quad \textup{if задача } i \textup{ выполняется после задачи } j \textup{ работником } w\\
			0, \quad \textup{в противном случае}
		\end{cases}
	\end{align*}
	\begin{center}
		$u_{i,w} \geq 0$ -- начало выполнения задачи $i$ работником $w$
	\end{center}
	
	\newpage
	
	\subsection*{Целевая функция}
	
	1) Минимизация времени окончания работ по проектам:
	\[ \sum_{w \in W} \sum_{g \in G} \sum_{source \in sources(g)} \sum_{sink \in sinks(g)} u_{sink, w} - u_{source, w} \longrightarrow \min \]
	
	2) Удовлетворение предпочтений сотрудников:
	
	\[ \sum_{w \in W} \sum_{i \in \pi_w} position(i) \sum_{j \in Q_w} x_{i,j,w} \longrightarrow \min \]
	
	3) Минимизация количества переключений работников между проектами:
	
	\[ \sum_{w \in W} \sum_{i \in Q_w} \; \sum_{j \in \{q \, | \, q \in Q_w \forall \, g \in G \; \forall \, r_{g,i}, r_{g,q} \in R: \, r_{g,i} \neq r_{g,q}\}} \longrightarrow \min \]\\
	
	\textit{Итоговая целевая функция:}
	
	\[ \sum_{w \in W} \sum_{g \in G} \sum_{source \in sources(g)} \sum_{sink \in sinks(g)} u_{sink, w} - u_{source, w} + \sum_{w \in W} \sum_{i \in \pi_w} position(i) \sum_{j \in Q_w} x_{i,j,w} + \]\\
	\[ + \sum_{w \in W} \sum_{i \in Q_w} \; \sum_{j \in \{q \, | \, q \in Q_w \forall \, g \in G \; \forall \, r_{g,i}, r_{g,q} \in R: \, r_{g,i} \neq r_{g,q}\}} K \cdot x_{i,j,k} \longrightarrow \min \]\\
	
	\subsection*{Ограничения}
	
	1) Каждую задачу выполняет ровно один сотрудник из списка допустимых:
	
	\begin{ceqn}
		\begin{align*}
			\sum_{w \in W_j} \sum_{i \in Q_w} x_{i,j,k} = 1, \quad \forall \; j \in Q
		\end{align*}
	\end{ceqn}

	2) Задача может начать выполняться, если её задачи-предки выполнены:
	
	\begin{ceqn}
		\begin{align*}
			\sum_{w \in W_j} u_{j,w} - \sum_{w \in W_i} u_{i,w} \geq t_i, \quad \forall \; i, j \in Q: \; m_{i,j} = 1
		\end{align*}
	\end{ceqn}

	\newpage

	3) Дедлайн не нарушается:
	
	\begin{ceqn}
		\begin{align*}
			\sum_{w \in W} \sum_{i \in \{q \; | \; q \in Q: \; r_{g,q} = 1\}} u_{i, w} + t_i \leq T_g, \quad \forall \; g \in G
		\end{align*}
	\end{ceqn}

	4) В каждый момент времени сотрудник выполняет не более одной задачи (ограничение на потенциалы):

	\begin{ceqn}
		\begin{align*}
			u_{i,w} - u_{j,w} + K \cdot x_{i,j,w} \leq K - t_i, \quad \forall \; w \in W, \, \forall \; i, j \in Q_w: i \neq j
		\end{align*}
	\end{ceqn}

	5) Естественные ограничения:
	\begin{ceqn}
		\begin{align*}
			x_{i,j,w} \in \{0;1\}, \quad \forall \; w \in W, \, \forall \; i, j \in Q_w: i \neq j
		\end{align*}
	\end{ceqn}

	\begin{ceqn}
		\begin{align*}
			u_{i, w} \geq 0, \quad \forall \; w \in W, \, \forall \; i \in Q_w
		\end{align*}
	\end{ceqn}
\end{document}
