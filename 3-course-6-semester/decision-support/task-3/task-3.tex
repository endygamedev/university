\documentclass[14pt,fleqn]{extarticle}
\usepackage[T2A,T1]{fontenc}
\usepackage[utf8]{inputenc}
\usepackage[russian]{babel}
\usepackage{amsmath}
\usepackage{graphicx}
\usepackage{tabularx}
\usepackage{boldline}
\usepackage{makecell}
\usepackage{arydshln}
\usepackage{mathtools}
\usepackage{enumitem}
\usepackage[a4paper, total={6.5in, 9.5in}]{geometry}

\graphicspath{ {./images/} }
\setlength{\mathindent}{0pt}
\setlength\parindent{0pt}
\begin{document}
	\begin{titlepage}
		\includegraphics[scale=0.12]{logo}
		\begin{center}
			\textbf{МИНОБРНАУКИ РОССИИ}\\
			\vspace{0.2cm}
			\textbf{Федеральное государственное бюджетное образовательное учреждение высшего образования}\\
			\textbf{«САНКТ-ПЕТЕРБУРГСКИЙ ГОСУДАРСТВЕННЫЙ ЭКОНОМИЧЕСКИЙ УНИВЕРСИТЕТ»}\\
			\vspace{0.6cm}
			Факультет информатики и прикладной математики\\
			Кафедра прикладной математики и экономико-математических методов\\
			\vspace{1cm}
			\textbf{ОТЧЁТ}\\
			по дисциплине:\\
			\textbf{«Теория и системы поддержки принятия решений»}\\
			на тему:\\
			\textbf{«Разностные уравнения. Задание 3»}\\
		\end{center}
		\vspace{1cm}
		Направление: 01.03.02\\
		Обучающийся: Бронников Егор Игоревич\\
		Группа: ПМ-1901\\
		\vfill
		\begin{center}
			Санкт-Петербург\\
			2022\\
		\end{center}
	\end{titlepage}

	\section*{Задача 9}
	\textit{Задание:} Найти общее решение однородного линейного разностного уравнения:
	\begin{center}
		$u_{s+4} - 7u_{s+3} + 22u_{s+2} - 32u_{s+1} + 16u_s = 0$
	\end{center}

	\textit{Решение:}\\
	Перепишем исходное уравнение в виде для $s \geq 4$:
	\begin{center}
		$u_{s} - 7u_{s-1} + 22u_{s-2} - 32u_{s-3} + 16u_{s-4} = 0$
	\end{center}
	Для получения общего решения однородного разностного уравнения находим фундаментальную систему решений, элементы которой ищем в виде:
	\begin{center}
		$\breve{u_s} = Ch^s$
	\end{center}
	Тогда
	\begin{center}
		$h^s - 7h^{s-1} + 22h^{s-2} - 32h^{s-3} + 16h^{s-4} = 0$
	\end{center}
	Находим харатеристическое уравнение:
	\begin{center}
		$H(t) = h^4 - 7h^3 + 22h^2 - 32h + 16 = 0$
	\end{center}
	Решаем уравнение, получим следующие корни: $h_1 = 1, h_2 = 2, h_3 = 2-2i$\\$h_4 = 2+2i$.\\
	Можно видеть, что у нас нет кратных корней, но есть комлексные корни. Запишем представление комлексного числа в тригонометрической форме:
	\begin{center}
		$\hat{u_s} = r^s((C_1+C_2)\cos(s\phi)+i(C_1-C_2)\sin(s\phi))$
	\end{center}
	Отсюда следует, чтобы получить действительное решение, постонные надо взять комлексно-сопряжёнными: $C_{1,2} = U(\cos\theta \pm i\sin\theta)$.\\
	В итоге, получается:
	\begin{center}
		$\hat{u_s} = 2Ur^s\cos(s\phi+\theta)$
	\end{center}
	где $r = 2\sqrt{2}, \phi = \dfrac{\pi}{4}$\\\\
	Таким образом, общее решение будет выглядеть следующим образом:
	\begin{center}
		$\breve{u_s} = C_1 + 2^sC_2 + 2U(2\sqrt{2})^s\cos(\dfrac{\pi}{4}s+\theta)$
	\end{center}

	\newpage
	
	\section*{Задача 10}
	\textit{Задание:} Найти  общее решение неоднородного линейного разностного уравнения первого порядка методом итераций и методом обратного оператора при помощи z-преобразования:
	\begin{center}
		$u_{s+1} = \dfrac{6}{7}u_s + \dfrac{s+6}{s+7}$
	\end{center}
	\subsection*{Метод итераций}
	Пусть дано $u_0$:\\
	
	$u_1 = \dfrac{6}{7}u_0 + \dfrac{7}{8}$\\
	
	$u_2 = \dfrac{6}{7}u_1 + \dfrac{8}{9} = \dfrac{6}{7}\left(\dfrac{6}{7}u_0 + \dfrac{7}{8}\right) + \dfrac{8}{9} = \left(\dfrac{6}{7}\right)^2u_0 + \dfrac{6 \cdot 7}{7 \cdot 8} + \dfrac{8}{9}$\\\\
	
	$u_3 = \dfrac{6}{7}u_2 + \dfrac{9}{10} = \dfrac{6}{7}\left(\left(\dfrac{6}{7}\right)^2u_0 + \dfrac{6 \cdot 7}{7 \cdot 8} + \dfrac{8}{9}\right) + \dfrac{9}{10} =$\\
	\[
		\qquad = \left(\dfrac{6}{7}\right)^3u_0 + \left(\dfrac{6}{7}\right)^2\cdot\dfrac{7}{8} + \dfrac{6}{7}\cdot\dfrac{8}{9} + \dfrac{9}{10}
	\]
	\newline
	
	$u_4 = \dfrac{6}{7}u_3 + \dfrac{10}{11} = \dfrac{6}{7}\left(\left(\dfrac{6}{7}\right)^3u_0 + \left(\dfrac{6}{7}\right)^2\cdot\dfrac{7}{8} + \dfrac{6}{7}\cdot\dfrac{8}{9} + \dfrac{9}{10}\right) + \dfrac{10}{11} =$\\
	\[
		\qquad = \left(\dfrac{6}{7}\right)^4u_0 + \left(\dfrac{6}{7}\right)^3\cdot\dfrac{7}{8} + \left(\dfrac{6}{7}\right)^2\cdot\dfrac{8}{9} + \dfrac{6}{7}\cdot\dfrac{9}{10} + \dfrac{10}{11}
	\]
	
	\begin{center}
		$\ldots$
	\end{center}
	Общая формула при $s \geq 1$:\\\\
	$u_s = \left(\dfrac{6}{7}\right)^su_0 + \smashoperator[r]{\sum_{i=0}^{s-1}} \left(\dfrac{6}{7}\right)^i \dfrac{(s-i) + 6}{(s-i) + 7}$
	\newpage
	\subsection*{Z-преобразование}
\end{document}