\documentclass[14pt,fleqn]{extarticle}
\usepackage[T2A,T1]{fontenc}
\usepackage[utf8]{inputenc}
\usepackage[russian]{babel}
\usepackage{amsmath}
\usepackage{graphicx}
\usepackage{tabularx}
\usepackage{boldline}
\usepackage{makecell}
\usepackage{arydshln}
\usepackage{mathtools}
\usepackage{enumitem}
\usepackage[a4paper, total={6.5in, 9.5in}]{geometry}

\graphicspath{ {./images/} }
\setlength{\mathindent}{0pt}
\setlength\parindent{0pt}
\begin{document}
	\begin{titlepage}
		\includegraphics[scale=0.12]{logo}
		\begin{center}
			\textbf{МИНОБРНАУКИ РОССИИ}\\
			\vspace{0.2cm}
			\textbf{Федеральное государственное бюджетное образовательное учреждение высшего образования}\\
			\textbf{«САНКТ-ПЕТЕРБУРГСКИЙ ГОСУДАРСТВЕННЫЙ ЭКОНОМИЧЕСКИЙ УНИВЕРСИТЕТ»}\\
			\vspace{0.6cm}
			Факультет информатики и прикладной математики\\
			Кафедра прикладной математики и экономико-математических методов\\
			\vspace{1cm}
			\textbf{ОТЧЁТ}\\
			по дисциплине:\\
			\textbf{«Теория и системы поддержки принятия решений»}\\
			на тему:\\
			\textbf{«Разностные уравнения. Задание 3»}\\
		\end{center}
		\vspace{1cm}
		Направление: 01.03.02\\
		Обучающийся: Бронников Егор Игоревич\\
		Группа: ПМ-1901\\
		\vfill
		\begin{center}
			Санкт-Петербург\\
			2022\\
		\end{center}
	\end{titlepage}

	\section*{Задача 9}
	\textit{Задание:} Найти общее решение однородного линейного разностного уравнения:
	\begin{center}
		$u_{s+4} - 7u_{s+3} + 22u_{s+2} - 32u_{s+1} + 16u_s = 0$
	\end{center}

	\textit{Решение:}\\
	Перепишем исходное уравнение в виде для $t \geq 4$:
	\begin{center}
		$u_{t} - 7u_{t-1} + 22u_{t-2} - 32u_{t-3} + 16u_{t-4} = 0$
	\end{center}
	Будем искать решение в виде: $\breve{u_t} = Ch^t$
	\begin{center}
		$h^t - 7h^{t-1} + 22h^{t-2} - 32h^{t-3} + 16h^{t-4} = 0$
	\end{center}
	Находим харатеристическое уравнение:
	\begin{center}
		$H(t) = h^4 - 7h^3 + 22h^2 - 32h + 16 = 0$
	\end{center}
	Решаем уравнение, получим следующие корни: $h_1 = 1, h_2 = 2, h_3 = 2-2i$\\$h_4 = 2+2i$.\\
	Можно видеть, что у нас нет кратных корней, но есть комлексные корни. Запишем представление комлексного числа в тригонометрической форме:
	\begin{center}
		$\hat{y_t} = r^t((C_1+C_2)\cos(t\phi)+i(C_1-C_2)\sin(t\phi))$
	\end{center}
	Отсюда следует, чтобы получить действительное решение, постонные надо взять комлексно-сопряжёнными: $C_{1,2} = U(\cos\theta \pm i\sin\theta)$.\\
	В итоге, получается:
	\begin{center}
		$\hat{y_t} = 2Ur^t\cos(t\phi+\theta)$
	\end{center}
	где $r = 2\sqrt{2}, \phi = \dfrac{\pi}{4}$\\\\
	Таким образом, общее решение будет выглядеть следующим образом:
	\begin{center}
		$\breve{u_t} = C_1 + 2^tC_2 + 2U(2\sqrt{2})^t\cos(\dfrac{\pi}{4}t+\theta)$
	\end{center}
\end{document}