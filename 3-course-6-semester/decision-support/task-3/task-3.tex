\documentclass[14pt,fleqn]{extarticle}
\usepackage[T2A,T1]{fontenc}
\usepackage[utf8]{inputenc}
\usepackage[russian]{babel}
\usepackage{amsmath}
\usepackage{graphicx}
\usepackage{tabularx}
\usepackage{boldline}
\usepackage{makecell}
\usepackage{arydshln}
\usepackage{mathtools}
\usepackage{enumitem}
\usepackage[a4paper, total={6.5in, 9.5in}]{geometry}

\graphicspath{ {./images/} }
\setlength{\mathindent}{0pt}
\setlength\parindent{0pt}
\begin{document}
	\begin{titlepage}
		\includegraphics[scale=0.12]{logo}
		\begin{center}
			\textbf{МИНОБРНАУКИ РОССИИ}\\
			\vspace{0.2cm}
			\textbf{Федеральное государственное бюджетное образовательное учреждение высшего образования}\\
			\textbf{«САНКТ-ПЕТЕРБУРГСКИЙ ГОСУДАРСТВЕННЫЙ ЭКОНОМИЧЕСКИЙ УНИВЕРСИТЕТ»}\\
			\vspace{0.6cm}
			Факультет информатики и прикладной математики\\
			Кафедра прикладной математики и экономико-математических методов\\
			\vspace{1cm}
			\textbf{ОТЧЁТ}\\
			по дисциплине:\\
			\textbf{«Теория и системы поддержки принятия решений»}\\
			на тему:\\
			\textbf{«Разностные уравнения. Задание 3»}\\
		\end{center}
		\vspace{1cm}
		Направление: 01.03.02\\
		Обучающийся: Бронников Егор Игоревич\\
		Группа: ПМ-1901\\
		\vfill
		\begin{center}
			Санкт-Петербург\\
			2022\\
		\end{center}
	\end{titlepage}

	\section*{Задача 9}
	\textit{Задание:} Найти общее решение однородного линейного разностного уравнения:
	\begin{center}
		$u_{s+4} - 7u_{s+3} + 22u_{s+2} - 32u_{s+1} + 16u_s = 0$
	\end{center}

	\textit{Решение:}\\
	Перепишем исходное уравнение в виде для $s \geq 4$:
	\begin{center}
		$u_{s} - 7u_{s-1} + 22u_{s-2} - 32u_{s-3} + 16u_{s-4} = 0$
	\end{center}
	Для получения общего решения однородного разностного уравнения находим фундаментальную систему решений, элементы которой ищем в виде:
	\begin{center}
		$\breve{u}_s = Ch^s$
	\end{center}
	Тогда
	\begin{center}
		$h^s - 7h^{s-1} + 22h^{s-2} - 32h^{s-3} + 16h^{s-4} = 0$
	\end{center}
	Находим харатеристическое уравнение:
	\begin{center}
		$H(h) = h^4 - 7h^3 + 22h^2 - 32h + 16 = 0$
	\end{center}
	Решаем уравнение, получим следующие корни: $h_1 = 1, h_2 = 2, h_3 = 2-2i$\\$h_4 = 2+2i$.\\
	Можно видеть, что у нас нет кратных корней, но есть комлексные корни. Запишем представление комлексного числа в тригонометрической форме:
	\begin{center}
		$\hat{u}_s = r^s((C_1+C_2)\cos(s\phi)+i(C_1-C_2)\sin(s\phi))$
	\end{center}
	Отсюда следует, чтобы получить действительное решение, постонные надо взять комлексно-сопряжёнными: $C_{1,2} = U(\cos\theta \pm i\sin\theta)$.\\
	В итоге, получается:
	\begin{center}
		$\hat{u}_s = 2Ur^s\cos(s\phi+\theta)$
	\end{center}
	где $r = 2\sqrt{2}, \phi = \dfrac{\pi}{4}$\\\\
	Таким образом, общее решение будет выглядеть следующим образом:
	\begin{center}
		$\breve{u}_s = C_1 + 2^sC_2 + 2U(2\sqrt{2})^s\cos(\dfrac{\pi}{4}s+\theta)$
	\end{center}

	\newpage
	
	\section*{Задача 10}
	\textit{Задание:} Найти  общее решение неоднородного линейного разностного уравнения первого порядка методом итераций и методом обратного оператора при помощи z-преобразования:
	\begin{center}
		$u_{s+1} = \dfrac{6}{7}u_s + \dfrac{s+6}{s+7}$
	\end{center}
	\subsection*{Метод итераций}
	Пусть дано $u_0$:\\
	
	$u_1 = \dfrac{6}{7}u_0 + \dfrac{7}{8}$\\
	
	$u_2 = \dfrac{6}{7}u_1 + \dfrac{8}{9} = \dfrac{6}{7}\left(\dfrac{6}{7}u_0 + \dfrac{7}{8}\right) + \dfrac{8}{9} = \left(\dfrac{6}{7}\right)^2u_0 + \dfrac{6 \cdot 7}{7 \cdot 8} + \dfrac{8}{9}$\\\\
	
	$u_3 = \dfrac{6}{7}u_2 + \dfrac{9}{10} = \dfrac{6}{7}\left(\left(\dfrac{6}{7}\right)^2u_0 + \dfrac{6 \cdot 7}{7 \cdot 8} + \dfrac{8}{9}\right) + \dfrac{9}{10} =$\\
	\[
		\qquad = \left(\dfrac{6}{7}\right)^3u_0 + \left(\dfrac{6}{7}\right)^2\cdot\dfrac{7}{8} + \dfrac{6}{7}\cdot\dfrac{8}{9} + \dfrac{9}{10}
	\]
	\newline
	
	$u_4 = \dfrac{6}{7}u_3 + \dfrac{10}{11} = \dfrac{6}{7}\left(\left(\dfrac{6}{7}\right)^3u_0 + \left(\dfrac{6}{7}\right)^2\cdot\dfrac{7}{8} + \dfrac{6}{7}\cdot\dfrac{8}{9} + \dfrac{9}{10}\right) + \dfrac{10}{11} =$\\
	\[
		\qquad = \left(\dfrac{6}{7}\right)^4u_0 + \left(\dfrac{6}{7}\right)^3\cdot\dfrac{7}{8} + \left(\dfrac{6}{7}\right)^2\cdot\dfrac{8}{9} + \dfrac{6}{7}\cdot\dfrac{9}{10} + \dfrac{10}{11}
	\]
	
	\begin{center}
		$\ldots$
	\end{center}
	Общая формула при $s \geq 1$:\\\\
	$u_s = \left(\dfrac{6}{7}\right)^su_0 + \smashoperator[r]{\sum_{i=0}^{s-1}} \left(\dfrac{6}{7}\right)^i \dfrac{(s-i) + 6}{(s-i) + 7}$
	\newpage
	\subsection*{Z-преобразование}
	1) Применим z-преобразование к рассматриваемому уравнению, заменяя $u_s$ на $\tilde{u}(z)$, $u_{s-1}$ на $z^{-1}(\tilde{y}(z)+y_{-1}z)$.
	\begin{center}
		$Z\{u_s\} = \tilde{u}(z) = \dfrac{6}{7}z^{-1}(u_{-1}z + \tilde{u}(z))+\tilde{x}(z)$\\
	\end{center}

	2) Из полученного выражения выразим $\tilde{u}(z)$.
	\begin{center}
		$\tilde{u}(z) = \dfrac{6}{7}z^{-1}u_{-1}z + \dfrac{6}{7}z^{-1}\tilde{u}(z)+\tilde{x}(z)$\\
		$\tilde{u}(z) - \dfrac{6}{7}z^{-1}\tilde{u}(z) = \dfrac{6}{7}u_{-1} + \tilde{x}(z)$\\
		$\tilde{u}(z) (1 - \dfrac{6}{7}z^{-1}) = \dfrac{6}{7}u_{-1} + \tilde{x}(z)$\\
		$\tilde{u}(z) = \dfrac{\dfrac{6}{7}u_{-1} + \tilde{x}(z)}{1 - \dfrac{6}{7}z^{-1}}$\\
		$\tilde{u}(z) = \dfrac{\dfrac{6}{7}u_{-1} + \tilde{x}(z)}{z^{-1}(z - \dfrac{6}{7})}$\\
		$\tilde{u}(z) = \dfrac{\dfrac{6}{7}u_{-1}z + z\tilde{x}(z)}{z - \dfrac{6}{7}}$\\
	\end{center}
	3) Разложим $\tilde{u}(z)$ на простые дроби.\\
	Получившееся выражение уже является простой дробью, так что ничего делать не надо.\\
	
	4) Выполним обратное z-преобразование.\\
	Воспользуемся способом в котором коэффициенты берутся из ряда Лорана.
	\begin{center}
		$u_s = \smashoperator[r]{\sum_{\zeta_1=\frac{6}{7}}} Res\dfrac{\left(\dfrac{6}{7}u_{-1} + \tilde{x}(z)\right)zz^{s-1}}{z - \dfrac{6}{7}} = \lim_{z\to\frac{6}{7}} \left(z - \dfrac{6}{7}\right)\dfrac{\dfrac{6}{7}u_{-1}z^s + \tilde{x}(z)z^s}{z - \dfrac{6}{7}} = $
	\end{center}
		$= \left(\dfrac{6}{7}\right)^{s+1} u_{-1} + \tilde{x}\left(\dfrac{6}{7}\right)\left(\dfrac{6}{7}\right)^s$\\
	где $\tilde{x}\left(\dfrac{6}{7}\right) = \smashoperator[r]{\sum_{k=0}^{\infty}}x_k\left(\dfrac{6}{7}\right)^{s-k} = \smashoperator[r]{\sum_{k=1}^{s}}x_k\left(\dfrac{6}{7}\right)^{s-k} = \smashoperator[r]{\sum_{i=0}^{s-1}}x_{s-i}\left(\dfrac{6}{7}\right)^{i} = \smashoperator[r]{\sum_{i=0}^{s-1}}\dfrac{(s-i)+6}{(s-i)+7}\left(\dfrac{6}{7}\right)^i$
	
	\newpage
	Таким образом, мы получили следующее решение:\\
	\begin{center}
		$u_s =  \left(\dfrac{6}{7}\right)^{s+1} u_{-1} + \smashoperator[r]{\sum_{i=0}^{s-1}}\dfrac{(s-i)+6}{(s-i)+7}\left(\dfrac{6}{7}\right)^i = \left(\dfrac{6}{7}\right)^{s} u_{0} + \smashoperator[r]{\sum_{i=0}^{s-1}}\dfrac{(s-i)+6}{(s-i)+7}\left(\dfrac{6}{7}\right)^i$
	\end{center}

	\subsection*{Обратный оператор}
	Применим к исходному уравнению оператор сдвига на шаг назад ($\mathcal{B}$):
	\begin{center}
		$u_s = \dfrac{6}{7}\mathcal{B}u_s + \dfrac{s+6}{s+7}$
	\end{center}
	Выразим $u_s$:
	\begin{center}
		$u_s - \dfrac{6}{7}\mathcal{B}u_s = \dfrac{s+6}{s+7}$
	\end{center}
	\begin{center}
		$\left(1 - \dfrac{6}{7}\mathcal{B}\right)u_s = \dfrac{s+6}{s+7}$
	\end{center}
	\begin{center}
		$u_s =\left(1 - \dfrac{6}{7}\mathcal{B}\right)^{-1}\dfrac{s+6}{s+7}$
	\end{center}
	Воспользуемся свойством обратного оператора (разложением в ряд):
	\begin{center}
		$u_s = \left(1 + \dfrac{6}{7}\mathcal{B} + \left(\dfrac{6}{7}\right)^2\mathcal{B}^2 + \left(\dfrac{6}{7}\right)^3\mathcal{B}^3 + \ldots\right)\dfrac{s+6}{s+7}$
	\end{center}
	Таким образом, мы получили следующее решение:\\
	\begin{center}
		$u_s = \dfrac{s+6}{s+7} + \dfrac{6(s-1)+6}{7(s-1)+7} + \left(\dfrac{6}{7}\right)^2\dfrac{(s-2)+6}{(s-2)+7} + \ldots$
	\end{center}
	\newpage
	\section*{Задача 11}
	\textit{Задание:} Решить неоднородное разностное уравнение (решить задачу $\mbox{Коши}$):
	\begin{center}
		$y(s+2) - 9y(s+1) + 20y(s) = \cos(s) - 2^s$\\
		$y(1) = y(2) = 0$
	\end{center}
	\textit{Решение:}\\
	Общее решение неоднородного разностного уравнения:
	\begin{center}
		$y_s = \hat{y}_s + \breve{y}_s$
	\end{center}
	Составим характеристическое уравнение для однородного уравнения.\\
	Для получения общего решения однородного разностного уравнения находим фундаментальную систему решений, элементы которой ищем в видел:
	\begin{center}
		$\hat{y}(s) = Ch^s$
	\end{center}
	Находим харатеристическое уравнение:
	\begin{center}
		$y(s+2) - 9y(s+1) + 20y(s) = 0$
	\end{center}
	\begin{center}
		$H(h) = h^2 - 9h + 20 = 0$
	\end{center}
	Решаем уравнение, получим следующие корни: $h_1 = 4$, $h_2 = 5$.\\
	Можно видеть, что у нас нет кратных вещественных корней и нет комплексных корней, значит мы можем записать общее решение однородного уравнения:
	\begin{center}
		$\hat{y}(s) = 4^sC_1 + 5^sC_2$
	\end{center}
	где $C_1$ и $C_2$ -- произвольные константы.\\
	
	Найдём частное решение уравнения. Запишем уравнение в операторной форме:
	\begin{center}
		$a(\mathcal{B}) y(s) = \cos(s) - 2^s$
	\end{center}
	где $a(\mathcal{B})$ -- операторный полином, $\mathcal{B}$ -- оператор сдвига на шаг назад.\\
	
	Тогда формальное частное решение будет выглядеть следующим образом:
	\begin{center}
		$\breve{y}_s = a^{-1}(\mathcal{B}) (\cos(s) - 2^s)$
	\end{center}
	\newpage
	\begin{center}
		$\breve{y}_s = (1 - 9\mathcal{B} + 20\mathcal{B}^2)^{-1} (\cos(s) - 2^s)$
	\end{center}
	Так как у нас получается обратный полином второй степени и $h_1$, $h_2$ -- действительны и различны, то корни $\zeta_{1,2} = \dfrac{1}{h_{1,2}}$, обратного характеристического полинома $a(\zeta) = 0$ также будут действительны и различны.
	Тогда можно сделать замену: $a(\zeta) = -a_2(\zeta-\zeta_1)(\zeta-\zeta_2)$\\
	Таким образом:
	\begin{center}
		$\breve{y}_s = -\dfrac{1}{a_2} \dfrac{1}{(\zeta-\zeta_1)(\zeta-\zeta_2)} x_s$
	\end{center}
	Разложим $a^{-1}(\zeta) на простые дроби:$\\
	\begin{center}
		$a^{-1}(\zeta) = -\dfrac{1}{a_2} \dfrac{1}{(\zeta-\zeta_1)(\zeta-\zeta_2)} = -\dfrac{1}{a_2} \left(\dfrac{g_1}{\zeta-\zeta_1} + \dfrac{g_2}{\zeta-\zeta_2}\right)$
	\end{center}
	Постоянные $g_1$ и $g_2$ определяеются из соотношения:
	\begin{center}
		$1 = g_1(\zeta-\zeta_2) + g_2(\zeta-\zeta_1)$
	\end{center}
	т.е. при $\zeta = \zeta_1$: $g_1 = \dfrac{1}{\zeta_1-\zeta_2}$, а при $\zeta = \zeta_2$: $g_2 = \dfrac{1}{\zeta_2-\zeta_1}$
	
	Таким образом,
	\begin{center}
		$\breve{y}_s = -\dfrac{1}{a_2} \left(\dfrac{1}{(\zeta_1-\zeta_2)(\zeta-\zeta_1)} + \dfrac{1}{(\zeta_2-\zeta_1)(\zeta-\zeta_2)}\right) x_s$
	\end{center}
	и, переобозначая $\zeta = \mathcal{B}, \mathcal{B}_1 = \dfrac{\mathcal{B}}{\zeta_1}, \mathcal{B}_2 = \dfrac{\mathcal{B}}{\zeta_2}$, получаем:
	\begin{center}
		$\breve{y}_s = \dfrac{1}{a_2} \left(\dfrac{(1-\mathcal{B}_1)^{-1}}{(\zeta_1-\zeta_2)\zeta_1}+\dfrac{(1-\mathcal{B}_2)^{-1}}{(\zeta_2-\zeta_1)\zeta_2}\right) x_s$
	\end{center}

	Учитывая, что $(1-\mathcal{B}_i)^{-1} = (1+\mathcal{B}_i+\mathcal{B}_i^2+\dots)$, $i=1,2$, выражение в больших круглых скобках имеет вид:
	\begin{center}
		$\left(\dfrac{1/\zeta_1 + \mathcal{B}/\zeta_1^2 + \mathcal{B}^2/\zeta_1^3 + \dots}{\zeta_1 - \zeta_2} + \dfrac{1/\zeta_2 + \mathcal{B}/\zeta_2^2 + \mathcal{B}^2/\zeta_2^3 + \dots}{\zeta_2 - \zeta_1}\right) = \dfrac{1}{\zeta_1-\zeta_2} \smashoperator[r]{\sum_{k=0}^{s-1}}\left(\dfrac{1}{\zeta_1^k} - \dfrac{1}{\zeta_2^k}\right)\mathcal{B}^{k-1}$
	\end{center}
	\newpage
	Окончательно, получаем частное решение уравнения:
	\begin{center}
		$\breve{y}_s = \dfrac{1}{a_2} \dfrac{h_1 h_2}{h_2-h_1}\smashoperator[r]{\sum_{k=0}^{s-1}}\left(h_1^k-h_2^k\right)x_{s-(k-1)} = $
	\end{center}
	\begin{center}
		$ = \dfrac{1}{20} \dfrac{4 \times 5}{5-4}\smashoperator[r]{\sum_{k=0}^{s-1}}\left(4^k-5^k\right)(\cos(s-(k-1))-2^{s-(k-1)}) = $
	\end{center}
	\begin{center}
		$ = \smashoperator[r]{\sum_{k=0}^{s-1}}\left(4^k-5^k\right)(\cos(s-(k-1))-2^{s-(k-1)})$
	\end{center}
	Таким образом, общее решение неоднородного разностного уравнения будет выглядеть следующим образом:
	\begin{center}
		$y_s = \hat{y}_s + \breve{y}_s = 4^sC_1 + 5^sC_2 + \smashoperator[r]{\sum_{k=0}^{s-1}}\left(4^k-5^k\right)(\cos(s-(k-1))-2^{s-(k-1)})$
	\end{center}
	Найдём значения $C_1$ и $C_2$, зная, что $y(1) = y(2) = 0$:
	\begin{center}
		$\begin{cases}
			y(1) = y_1 = 4C_1 + 5C_2 = 0 \\
			y(2) = y_2 = 16C_1 + 25C_2 + (4-5)(\cos(2) - 4) = 0\\
		\end{cases}$
	\end{center}
	Решение системы:
	\begin{center}
		$C_1 = \dfrac{1}{4}(4-\cos(2)), C_2 = \dfrac{1}{5}(-4+\cos(2))$
	\end{center}
	Таким образом, общее решение примет вид:
	\begin{center}
		$y_s = \smashoperator[r]{\sum_{k=0}^{s-1}}\left(4^k-5^k\right)(\cos(s-(k-1))-2^{s-(k-1)})$
	\end{center}
	\newpage
	\section*{Задача 12}
	\textit{Задание:} Модель делового цикла Самуэльсона-Хикса предполагает прямую
	пропорциональность объемов инвестирования приросту национального
	дохода (принцип акселерации), т.е. $I_t = V(X_{t-1} - X_{t-2})$, где коэффициент $\mbox{V > 0}$ -- фактор акселерации, $I_t$ - величина инвестиций в период $t$, $X_{t-1}, X_{t-2}$ величины
	национального дохода соответственно в $(t-1)$-ом и $(t-2)$-ом периодах.
	Предполагается также, что спрос на данном этапе $C_t$ зависит от величины
	национального дохода на предыдущем этапе $X_{t-1}$ линейным образом $Ct = aX_{t-1} + b$. Условие равенства спроса и предложения имеет вид $X_t = I_t + C_t$. Тогда
	приходим к уравнению Хикса:
	\begin{center}
		$X_t = (a+V)X_{t-1} - VX_{t-2} + b$
	\end{center}
	Стационарная последовательность $X_t^* = c = const$ является решением
	уравнения Хикса только при $c = b(1-a)^{-1}$, множитель $(1 - a)^{-1}$ называется
	мультипликатором Кейнса (одномерный аналог матрицы полных затрат).\\
	Рассмотреть уравнение Хикса. Какова динамика роста национального
	дохода?\\
	При $V = a = \dfrac{1}{2}, b = 1$ уравнение Хикса будет иметь вид:\\
	\begin{center}
		$X_t = X_{t-1} - \dfrac{1}{2}X_{t-2} + 1$
	\end{center}
	\begin{center}
		$X_t - X_{t-1} + \dfrac{1}{2}X_{t-2} = 1$
	\end{center}
	\textit{Решение:}\\
	Для получения общего решения однородного разностного уравнения находим фундаментальную систему решений, элементы которой ищем в виде:
	\begin{center}
		$\hat{X}_t = Ch^t$
	\end{center}
	Находим харатеристическое уравнение:
	\begin{center}
		$H(h) = h^2 - h + \dfrac{1}{2} = 0$
	\end{center}
	Решаем уравнение, получим следующие корни: $h_1 = \dfrac{1}{2}-\dfrac{i}{2}$, $h_2 = \dfrac{1}{2} + \dfrac{i}{2}$.\\
	Можно видеть, что у нас нет кратных корней, но есть комлексные корни.
	\newpage
	Запишем представление комлексного числа в тригонометрической форме:
	\begin{center}
		$\hat{X}_t = r^t((C_1+C_2)\cos(t\phi)+(C_1-C_2)\sin(t\phi))$
	\end{center}
	где $r = \dfrac{1}{\sqrt{2}}$\\
	Таким образом, общее однородное решение уравнения можно представить в виде:
	\begin{center}
		$\hat{X}_t = \left(\dfrac{1}{\sqrt{2}}\right)^t((C_1+C_2)\cos(t\phi)+(C_1-C_2)\sin(t\phi))$
	\end{center}
	Найдём частное решение уравнения. Запишем уравнение в операторной форме:
	\begin{center}
		$\breve{X}_t = a^{-1}(\mathcal{B}) a_0 = (1-a_1\mathcal{B} - a_2\mathcal{B}^2)^{-1}a_0 = (1 - (a_1\mathcal{B} + a_2\mathcal{B}^2))^{-1}a_0 = a_0 \smashoperator[r]{\sum_{k=1}^{\infty}}(a_1+a_2)^k$
	\end{center}
	где $a(\mathcal{B})$ -- операторный полином, $\mathcal{B}$ -- оператор сдвига на шаг назад.\\
	Таким образом, общее частное решение уравнения можно представить в виде:
	\begin{center}
		$\breve{X}_t = \smashoperator[r]{\sum_{k=0}^{\infty}}\left(-1 + \dfrac{1}{2}\right)^k = \smashoperator[r]{\sum_{k=0}^{\infty}}\left(-\dfrac{1}{2}\right)^k = \dfrac{2}{3}$
	\end{center}
	В итоге, общее решение уравнения примет следующую форму:
	\begin{center}
		$X_t = \hat{X}_t + \breve{X}_t = \left(\dfrac{1}{\sqrt{2}}\right)^t((C_1+C_2)\cos(t\phi)+(C_1-C_2)\sin(t\phi)) + \dfrac{2}{3}$
	\end{center}
	\subsection*{Выводы}
	Таким образом, данное решение осциллирует (колеблется), так как у нас присутствует $\cos(t\phi)$ и $\sin(t\phi)$ и при достаточно больших $t$ наше решение будет асимптотически будет сходиться к $\dfrac{2}{3}$.
	
	\newpage
	\section*{Задача 13}
	\textit{Задание:} Рассмотрим путинную модель рынка. Предположим, что спрос и предложение задаются линейными функциями, но спрос заисит от цены в данный момент времени, а предложение зависит от цены на предыдущем этапе, т.е.
	\begin{center}
		$d_t = a - b p_t$ (функция спроса)
	\end{center}
	\begin{center}
		$s_t = m + n p_{t-1}$ (функция предложения)
	\end{center}
	где $a, b, m, n$ -- положительные действительные числа.\\
	Считая, $s_t = d_t$ получаем линейное разностное уравнение первого порядка с постоянными коэффициентами,\\
	\begin{center}
		$a - b p_t = m + n p_{t-1}$
	\end{center}
	здесь $a = 1, b = \dfrac{3}{2}, m = \dfrac{1}{2}, n = \dfrac{1}{4}$
	\begin{center}
		$1 - \dfrac{3}{2} p_t = \dfrac{1}{2} + \dfrac{1}{4} p_{t-1}$
	\end{center}
	\begin{center}
		$\dfrac{3}{2} p_t + \dfrac{1}{4} p_{t-1} = \dfrac{1}{2}$
	\end{center}
	\textit{Решение:}\\
	Для получения общего решения однородного разностного уравнения находим фундаментальную систему решений, элементы которой ищем в виде:
	\begin{center}
		$\hat{p}_t = Ch^t$
	\end{center}
	Находим харатеристическое уравнение:
	\begin{center}
		$H(h) = \dfrac{3}{2}h + \dfrac{1}{4} = 0$
	\end{center}
	Решаем уравнение, получим следующее решение: $h = -\dfrac{1}{6}$.\\
	Можно видеть, что у нас нет кратных вещественных корней и нет комплексных корней, значит мы можем записать общее решение однородного уравнения:
	\begin{center}
		$\hat{p}_t = \left(-\dfrac{1}{6}\right)^tC$
	\end{center}
	Найдём частное решение уравнения. Запишем уравнение в операторной форме:
	\begin{center}
	$\breve{p}_t = a^{-1}(\mathcal{B}) a_0 = (1 + a_1\mathcal{B})^{-1}a_0 = (1 - (-a_1\mathcal{B}))^{-1}a_0 = a_0 \smashoperator[r]{\sum_{k=1}^{\infty}}(-a_1)^k$
	\end{center}
	где $a(\mathcal{B})$ -- операторный полином, $\mathcal{B}$ -- оператор сдвига на шаг назад.\\
	Таким образом, общее частное решение уравнения можно представить в виде:
	\begin{center}
		$\breve{p}_t = \smashoperator[r]{\sum_{k=0}^{\infty}}\left(-\dfrac{1}{4}\right)^k = \dfrac{4}{5}$
	\end{center}
	В итоге, общее решение уравнения примет следующую форму:
	\begin{center}
		$p_t = \hat{p}_t + \breve{p}_t = \left(-\dfrac{1}{6}\right)^tC + \dfrac{4}{5}$
	\end{center}
	\subsection*{Выводы}
	Таким образом, данное решение при достаточно больших $t$ будет асимптотически сходиться к $\dfrac{4}{5}$.
\end{document}