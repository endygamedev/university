\documentclass[14pt,fleqn]{extarticle}
\usepackage[T2A,T1]{fontenc}
\usepackage[utf8]{inputenc}
\usepackage[russian]{babel}
\usepackage{amsmath}
\usepackage{graphicx}
\usepackage{tabularx}
\usepackage{boldline}
\usepackage{makecell}
\usepackage{arydshln}
\usepackage{mathtools}
\usepackage{enumitem}
\usepackage[a4paper, total={6.5in, 9.5in}]{geometry}

\graphicspath{ {./images/} }
\setlength{\mathindent}{0pt}
\setlength\parindent{0pt}

\def\at{
	\left.
	\vphantom{\int}
	\right|
}

\begin{document}
	\begin{titlepage}
		\includegraphics[scale=0.12]{logo}
		\begin{center}
			\textbf{МИНОБРНАУКИ РОССИИ}\\
			\vspace{0.2cm}
			\textbf{Федеральное государственное бюджетное образовательное учреждение высшего образования}\\
			\textbf{«САНКТ-ПЕТЕРБУРГСКИЙ ГОСУДАРСТВЕННЫЙ ЭКОНОМИЧЕСКИЙ УНИВЕРСИТЕТ»}\\
			\vspace{0.6cm}
			Факультет информатики и прикладной математики\\
			Кафедра прикладной математики и экономико-математических методов\\
			\vspace{1cm}
			\textbf{ОТЧЁТ}\\
			по дисциплине:\\
			\textbf{«Теория и системы поддержки принятия решений»}\\
			на тему:\\
			\textbf{«Представление функций рядом Фурье. Задание 4»}\\
		\end{center}
		\vspace{1cm}
		Направление: 01.03.02\\
		Обучающийся: Бронников Егор Игоревич\\
		Группа: ПМ-1901\\
		\vfill
		\begin{center}
			Санкт-Петербург\\
			2022\\
		\end{center}
	\end{titlepage}
	
	\section*{Задача 14}
	\textit{Задание:} Разложить в ряд Фурье периодическую функцию $f(x)$ с периодом $2l$, заданную в интервале $(-l, l)$:
	\begin{center}
		$l = \pi, \hspace{0.5cm} f(x) = \pi + x$
	\end{center}

	\textit{Решение:}\\
	Разложение в ряд Фурье на интервале $(-\pi, \pi)$ имеет вид:
	\begin{center}
		$f(x) = \dfrac{a_0}{2} + \smashoperator[r]{\sum_{n=1}^{\infty}}a_n \cdot \cos\left(\dfrac{\pi n x}{\pi}\right) + b_n \cdot \sin\left(\dfrac{\pi n x}{\pi}\right)$
	\end{center}
	\[ a_0 = \dfrac{1}{\pi} \int_{-\pi}^{\pi} f(x) \,dx \]
	\[ a_n = \dfrac{1}{\pi} \int_{-\pi}^{\pi} f(x) \cos\left(\dfrac{\pi n x}{\pi}\right) \,dx = \dfrac{1}{\pi} \int_{-\pi}^{\pi} f(x) \cos\left(n x\right) \,dx \]
	\[ b_n = \dfrac{1}{\pi} \int_{-\pi}^{\pi} f(x) \sin\left(\dfrac{\pi n x}{\pi}\right) \,dx = \dfrac{1}{\pi} \int_{-\pi}^{\pi} f(x) \sin\left(n x\right) \,dx \]
	Таким образом, подставляя исходные данные, получаем:
	\[ \boldsymbol{a_0} = \dfrac{1}{\pi} \int_{-\pi}^{\pi} (\pi + x) \,dx = \dfrac{1}{\pi} \left(\dfrac{x^2}{2} + \pi x\right) \at_{-\pi}^{\pi} = \dfrac{1}{\pi} \left(\dfrac{3\pi^2}{2} - \left(-\dfrac{\pi^2}{2}\right)\right) = 2\pi \]
	\vspace{1cm}	
	\[ \boldsymbol{a_n} = \dfrac{1}{\pi} \int_{-\pi}^{\pi} (\pi + x) \cos\left(n x\right) \,dx = \dfrac{1}{\pi} \left(\dfrac{x \sin(nx)}{n} + \dfrac{\pi \sin(nx)}{n} + \dfrac{\cos(nx)}{n^2}\right) \at_{-\pi}^{\pi} = \]
	\[	= \dfrac{1}{\pi} \left(2\pi\cdot\dfrac{\sin(\pi n)}{n} + \dfrac{\cos(\pi n)}{n^2} - \dfrac{\cos(\pi n)}{n^2}\right) = \dfrac{2\sin(\pi n)}{n} = 0 \]
	\vspace{1cm}
	\[ \boldsymbol{b_n} = \dfrac{1}{\pi} \int_{-\pi}^{\pi} (\pi + x) \sin\left(n x\right) \,dx = \dfrac{1}{\pi} \left(-\dfrac{x \cos(nx)}{n} - \dfrac{\pi \cos(nx)}{n} + \dfrac{\sin(nx)}{n^2}\right) \at_{-\pi}^{\pi} = \]
	\[	= \dfrac{1}{\pi} \left(-2\pi\cdot\dfrac{\cos(\pi n)}{n} + \dfrac{\sin(\pi n)}{n^2} + \dfrac{\sin(\pi n)}{n^2}\right) = -2\cdot\dfrac{\cos(\pi n)}{n}= -2\dfrac{(-1)^n}{n}\]
	\newpage
	Окончательно, получаем искомое разложение:
	\begin{center}
		$f(x) = \pi + \smashoperator[r]{\sum_{n=1}^{\infty}}\left(-2\dfrac{(-1)^n}{n}\right)\cdot\sin(nx)$
	\end{center}
	Построим график полученного ряда и исходной функции:
	\begin{center}
		\includegraphics[scale=0.5]{plot}
	\end{center}
	Также можно записать равенство Парсеваля. Оно в общем случае принимает следующую форму:
	\[ \dfrac{1}{\pi} \int_{-\pi}^{\pi} (f(x))^2 \,dx = \dfrac{a_0^2}{2}+\sum_{n=1}^{\infty}(a_n^2+b_n^2) \]
	где $a_0, a_n, b_n$ -- коэффициенты ряда Фурье\\
	Если мы подставим полученные значения коэффициентов ряда Фурье, то равенство Парсеваля запишется так:
	\[ \dfrac{1}{\pi} \int_{-\pi}^{\pi} (\pi+x)^2 \,dx = 2 \pi^2+\sum_{n=1}^{\infty}\left(\dfrac{4}{n^2}\right) \]
	\newpage
	\section*{Задача 15}
	Применим теперь дискретное преобразование Фурье:\\
	\begin{center}
		$f_F(y_k) = \smashoperator[r]{\sum_{l=0}^{N-1}} f\left(\dfrac{l}{N}\right) \cdot e^{\dfrac{-2\pi \cdot i \cdot y_k}{N}}$
	\end{center}
	Получим следующий результат:
	\begin{center}
		\includegraphics[scale=0.5]{fourier}
	\end{center}
\end{document}